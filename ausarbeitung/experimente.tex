\chapter{Experimente und Auswertung}
\label{c:experimente}
In diesem Kapitel werden wir das Adaptionsverhalten der beschriebenen Verfahren untersuchen. Dabei werden wir im Detail einzelne Aspekte der Verfahren genauer
betrachten und ihre Auswirkungen im Gesamtkontext darstellen. Wir bedienen uns dabei der in Kapitel \ref{c:implementierung} vorgestellten Simulationsumgebung. Da
wir keine Vergleichsm�glichkeiten mit anderen Verfahren haben, werden wir ausschlie�lich das System durch Modifikation verschiedener Parameter bzw. Algorithmen
``in sich'' untersuchen. Die Ergebnisse der empirischen Untersuchungen dienen dabei zur Best�tigung der Ideen, die unsere Entwicklung der Verfahren motiviert haben.

\section{Aufbau der Experimente:}
Damit die Experimente untereinander vergleichbar sind, haben wir eine einheitliche Parameterwahl getroffen. Parameter wurden nur dann gezielt modifiziert,
wenn dies f�r das jeweilige Experiment von entscheidender Bedeutung war.

\paragraph{Topologien:}
Die Simulationsumgebung besitzt eingebaute Topologien (z. B. $Topology\-One\-Sur\-roun\-ded$), welche gut f�r eine visuelle Kontrolle der Algorithmen geeignet
sind, da sie von ihrer Struktur her einfach und �bersichtlich aufgebaut sind. Um jedoch aussagekr�ftige Ergebnisse zu erhalten, die auch in Hinsicht auf
Netzwerktopologien, so wie sie im Internet vorzufinden sind, als realistisch eingestuft werden k�nnen, m�ssen wir andere Topologien heranziehen. Wir bedienen uns
Topologien, welche auf dem Transit-Stub-Modell \cite{Zegura1996} basieren. Dieses Modell spiegelt sehr gut reale Internetstrukturen wider. Bei diesem Modell
besteht das Netzwerk aus mehreren Dom�nen, die entweder vom Typ ``Stub-Dom�ne'' oder ``Transit-Dom�ne'' sind. W�hrend der Datenverkehr nur dann durch eine
Stub-Dom�ne flie�t, wenn der Ziel- bzw. Ausgangsknoten innerhalb dieser Stub-Dom�ne liegt, besteht diese Einschr�nkung f�r Transit-Dom�nen nicht. Transit-Dom�nen
dienen somit dazu, Stub-Dom�nen miteinander zu verbinden und leiten den Datenverkehr weiter. F�r die Stub-Dom�nen bilden sie Backbones.\\

Es gibt verschiedene Tools, um Topologien basierend auf dem Transit-Stub-Modell zu generieren. Eines davon ist BRITE \cite{Medina:2001BRITE}, welches hier Verwendung
fand, um die notwendigen Topologien zu generieren. F�r die Simulation sind zwei Topologien notwendig: eine Sublayer-Topologie, welche die physischen Verbindungen
zwischen den Knoten darstellt und eine Toplayer-Topologie, welche das Overlay-Netzwerk repr�sentiert. Bei allen Experimenten wurde eine Sublayer-Topologie
bestehend aus 2000 Knoten verwendet. Das Overlay-Netzwerk bestimmt sich dann aus den Broker-Knoten (hier 200 Knoten), welche fest gew�hlt wurden, und den
Subscriber-Knoten, deren Zahl sich aus der H�lfte der verbleibenden Knoten bestimmt (also 900) und die zuf�llig den einzelnen Brokern zugewiesen wurden, jedoch so,
dass jeder Broker in etwa gleich viele Subscriber verwaltet. Die �brigen Knoten sind lediglich Transfer-Knoten, welche f�r die Weiterleitung des Datenverkehrs
zust�ndig sind.\\

Da ein einziger Durchlauf eines Experimentes m�glicherweise keine repr�sentativen Ergebnisse liefert, wurde jedes Experiment 30 Mal mit unterschiedlichen
Seed-Werten (Ausgangswerte f�r Zufallsfunktion) durchgef�hrt. F�r jeden gemessenen Wert wurden aus den verschiedenen Ergebnissen Mittelwerte und Konfidenzintervalle
berechnet.
