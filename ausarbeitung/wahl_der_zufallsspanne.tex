\subsubsection{Wahl der Zufallsspanne}
Es stellt sich die Frage, wie die Gr��e der Zufallsspanne $\Delta Z$ bestimmt werden soll. Zun�chst werden wir darstellen, wie $\Delta Z$, die Anzahl
der Subscriber und die Anzahl der Anfragen an den RSS-Server zusammenh�ngen. Bei konstanter Subscriberanzahl im Overlay-Netzwerk bewirkt eine
Verringerung von $\Delta Z$, dass potenziell mehr Subscriber innerhalb dieser festen Zeitspanne Anfragen an den Server stellen (es muss sich
die gleiche Anzahl von geplanten Anfragen auf engerem Raum verteilen). Das hei�t, die
zeitlichen Differenzen der einzelnen potenziellen Anfragen sind geringer. Dies verringert die Chance, dass �ber das Brokersystem verteilte
Feeds potenzielle Anfragen bei Subscribern unterbinden. Entsprechend Umgekehrtes geschieht bei einer Vergr��erung von $\Delta Z$. Dass dem
wirklich so ist, zeigen statistische Analysen in der verwendeten Simulation. Eine Vergr��erung von $\Delta Z$ bewirkt also eine Verringerung
der sich im Netz befindlichen Nachrichten, da �berfl�ssige �bermittlungen der Feeds durch den RSS-Server unterbleiben bzw. abnehmen. Dies kann aber auch
bedeuten, dass der Aktualit�tsgrad der Feeds abnimmt, da die maximale zeitliche Differenz zwischen Aktualisierung eines Feeds auf Server-Seite
und dessen �bermittlung an den Subscriber gr��er wird.\\
Es h�ngt also stark vom Interesse der Klienten bzw. von der angestrebten Dienstg�te ab, nach welchen Kriterien $\Delta Z$
angepasst werden soll. Da sich die Anzahl der Subscriber w�hrend der Laufzeit
des Systems stark �ndern kann, muss sich $\Delta Z$ eventuell adaptierend verhalten.\\

Wir definieren zwei unterschiedliche Ziele, nach denen die Dienstg�te bestimt werden soll:
\begin{description}
\item [Geringe Netzbelastung:] Der RSS-Server soll so wenig wie m�glich durch Anfragen belastet werden und �berfl�ssiger Datenverkehr sollte weitesgehend vermieden werden.
\item [Bevorzugte Polling-Periode:] Dem Wunsch eines Klienten, den neuesten RSS-Feed nach sp�testens einer von ihm vorgegebenen
        Zeitspanne zu erhalten, soll entsprochen werden.
\end{description}

Sicherlich kann auch ein Kompromiss zwischen diesen unterschiedlichen Zielen angestrebt werden.
Um diese Arbeit in einem angemessenen Rahmen zu halten, konzentrieren wir uns jedoch ausschlie�lich auf das Ziel der \glqq bevorzugten Polling-Periode\grqq{}.

%%% Local Variables: 
%%% mode: latex
%%% TeX-master: "diplomarbeit"
%%% End: 
