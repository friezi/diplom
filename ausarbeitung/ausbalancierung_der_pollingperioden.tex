\subsection{Ausbalancierung der Polling-Perioden}
\label{cs:ausbalancierung_der_polling-perioden}
Wie wir schon in Abschnitt \ref{cssp:ung_u_probl:dysbalance} auf Seite \pageref{cssp:ung_u_probl:dysbalance} beschrieben haben, kann es vorkommen, dass Subscriber
aufgrund von Dysbalancen der Polling-Perioden im System ``ausgesperrt'' werden: Subscriber, deren Polling-Periode sehr niedrig ist, haben fast ausschlie�lichen
Zugriff auf den Server, w�hrend die Polling-Perioden der �brigen Subscriber unaufh�rlich in die H�he schie�en, da ihre Anfragen keinen Zugang zur Server-Queue
finden. Um diesem Problem beizukommen, m�ssen bei Dysbalancen im System Subscriber mit niedrigen Polling-Perioden diese erh�hen, w�hrend Subscriber mit hohen
Polling-Perioden Gelegenheit bekommen m�ssen, diese zu erniedrigen.\\

Um dieses Ziel zu erreichen, verwenden wir folgendes Verfahren: speist ein Subscriber einen RSS-Feed in das Notifikationssystem ein (er sendet den Feed an den
ihm zugewiesenen Broker), so �bermittelt er als zus�tzliches Attribut den von ihm gemessenen $rtt$. Erh�lt ein Subscriber einen RSS-Feed vom Notifikationssystem,
so stellt sein neuer $rtt$-Wert den Mittelwert aus seinem eigenen $rtt$ und dem �bermitteltem $rtt$ dar, also:
\[rtt:=\frac{rtt+feed.rtt}{2}\]
Es handelt sich dabei um ein kooperatives Modell: die gegenseitige Unterst�tzung der Subscriber wird vorausgesetzt, ein gemeinsames Ziel verlangt die Einschr�nkung
des einzelnen.\\

Das Verfahren f�hrt dazu, dass Subscriber mit einer hohen Polling-Periode, die einen Feed mit einem niedrigen $rtt$ erhalten, ihre Polling-Periode senken,
allerdings nicht auf das Niveau des Subscribers, welcher den Feed ausgesandt hat. Somit wird eine zu schnelle �berlastung des Systems vermieden.
Im Gegenzug wird ein Subscriber mit einer gesenkten Polling-Periode schneller in den Genuss einer Server-Antwort kommen, so dass er ebenfalls einen Feed mit
seinem noch relativ hohen $rtt$ in das Notifikationssystem einspeisen kann.Das f�hrt dazu, dass Subscriber mit einer niedrigen Polling-Periode diese bei Erhalt
jenes Feeds erh�hen.\\

Dies hat noch einen weiteren positiven Nebeneffekt: angenommen, alle Subscriber haben eine hohe Polling-Periode aufgrund einer l�nger anhaltenden Server-�berlastung
eingestellt. Kommt es zu einer pl�tzlichen Reaktionsfreudigkeit des Servers, so wird der erste Subscriber, der diese feststellt, nicht nur seine Polling-Periode
entsprechend anpassen, sondern alle anderen Subscriber, die die Ansprechbarkeit des Servers nicht bemerkt haben, mitziehen. Ein pl�tzlicher �berm��iger Ansturm auf
den Server sollte aber ausbleiben, da die �brigen Subscriber ihre Polling-Perioden nicht drastisch, sondern nur allm�hlich senken. Die Subscriber, welche ihre
Polling-Periode drastisch gesenkt haben, werden gezwungen sein, ihre Polling-Periode wieder anzuheben. Dadurch kommt es zu einer homogeneren Adaption der
Polling-Perioden an die Server-Auslastung. Um es bildlich auszudr�cken: diejenigen Subscriber,
welche das Ziel sehen (``Server reagiert schnell'') rennen los und rufen den anderen zu: ``Los, kommt mit!'', wogegen diese Antworten: ``Moment, wir k�nnen nicht so
schnell!''. Entsprechend umgekehrtes gilt bei pl�tzlicher Server-�berlastung.\\

Dadurch, dass der $rtt$ und nicht der $cpp$ �bermittelt wird, kann jeder Subscriber seinen eigenen $cpp$ entsprechend seinen Voreinstellungen ($ppp$) unabh�ngig
der Voreinstellungen anderer Subscriber berechnen.\\

Die Wirksamkeit der ausbalancierenden Methode gegen�ber der nicht-ausbalancierenden Methode wurde mit Hilfe der selbst entwickelten Simulationsumgebung gezeigt,
die Ergebnisse sind in Kapitel \ref{c:experimente} zusammengetragen.\\

Wollen wir kurz die Vorteile und m�glichen Nachteile dieser Methode auflisten:
\begin{itemize}
\item Vorteile
  \begin{itemize}
  \item Polling-Perioden werden Ausbalanciert
  \item Bei unbemerkter Ver�nderung der Serverbelastung werden Subscriber mitgezogen
  \item Ansturm auf Server wird vermieden
  \end{itemize}
\item Nachteile
  \begin{itemize}
  \item Kooperatives Modell kann ausgetrickst werden: spielen einige Subscriber nicht mit, kann das System ausgehebelt werden.
  \end{itemize}
\end{itemize}

Das Verfahren erfordert auf der Ebene des Notifikationssystems eine neue Datenstruktur, um den neu hinzugekommenen Wert $rtt$ aufzunehmen. Folgende auf Java-Syntax
beruhende Datenstrukturen sind denkbar:

\begin{verbatim}
class RichRSSFeed extends RSSFeed{

    RSSFeed feed;
    RichInformation additional_information;

}

class RichInformation{

    long rtt;

}

\end{verbatim}

Der Vorteil, den $rtt$ in eine eigene Klasse zu packen, liegt darin, dass die Klasse $Rich\-Information$ erweiterbar ist um eventuelle zus�tzliche Informationen.\\

Nun muss bei jedem Erhalt eines RSS-Feeds der $cpp$ angepasst werden. Wir m�ssen au�erdem unsere Methoden zur Aktualisierung des $RQT$ anpassen. Dabei tritt ein
Problem auf: erh�lt ein Subscriber immer neue Feeds �ber das Notifikationssystem, bevor sein $RQT$ abgelaufen ist, so wird dieser immer wieder neu gesetzt, so dass
eine eigenm�chtige Aussendung eines Feed-Requests durch den Subscriber ausbleibt, falls sich der $cpp$ dabei st�ndig erh�ht. Um hier Abhilfe zu schaffen,
setzen wir den $RQT$ nur dann neu, wenn durch den neu gesetzten $RQT$ der Feed-Request fr�her ausgesendet werden w�rde. Ansonsten kommt der neue $cpp$ erst bei
der n�chsten Runde in Betracht.

\begin{verbatim}

aktualisiereRQTdurchAltenBrokerFeed {

    deltaTTR = berechneDeltaTTR(cpp);
    if ( deltaTTR < RQT.ZeitdifferenzBisAblauf )
      aktualisiereRQT(deltaTTR);

}

aktualisiereRQTdurchNeuenBrokerFeed {

    deltaTTR = berechneDeltaTTR(cpp);
    if ( deltaTTR < RQT.ZeitdifferenzBisAblauf )
      aktualisiereRQT(deltaTTR);

}

aktualisiereRQTdurchAltenServerFeed {

    aktualisiereRQT(cpp);

}

aktualisiereRQTdurchNeuenServerFeed {

    deltaTTR = berechneDeltaTTR(cpp);
    aktualisiereRQT(deltaTTR);

}


\end{verbatim}
