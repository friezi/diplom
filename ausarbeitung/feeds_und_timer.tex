\subsection{Feeds und Timer}
\label{css:feeds_und_timer}
Bisher sind wir noch nicht weiter darauf eingegangen, welche ausl�senden Faktoren f�r das Setzen der Timer $RT$ und $RQT$ verantwortlich sind. Wir haben nur den
Timer $RT$ n�her betrachtet: ausl�sendes Ereignis ist die Aussendung eines Feed-Requests an den RSS-Server. Da es sich bei einer Anfrage allerdings um eine
TCP-Verbindung handelt, soll $RT$ erst dann gesetzt werden, wenn die Verbindung abgebrochen wird, bzw. wenn sie gar nicht zustande kommt.
Das Setzen des Timer-Intervalls $rto$ erfolgt wie im vorherigen Abschnitt besprochen.\\

Bleibt zu bestimmen, wann und wie $RQT$ gesetzt wird. Erh�lt ein Subscriber einen RSS-Feed, so muss er bestimmen, wann der n�chste Feed-Request an den RSS-Server
gesendet werden soll. Dazu m�ssen wir einen erhaltenen Feed nach Aktualit�t und Herkunft unterscheiden. Erh�lt ein Subscriber einen neueren Feed von einem
Broker oder von einem RSS-Server, so
m�ssen $\varDelta Z$ und $t_x$, also $\varDelta ttr$, in Abh�ngigkeit von den enthaltenen Events entsprechend der in Abschnitt \ref{cs:der_grundlegende_algorithmus}
beschriebenen Methode neu bestimmt werden: da ein Subscriber Inhalte prinzipiell bei mehreren RSS-Anbietern abonniert haben kann, m�ssen f�r jeden
RSS-Server, auf den sich ein neues Event bezieht, diese Werte
neu berechnet und der entsprechende $RQT$ neu gesetzt werden. Erh�lt ein Subscriber dagegen einen Feed von einem Broker ohne neue Informationen,
so bleibt der laufende $RQT$ bestehen, denn dies ist genau jener Fall, in welchem es in der Verantwortung des Subscribers liegt, in K�rze hinzukommende
Ereignisse zu entdecken. Bei Erhalt eines Feeds von einem RSS-Server ohne neue Informationen muss $RQT$
neu gesetzt werden, denn es ist davon auszugehen, dass dem Feed ein Feed-Request aufgrund eines abgelaufenen $RQT$ vorausgegangen ist.
Dabei wird das $RQT$-Intervall auf $cpp$ gesetzt, $\varDelta Z$ wird nicht neu berechnet und $t_x$ wird auf $t_x:=t_0+cpp$ gesetzt.\\

Das ganze als algorithmisches Beispiel in einer an Java angelehnter Pseudonotation:
\lstset{language=Java}
\begin{lstlisting}

aktualisiere_RQT_durch_alten_Brokerfeed {
}

aktualisiere_RQT_durch_neuen_Brokerfeed {

    berechne_delta_ttr(cpp);
    setze_RQT(delta_ttr);

}

aktualisiere_RQT_durch_alten_Serverfeed {

    berechne_rtt();
    berechne_artt();
    setze_cpp(artt);
    stoppe_RT();
    setze_rto_zur�ck();
    setze_RQT(cpp);

}

aktualisiere_RQT_durch_neuen_Serverfeed {

    berechne_rtt();
    berechne_artt();
    setze_cpp(artt);
    stoppe_RT();
    setze_rto_zur�ck();
    berechne_delta_ttr(cpp);
    setze_RQT(delta_ttr);

}
\end{lstlisting}



%%% Local Variables: 
%%% mode: latex
%%% TeX-master: "diplomarbeit"
%%% End: 
