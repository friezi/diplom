\addcontentsline{toc}{section}{Glossar}
\glentry{ack}{Acknowledgement: Best�tigungsnachricht f�r erhaltenes Datenpaket}
\glentry{Z}{Zufallsintervall}
\glentry{$\varDelta Z$}{Zufallsspanne}
\glentry{$\varDelta ttr$}{Zeit zwischen $t_0$ und $ttr$}
\glentry{$\varDelta ttl$}{Zeit zwischen $t_0$ und $nextBuild$}
\glentry{$\lambda$}{mittlere Ankunftsrate der Anfragen}
\glentry{$\bar x$}{mittlere Bearbeitungszeit pro Anfrage}
\glentry{$\rho$}{utilization factor $:=\lambda\bar x$}
\glentry{$\Gamma_V$}{Funktion zur Bestimmung des Verz�gerungsgrades von RSS-Feeds}
\glentry{$\Gamma_A$}{Funktion zur Bestimmung des Aktualit�tsgrades von RSS-Feeds}
\glentry{IP}{Internet Protocol}
\glentry{TCP}{Transmission Control Protocol}
\glentry{UDP}{User Datagram Protocol}
\glentry{DHT}{Distributed Hashtable}
\glentry{ttl}{time to live: Zeitraum, indem eine Information als aktuell eingestuft wird und sich voraussichtlich nicht �ndern wird}
\glentry{ttr}{time to refresh: Zeitpunkt einer erneuten Anfrage}
\glentry{t$_x$}{Einstiegspunkt}
\glentry{t$_0$}{aktueller Zeitpunkt}
\glentry{nextBuild}{Zeitpunkt, an dem ein Server voraussichtlich neue Informationen bereitstellen wird}
\glentry{rto}{retransmission timeout intervall: Zeit bis zum erneuten Aussenden einer Anfrage}
\glentry{rtt}{N�herungswert f�r die roundtrip-time}
\glentry{artt}{skalierter rtt}
\glentry{srtt}{smoothed roundtrip time: gegl�tteter Wert bei der Berechnung des rtt zur Stauvermeidung bei TCP}
\glentry{serviceTimeFactor}{Simulationsparameter: Faktor f�r die Bearbeitungszeit pro Feed-Request; mit seiner Hilfe kann eine vor�bergehende
        vermehrte Serverbelastung bzw. ein weniger leistungsf�higer Hostrechner simuliert werden.}
\glentry{Aussendung}{Aussendung eines Datenpakets bzw. einer Anfrage an einen RSS-Server}
\glentry{Wiederholung}{Wiederholte Aussendung eines Datenpakets bzw. einer Anfrage an einen RSS-Server}
\glentry{roundtrip-time}{Zeit zwischen dem Versenden einer Anfrage und dem Erhalt der Antwort}
\glentry{Feed-Request}{eine Anfrage an den RSS-Server nach einem RSS-Feed}
\glentry{feed.artt}{$artt$ als Bestandteil eines erweiterten Feeds}
\glentry{RQT}{Request-Timer: Timer, nach dessen Ablauf ein Feed-Request ausgesandt wird}
\glentry{RT}{Retransmission-Timer: Timer, nach dessen Ablauf ein Feed-Request erneut ausgesandt wird (ohne Erhalt eines RSS-Feeds)}
\glentry{ppp}{bevorzugte Polling-Periode}
\glentry{cpp}{aktuelle Polling-Periode}
\glentry{mpp}{maximale Polling-Periode}
\glentry{icpp}{initial-cpp: der f�r die Berechnung von $rto$ und $rtt$ grundlegende Wert; wird durch einen $feed.rtt$ nicht modifiziert}

\glentry{RSS}{Relly Simple Syndication: Technik zur Bereitstellung von Kurznachrichten im WorldWideWeb}
\glentry{Blog}{auch Weblog: Webseite, die periodisch neue Eintr�ge enth�lt}
\glentry{Cache}{Puffer-Speicher: dient dem schnelleren Zugriff auf oder zur l�ngeren Verf�gbarkeit von Daten}
\glentry{Graph}{ein mathematisches Gebilde bestehend aus Knoten, die durch Kanten (ungerichtet oder gerichtet als Pfeil) verbunden sein k�nnen}
\glentry{Proxy}{Stellvertreter: Anfragen an das Internet werden zun�chst an den Proxy geleitet, der gew�hnlicherweise einen Cache unterh�lt und den
        Anfragenden mit den gw�nschten Daten versorgt oder die Anfrage weiterleitet.}
\glentry{RDF}{Resource description Framework: fromale Sprache zur Beschreibung von Webinhalten}
\glentry{SQL}{deklarative Anfragesprache f�r relationale Datenbanken}
\glentry{KTick}{1 KTick = 1000 Ticks (siehe Tick)}
\glentry{Tick}{Metrik einer Simulation: ein Tick ist die kleinste Simulationseinheit}
\glentry{URL}{Uniform Resource Locator: Zeichenkette, die eine Ressource in Computernetzwerken identifiziert}
\glentry{WWW}{World Wide Web}
\glentry{Id}{Identifier bzw. Bezeichner}
\glentry{Multicast}{das gleichzeitige Aussenden einer Nachricht an eine Menge von Empf�ngern}
\printglossary
