\documentclass{article}
\usepackage[latin1]{inputenc}
\usepackage{latexsym}
\usepackage[german]{babel}
\usepackage[a4paper]{geometry}
\pagestyle{empty}
%\usepackage{charter}
%\geometry{textwidth=17cm, textheight=22cm} 
\parindent0em

\begin{document}

Hier ein paar Anmerkungen zu den einzelnen Kapiteln. Die Gliederung ist ein erster Entwurf. Es kann sein, dass ich im Verlauf der Arbeit
�nderungen vornehmen werde in der Form, dass �berschriften wechseln, Abschnitte gestrichen werden oder Abschnitte neu hinzu kommen,
wenn das n�tig zu sein schein.

\paragraph{Kapitel 2 - 4:}
Vorstellung der relevanten Begriffe. Was ich in das Kapitel �ber Peer-To-Peer mit aufnehme (und ob es �berhaupt n�tig ist),
weiss ich noch nicht, wird sicherlich einer der letzten Punkte sein.

\paragraph{Kapitel 5:}
Hier m�chte ich genauer auf die derzeitige Problematik bei RSS eingehen. Ich definiere die Zielsetzung, aufgrund dessen ein neues Konzept
erstellt werden soll.

\paragraph{Kapitel 6:}
Hier werde ich Schritt f�r Schritt Design und Realisierung des neuen Ansatzes herleiten. Daf�r werden die relevanten Parameter eingef�hrt und
deren Bestimmung erkl�rt. Hiervon sind schon einige Abschnitte ausformuliert. Das Design spaltet sich im weiteren Verlauf in zwei Hauptrichtungen
(Abschnitte 6.6 und 6.7), wobei das Hauptaugenmerk auf 6.7 liegen soll. Ich werde �hnliche L�sungsans�tze f�r Teilprobleme referenzieren (z.B.
TCP u. a.) und darstellen, warum diese L�sungsans�tze bei der Problematik meines Themas nicht greifen.

\paragraph{Kapitel 7:}
Die Reaktion des Systems auf bestimmte ver�nderliche Parameter soll getestet und bewertet werden. Im Falle der Pr�ferenz
``bevorzugte Polling-Rate'' ist die entscheidende Frage, ob durch Anpassung der Polling-Raten der RSS-Server an seine Belastungsgrenze
gebracht wird. Dazu wird folgendes getestet:\\
\textbf{initiale Phase}: Einpegelung des Klienten-Pollings bei folgenden ver�nderbaren Parametern:
\begin{itemize}
\item Gr��e der Queue
\item maximale Polling-Rate
\item Anzahl der Subscriber
\item Reaktionszeit des Servers bzw. Serverbelastung
\end{itemize}
\textbf{laufender Betrieb:} Einpegelung des Klienten-Pollings bei folgenden ver�nderbaren Parametern:
\begin{itemize}
\item Reaktionszeit des Servers bzw. Serverbelastung
\end{itemize}

Zus�tzlich soll demonstriert werden, wie sich das Fehlen bestimmter Techniken in einigen Algorithmen auswirkt, so z. B. das Fehlen des
Ausbalancierens der Polling-Raten auf die Polling-Raten der einzelnen Subscriber. Die Algorithmen sollen also ``in sich'' getestet und bewertet
werden und nicht im Vergleich zu anderen Algorithmen, da ich keine anderen Algorithmen zu dieser Problematik referenziere.

\end{document}
