\chapter{Adaptive Informationsverteilung mit RSS und verteiltem Publish/Subscribe}
Im Folgenden wollen wir eine Methode beschreiben, ereignisbasierte Informationen an interessierte Klienten zu verteilen. Konkret orientieren wir
uns dabei an RSS. Mit ``Ereignis'' meinen wir im Folgenden eine Informationseinheit (z. B. eine Nachrichten-Schlagzeile),
mehrere Ereignisse k�nnen in einem Datenpaket (RSS-Feed) gesammelt an den Klienten �bermittelt werden. Dazu betrachten wir zun�chst, nach welchem
Schema die Informationen entsprechend des RSS-Systems verbreitet werden, zeigen vorherrschende Probleme auf und schlagen ein Konzept vor,
wie diese Probleme vermieden bzw. verringert werden k�nnen. Auch wenn wir im Folgenden �berwiegend von RSS sprechen, so l�sst sich das Konzept auch auf
andere Informationssysteme �bertragen, die bestimmte Kriterien erf�llen. Je nachdem, auf welche Ebene wir uns beziehen, werden wir entweder von ``Informationen''
oder ``RSS-Feeds'' sprechen. Das neu entwickelte System wollen wir im Folgenden ``\pubsubrss'' nennen.
 
\section{Verteiltes Polling}
Betrachten wir die Gesamtheit der Subscriber und ihr Polling-Verhalten, so erkennen wir, dass aus Sicht eines RSS-Servers die Polling-Frequenz
dieser Gesamtheit (mittlere Ankunftsrate der Anfragen) gr��er ist als die Polling-Frequenz jedes einzelnen Subscribers.
W�nschenswert w�re es, wenn jeder Subscriber von der
Polling-Frequenz der Gesamtheit profitieren k�nnte, so dass die gesteigerte Frequenz zu einer erh�hten Aktualit�t der RSS-Feeds beim entsprechenden
Subscriber f�hrt. Es bietet sich an, die Subscriber �ber ein Overlay-Netzwerk zu verbinden, so dass die RSS-Feeds zwischen den Subscribern ausgetauscht werden
k�nnen. Das Polling wird also auf die beteiligten Subscriber verteilt.
Wir haben es dabei mit einer Kombination aus einem Pull- und einem Push-Ansatz zu tun. Allerdings �bernimmt die Push-Funktion nicht der RSS-Server,
sondern sie wird von den beteiligten Einheiten des Overlay-Netzes �bernommen, von dem der RSS-Server kein Bestandteil ist. Es stellt sich die Frage,
warum wir nicht gleich den Push-Ansatz bezogen auf den RSS-Server favorisieren. RSS ist eine schon seit l�ngerer Zeit bestehende Technik.
Dar�ber hinaus ist RSS weit verbreitet.
Eine Modifikation des Grundkonzeptes w�rde f�r Anbieter, die es unterst�tzen, bedeuten, bestehende Software austauschen zu m�ssen und
ganz neue Serviceleistungen bereitstellen zu m�ssen. Dies w�rde m�glicherweise auf Ablehnung sto�en und damit nicht die gew�nschte
Verbreitung des neuen Konzeptes mit sich bringen. Ziel ist es, auf dem bestehenden Konzept aufzubauen und es in ein erweitertes Konzept zu
integrieren, um f�r den Benutzer als auch f�r den Dienstanbieter m�glichst ein Minimum an Aufwand zu erreichen (Deolasee et al.
besch�ftigen sich in \cite{bhide02adaptive} mit einer Kombination aus Push-Pull und beschreiben die Probleme, die sich aus reinen Pull- bzw.
Push-Ans�tzen ergeben).
\subsubsection*{Einbettung in PubSub}
Es ist naheliegend, das Publish-Subscribe-Kommunikationsparadigma auf unser Problem anzuwenden: Die Informationen, konkret also die RSS-Feeds, sollen �ber ein
Notifikationssystem an die Interessenten (Subscriber) ausgeliefert werden. Da sich die Funktion bzw. Rolle des RSS-Servers nicht �ndern
soll, muss die Funktion des Publishers eine andere Einheit �bernehmen. Es bietet sich an, die Rolle des Publishers ebenfalls den Klienten
zuzuweisen. Ein Klient erh�lt in der Rolle des Subscribers nach einer Anfrage seinerseits den RSS-Feed von einem RSS-Server.
Der Klient kann nun diesen Feed in
der Rolle des Publishers in das Notifikationssystem einspeisen. Das Notifikationssystem soll aus einem System von vernetzten Brokern
bestehen. Ein Broker, welcher einen Feed erh�lt, liefert diesen an die �brigen mit ihm verbundenen Subscriber bzw. Broker aus. F�r einen
Subscriber gibt es also zwei M�glichkeiten, einen Feed zu erhalten: entweder auf direkte Anfrage von einem RSS-Server (Pull) oder �ber
das Notifikationssystem (Push). Um einen neuen Feed zu erhalten, kann ein Subscriber selbst aktiv werden und den Feed vom RSS-Server anfordern, oder er kann warten,
bis ihm ein neuer Feed durch das Netzwerk �ber einen Broker �bermittelt wird. Es ergeben sich dabei folgende Fragestellungen: wann soll ein
Klient aktiv werden, um den RSS-Server zu kontaktieren, und wann soll ein Klient inaktiv bleiben, um den Feed �ber das Brokernetzwerk zu erhalten? Denn
folgende Problemsituation ist denkbar: kontaktieren alle Klienten gleichzeitig den entsprechenden RSS-Server, so bringt dies keine Vorteile, da dies dem alten Ansatz
entspricht; erreicht der neue Feed
einen Klienten �ber einen Broker, so besitzt der Klient diesen neuen Feed bereits, da er zuvor eigenm�chtig den RSS-Server kontaktiert hat.
\todo{Filter}
\subsubsection*{Ansatz: Ein Publisher}
Die einfachste L�sung w�re, einen dedizierten Publisher zu bestimmen, welcher als alleiniger Klient das Polling �bernimmt. Alle weiteren Klienten
w�rden die Feeds �ber das Broker-Netzwerk erhalten. Vorteil w�re, da� es relativ wenig
Anfragen an den RSS-Server g�be. Die Nachteile �berwiegen jedoch: in Abh�ngigkeit von der Netzstruktur kann es zu langen �bertragungszeiten der Feeds f�r einzelne
oder mehrere Subscriber kommen. Wir m�ssten eine geringere Aktualit�t der Daten \todo{in Kauf} nehmen. Zudem h�tten \todo{}wir das Problem des
\glqq Single Point of Failure\grqq: f�llt der dedizierte Publisher aus, kann die Verbreitung der Feeds zun�chst nicht mehr erfolgen.
Erst, nachdem das Netzwerk den Ausfall registriert hat und entsprechende Gegenma�nahmen eingeleitet hat (z. B. Bestimmung eines neuen dedizierten
Publishers), kann es zu einer weiteren Verbreitung der Feeds kommen. Dar�ber hinaus k�nnten die Klienten nicht von verteiltem Polling profitieren.
\subsubsection*{Ansatz: Mehrere Publisher}
Daher favorisieren wir eine L�sung, bei der zu gegebener Zeit nur eine gewisse Auswahl der Klienten gleichzeitig Anfragen an den RSS-Server senden.
Es muss also zu einer Abstimmung bzw. Koordinierung der Klienten untereinander kommen, um diese Auswahl zu bestimmen. Auch hierbei ist zu beachten,
dass Ausf�lle von Publishern im Netz nicht zu Datenverlust f�hren sollen; d. h. jeder Subscriber sollte die Informationen bzw. Feeds, die er zu erhalten w�nscht,
auch dann erhalten, wenn die f�r das Polling vorgesehenen Klienten ausfallen.


\section{Rolle der Broker}
Broker sind der zentrale Bestandteil des Notifikationssystems. Das Notifikationssystem besteht aus einer Reihe von Brokern,
die untereinander verbunden sind und ein zusammenh�ngendes Netz bilden (Overlay-Broker-Netzwerk, siehe dazu \cite{PietzuchBacon:2003:P2POverlay}).
Ein Broker empf�ngt Feeds von Brokern oder von Klienten/Publishern. Anschlie�end sorgt er f�r eine Verteilung der
Feeds an eine bestimmte Auswahl von mit ihm verbundenen Brokern bzw. Subscribern. Ein Broker kann also eine zentrale Sammelstelle f�r Feeds
unterschiedlicher Anbieter sein. Die einzelnen in den Feeds zusammengefassten Ereignisse k�nnen durch den Broker neu zusammengestellt werden.
Als Kriterien f�r neue Zusammenstellungen bieten sich die Aktualit�t der einzelnen Ereignisse sowie definierbare Filterregeln an. Einzelne 
Ereignisse, die den Broker bereits erreicht haben, brauchen nicht erneut weitergeleitet zu werden und finden daher nach unserem Konzept keinen
Eingang in die neu zusammengestellten Feeds. Dadurch k�nnen Netzressourcen gespart werden. Voraussetzung daf�r ist, dass der Broker einen Cache
unterh�lt, in dem Ereignisse zwischengespeichert werden.\\

Filter k�nnen durch Subscriber definiert und bei Brokern
hinterlegt werden. Aufgrund der Filterregeln k�nnen Ereignisse verschiedener
Anbieter aus den Feeds extrahiert und in einem neuen Feed gesammelt werden. Ein
Subscriber kann also eine individuelle Zusammenstellung der Ereignisse erhalten. Filterregeln und ihre Anwendung wurden schon ausf�hrlich
erforscht und sollen nicht Gegenstand dieser Arbeit sein. Deshalb fanden sie auch keinen Eingang in die weiter unten beschriebene
Simulationsumgebung. Sie erweitern die M�glichkeiten des Systems lediglich, haben aber keinen Einfluss auf das Grundkonzept, um das es hier geht.
Unser System \pubsubrss k�nnte mit bestehenden Pub/Sub-Systemen, welche Filtertechniken unterst�tzen (wie z. B. das System
REBECA \cite{MuFiBu:2001:ArchFrameECommApp}), kombiniert werden.\\

Der folgende Absatz beschreibt Vorg�nge, die eigentlich Teil des jeweiligen Notifikationsdienstes sind, von dem wir abstrahieren. Da jedoch ein
spezifisches Verhalten des Notifikationsdienstes auf eine optimale Funktionsweise des Systems Einfluss nehmen kann, werden wir einige Vorg�nge beschreiben.\\ 

Um sich dem entsprechenden Broker bekannt zu machen und Filter zu hinterlegen, muss sich ein Subscriber bei diesem Broker mit einer speziellen
Registrierungsnachricht registrieren.
Ein Broker braucht nur an diejenigen Subscriber aktuelle Feeds zu �bermitteln, welche auch tats�chlich online sind. Zus�tzlich ist es also notwendig,
dass ein Subscriber in regelm��igen Zeitabst�nden den jeweiligen Broker �ber seinen Online-Status unterrichtet (nennen wir eine solche Nachricht
\glqq KEEPALIVE-Nachricht\grqq{}). Erh�lt ein Broker eine KEEPALIVE-Nachricht eines Subscribers, f�r den er den Zustand \glqq ist offline\grqq{} gespeichert hat,
sollte er diesem eine Zusammenstellung der aktuellsten Ereignisse �bermitteln. Denn kommt es aufgrund von St�rungen im physischen Netzwerk oder
aufgrund von Netz�berlastung zu verloren gegangenen KEEPALIVE-Nachrichten, so gilt der Subscriber f�r seinen Broker als \glqq offline\grqq{}. Dieser wird ihm
daraufhin keine Feeds mehr �bermitteln.\\
Hat der Subscriber aufgrund von adaptiven Ma�nahmen seine aktuelle Polling-Periode stark angehoben, so wird er zun�chst keine weiteren Feeds beim
entsprechenden RSS-Server selbst�ndig erfragen, so dass ihm eventuell Informationen verloren gehen. Die Menge verloren gegangener Informationen
kann geringer gehalten werden, wenn dem Subscriber die letzten Ereignisse bei Wiedereintritt in das Overlay-Netzwerk als Feed �bermittelt werden.
Entsprechend sollte ein Subscriber seinen jeweiligen Broker dar�ber informieren, wenn er offline geht, um unn�tigen Datentransfer zu vermeiden.\\

Im Zusammenhang mit Brokern werden noch einige weitere Ma�nahmen und Nachrichtentypen notwendig sein, auf die wir jedoch erst in sp�teren Kapiteln zu sprechen
kommen werden, da sie Anpassungen an spezielle Bed�rfnisse darstellen (siehe Kapitel \glqq Churn\grqq{} \ref{cs:churn}) .
%%% Local Variables: 
%%% mode: latex
%%% TeX-master: "diplomarbeit"
%%% End: 

\section{Koordinierung der Subscriber}

Um das Netz nicht noch zus�tzlich zu belasten, sollte die Netzbelastung, die durch eventuelle Abstimmungsnachrichten entsteht, minimal sein.
Die Konzeption eines Algorithmus sollte unter folgenden Gesichtspunkten erfolgen:
\begin{itemize}
  \item Polling durch mehrere bzw. wechselnde Klienten
  \item Anfragen an den RSS-Server sollten nicht gleichzeitig f�r alle Klienten geschehen 
  \item Ausfall von Klienten im Overlay-Netzwerk soll Informationsverteilung nicht blockieren
  \item Netzbelastung durch Abstimmungsnachrichten sollte gering gehalten werden
\end{itemize} 
Im Folgenden beschreiben wir einen Algorithmus bzw. eine Technik, die unsere bisher gestellten Anforderungen erf�llt.
\subsection{Der Grundlegende Algorithmus}
\label{cs:der_grundlegende_algorithmus}
Es sei $t_0$ immer der aktuelle Zeitpunkt. Ausgehend von einem beliebigen Zeitpunkt
$t_x$ mit $t_0\leq t_x$ und einer Intervallspanne $\Delta Z$ w�hlt sich jeder
Subscriber $i$ innerhalb des Zufallsintervalls $Z:=[t_x,t_x+\Delta Z]$ einen zuf�lligen Zeitpunkt $ttr_i$ (``time to tefresh'', $ttr$ im allgemeinen), zu dem
er den aktuellen Feed vom RSS-Server erfragt (siehe Abb. \ref{Abb:determine_ttr}). Im Folgenden nennen wir $t_x$ Einstiegspunkt und $\Delta Z$ Zufallsspanne.

\begin{picturehere}{3}{1.5}{$ttr$s}{Abb:determine_ttr}
 
%\psset{xunit=1cm,yunit=1cm,runit=1cm}
%\begin{picture}(1.5,-0.5)(7,1)
\begin{picture}(7,1)(1.5,-0.5)
  \put(0,0){\vector(1,0){7}}
  \put(0,-0.2){\line(0,1){0.4}}
  \put(0,-0.5){$t_0$}
  \put(3,-0.2){\line(0,1){0.4}}
  \put(3,-0.5){$t_x$}
  \put(6,-0.2){\line(0,1){0.4}}
  \put(6,-0.5){$t_x+\Delta Z$}
  \put(5,-0.1){\line(0,1){0.2}}
  \put(4.5,0.4){$ttr_i$}
  \put(7.8,0){$time$}
\end{picture}
% \includegraphics{determine_ttr}
\end{picturehere}


Ist $ttr_i$ erreicht, so erfragt Subscriber $i$ den aktuellen Feed vom RSS-Server und setzt nun $ttr_i$ auf einen
Zufallswert innerhalb des
Zeitintervalls $Z:=[t_x,t_x+\Delta Z]$, wobei $t_x$ ebenfalls neu gew�hlt wird.
Erh�lt Subscriber $i$ vor dem Erreichen des Zeitpunktes $ttr_i$ einen Feed $feed_{new}$ von einem Broker zum 
Zeitpunkt $t_f$ (sei $feed_{old}$ der bisher bei $i$ gespeicherte Feed), so geschieht folgendes:
\pagebreak[3]
\begin{description}[\compact]
  \item [Fall I:] $feed_{new}$ ist nicht aktueller als $feed_{old}$:
    \begin{description}[\breaklabel\compact]
      \item keine �nderungen
    \end{description}
  \item[Fall II:] $feed_{new}$ ist aktueller als $feed_{old}$:
    \begin{description}[\breaklabel\compact]
      \item w�hle $t_x$ neu mit $t_0\leq t_x$
      \item  $ttr_i$ wird auf einen Zufallswert gesetzt innerhalb des Zeitintervalls\\
        \mbox{$Z:=[t_x,t_x+\Delta Z]$}
    \end{description}
\end{description}

Bezeichne $\Delta ttr_i$ die Zeitspanne zwischen $t_0$ und $ttr_i$ ($\Delta ttr$ im allgemeinen), also gilt $ttr_i:=t_0+\Delta ttr_i$.
Die $ttr$s der verschiedenen Subscriber sollten bei der Wahl einer geeigneten Zufallsfunktion �ber $Z$ gleichm��ig
verteilt sein. Durch die Wahl eines zuf�lligen Wertes innerhalb von $Z$ ist gew�hrleistet, dass nur in extremen Ausnahmef�llen (theoretisch) 
alle Klienten gleichzeitig den RSS-Server kontaktieren.  Nat�rlich kann es vorkommen, dass $ttr$s verschiedener
Subscriber auf den gleichen Zeitpunkt fallen (je nach Gr��e der Zufallsspanne $\Delta Z$ und der Anzahl der Klienten).
Die Verteilung unterliegt jedoch einem kontinuierlichen Wechsel, da die $ttr$s immer
wieder neu berechnet werden. Ausgehend von $t_x$ bildet $\Delta Z$ eine obere Schranke f�r den Erhalt des n�chsten Feeds, da jeder Klient nach
sp�testens der Zeit $\Delta Z$ selbst�ndig den Server kontaktiert, falls in der Zwischenzeit kein aktueller Feed erhalten wurde. Dadurch k�nnen lange
�bertragungszeiten zwischen den Klienten ausgeglichen werden.
Ausf�lle von Klienten k�nnen zwar zu Verz�gerungen beim
Erhalt der Feeds f�hren, sie k�nnen aber die �bermittlung der Feeds zwischen den �brigen Klienten nicht st�ren,
solange physikalisches Netz und Brokernetz intakt sind.
\subsection{Konkrete Anpassung an RSS -- Bestimmung relevanter Parameter}
Im Folgenden betrachten wir, wie sich die relevanten Parameter in Zusammenhang mit RSS bestimmen lassen.
\subsubsection{Bestimmung des Einstiegspunktes}
L�sst sich der Zeitpunkt $nextBuild$ (Neue-Info-Punkt), zu dem der RSS-Server einen neuen Feed
bereitstellt, innerhalb eines gewissen Toleranzbereiches genau bestimmen, dann k�nnen wir den Einstiegspunkt $t_x:=nextBuild$ setzen. Kann
$nextBuild$ innerhalb des gew�nschten Toleranzbereiches nicht genau bestimmt werden, kann es n�tig sein $t_x:=t_0$ zu setzen. Unter welchen
Umst�nden welche Variante vorzuziehen ist, werden wir sp�ter noch er�rtern.
\subsubsection{Bestimmung des Neue-Info-Punktes}
Um $nextBuild$ zu bestimmen, definieren wir zwei weitere Parameter: $ttl$ und $lastBuildDate$. $ttl$ steht f�r Time-To-Live und bezeichnet
die Zeit, die ein Feed aktuell bleibt, bevor er Server-seitig aktualisiert wird. $lastBuildDate$ steht f�r den Zeitpunkt, zu dem ein
Feed vom Server aktualisiert wurde.
Der RSS 2.0 Standard\cite{RSSSpecWi2004} sieht unter anderem die optionalen Parameter $lastBuildDate$ und $pubDate$ vor. Setzen wir voraus,
dass mindestens der Parameter $lastBuildDate$ vom Server bereitgestellt wird.
(Beschreibung siehe Kapitel \ref{ch_rss} auf Seite \pageref{op_rss}). $nextBuild$ l�sst sich
aufgrund des letzten aktuellen Feeds wie folgt berechnen:
\pagebreak[3]
\[nextBuild:=t_0+\Delta t\] mit \[\Delta t:=\left\{\begin{array}{r@{\quad:\quad}l}
    0 & (t_0-lastBuildDate)>ttl \\ttl-(t_0-lastBuildDate) & sonst
  \end{array}\right. \]

Alternativ k�nnte statt $lastBuildDate$ auch $pubDate$ zur Berechnung genommen werden.
\subsubsection{Bestimmung von Time-To-Live}
Um $ttl$ zu bestimmen, gibt es zwei M�glichkeiten:
\begin{itemize}
  \item {\bf Bereitstellung des $ttl$ durch den Informationsanbieter:}
    RSS 2.0\cite{RSSSpecWi2004} sieht ebenfalls den optionalen Parameter $ttl$ vor.

  \item {\bf Bestimmung des $ttl$ durch den Klienten:}
    Wird der Parameter $ttl$ vom Informationsanbieter nicht unterst�tzt, so kann $ttl$ heuristisch durch den Klienten bestimmt werden.
\end{itemize}

Wie wir sehen, sind $ttl$ und $nextBuild$ eng miteinander verkn�pft. Wollen wir $ttl$ und damit $nextBuild$ ermitteln k�nnen, stellt sich zun�chst die Frage,
ob und in welchen F�llen dies �berhaupt sinnvoll ist. Informationen k�nnen vielf�ltiger Art sein, Informationsanbieter k�nnen ganz unterschiedliche Gewohnheiten an
den Tag legen. Es h�ngt von der Vorhersagbarkeit des Auftretens neuer Daten und der zeitlichen M�glichkeit ab, diese Daten bereit zu stellen, mit welcher G�te
der $ttl$ berechnet werden kann. Stellen wir uns eine
Person vor, die regelm��ig jeden Tag ihr Tagebuch in einem Blog samt RSS-Feeds ver�ffentlicht. Sie besitzt ein nicht besonders
leistungsf�higes Rechnersystem, welches bei einer gro�en und dauerhaften Anzahl von Webzugriffen schnell �berlastet wird. Die Person steht jeden Tag um 8.00 Uhr auf,
so dass sie um 9.00 Uhr die Eintr�ge des vorherigen Tages bereit gestellt hat. Sie kann somit in den RSS-Feed einen $ttl$-Wert von 24 Stunden eintragen. So wie es
aussieht, spielt Aktualit�t in diesem Fall keine gro�e Rolle, so dass f�r die Interessenten ein relativ gro�er $\Delta Z$ Wert festgelegt werden kann (z. B. 12
Stunden). Ein RSS-Reader eines Interessenten braucht somit fr�hsten um 9.00 beim Anbieter nachzufragen und hat einen Spielraum von 12 Stunden. Mit dem von uns
geplanten Pub/Sub-RSS-System reichen in diesem Falle schon sehr wenige Subscriber aus (vielleicht sogar nur einer), um den aktuellen Feed an die gesamte Fangemeinde
zu �bermitteln. Betrachten wir nun einen anderen Fall: eine Nachrichtenagentur stellt rund um die Uhr die neuesten Schlagzeilen in einem RSS-Feed zur Verf�gung. Es
ist nicht absehbar, wann ein neues Weltereignis eintritt, so dass die Nachrichtenagentur nicht plant, den RSS-Feed mit dem Wert $ttl$ zu versorgen. Eine
heuristische Bestimmung des $ttl$ durch den Klienten ist wahrscheinlich mit einer gro�en Varianz behaftet und dadurch sehr ungenau. Und dennoch ist der Spielraum
gro�, was eine empirische Datenerhebung verdeutlicht.

\importgnuplotps{RSS-Feed-Aktualisierung}{Abb:rss_aktualisierung}{rss_aktualisierung}

Abbildung \ref{Abb:rss_aktualisierung} zeigt, wie oft und regelm��ig verschiedene Anbieter von RSS-Feeds (Spiegel, Heise, NY-Times, Slashdot, Sourceforge)
diese aktualisieren. Der gemessene Zeitraum erstreckt sich �ber
24 Stunden, die Abtastrate betrug 60 Sekunden. Hierbei f�llt auf, dass Spiegel und Heise in der Zeit zwischen ca. 0.00 und 5.00 Uhr keine Aktualisierungen
vornehmen, wogegen zu den �brigen Zeiten die Aktualisierungsintervalle schwanken. Zur Nachtzeit w�rde es sich also anbieten, den $ttl$ zu setzten. Auch bei
den NY-Times f�llt ein Zeitraum auf, indem nicht aktualisiert wird. Die zeitliche Differenz zu den deutschen Betreibern l�sst sich vermutlich durch eine
Zeitverschiebung erkl�ren. Bei den NY-Times f�llt weiterhin auf, dass in der �brigen Zeit Aktualisierungen nur st�ndlich vorgenommen werden. Also auch hier ein
Fall f�r einen vom Anbieter vorgegebenen $ttl$. Ebenfalls l�sst sich bei Slashdot und Sourceforge eine gewisse Linearit�t der Aktualisierungsintervalle 
feststellen, wenn sie auch um einiges k�rzer sind.


\subsubsection{Heuristische Bestimmung von Time-To-Live}
Hierzu gibt es verschiedene Verfahren. Um $ttl$
berechnen zu k�nnen, muss zun�chst die Rate gesch�tzt werden, mit der Feeds Server-seitig aktualisiert werden. Daf�r misst ein Subscriber innerhalb eines
Zeitintervalles $T$ die Anzahl $X$ der aufgetretenen Aktualisierungen eines Feeds. Eine Aktualisierung wird dann festgestellt, wenn ein Subscriber eine neuen Feed
erh�lt. Dabei kann das Attribut $PubDate$ der einzelnen Ereignisse (Items) eines RSS-Feeds herangezogen werden, um eine feinere Bestimmung der Aktualisierungen
vorzunehmen. Jedes neue Event steht dabei f�r eine Aktualisierung. Bei Eintritt eines Subscribers in das Netzwerk sollte der $ttl$ zun�chst auf $0$ gesetzt werden,
er wird dann w�hrend der Zeit, die sich ein Subscriber aktiv im Overlay-Netzwerk befindet, angepasst. Nat�rlich kann der errechnet Wert bei Verlassen des
Systems zwischengespeichert werden, damit er beim n�chsten Eintritt in das System wieder zur Verf�gung steht.\\

Zun�chst beschreiben wir eine simple und intuitive Methode, welche jedoch starke Verzerrungen aufweisen kann. Im Anschluss daran werden wir ein verbessertes
Verfahren vorstellen, welches von Cho und Garcia-Molina entwickelt wurde.
\paragraph{IntuitiveMethode:}
$\hat\mu_r:=\frac{X}{T}$ liefert eine gesch�tzte Aktualisierungsrate der Feeds. Das Verh�ltnis zwischen der tats�chlichen Aktualisierungsrate $\mu$ und der
Abtastrate $f$ (Anzahl der erhaltenen RSS-Feeds bzw. Ereignisse pro Zeiteinheit) $r:=\frac{\mu}{f}$ kann �ber die G�te von $\hat\mu$ Auskunft geben: gilt $r>1$,
so hat es mehr Aktualisierungen als Zugriffe (Feeds) gegeben, und der berechnete Wert $\hat\mu$ weist eine gewisse Ungenauigkeit auf. Liegt die gesamte Historie der
Akzualisierungen vor, so ist $\frac{X}{T}$ ein guter Sch�tzwert \cite{ChGM:2003:ChangeFrequency}. Da innerhalb eines Feeds mehrere Ereignisse (Items)
zusammengefasst sind, ist die Wahrscheinlichkeit geringer, dass Aktualisierungen verloren gehen, als wenn ein Feed nur ein Ereignis beinhalten w�rde.
Falls jedoch $\varDelta Z$ und $cpp$ sehr gro� gew�hlt sind bei einer gleichzeitig geringen Anzahl von Subscribern im Netzwerk, k�nnen neue Ereignisse
verloren gehen.

\paragraph{Verbesserte Methode:}
Um eine bessere Ann�herung von $\hat\mu$ an $\mu$ zu erreichen, haben Cho und Garcia-Molina in \cite{ChGM:2003:ChangeFrequency} ein anderes Verfahren zu Bestimmung
von Aktualisierungsraten entwickelt (entgegen der Berechnung bei Cho und Garcia-Molina haben wir die Aktualisierungsrate statt $\lambda$ mit $\mu$ bezeichnet, da
$\lambda$ in unserem Kontext schon belegt ist). Dabei gehen sie von der Annahme bzw. Beobachtung aus, dass die Aktualisierungsrate von Web-Inhalten durch einen
Poisson-Prozess bestimmt wird. Diese Beobachtung l�sst sich auf die von uns betrachteten RSS-Feeds �bertragen, da es sich bei diesen technisch gesehen ebenfalls
um Web-Inhalte handelt. Eine genaue Herleitung und Beschreibung des Verfahrens geht �ber den Rahmen dieser Arbeit hinaus und findet sich
in \cite{ChGM:2003:ChangeFrequency}.\\
Innerhalb des Zeitintervalls $[t;t+1]$ wird $\mu$ geliefert durch den Erwartungswert
\[E[X(t+1)-X(t)]=\sum^\infty_{k=0}k\frac{\mu^k e^{-\mu}}{k!}=\mu.\]
Dann wird bei einer unvollst�ndigen Historie der Aktualisierungen ein besserer Sch�tzwert geliefert durch:
\[\hat\mu:=-log\left(\frac{\bar X-0.5}{n-0.5}\right)\]
wobei $n$ die Anzahl der Zugriffe (also Feeds bzw. Ereignisse innerhalb eines Feeds) und $\bar X:=n-X$ die Anzahl der Zugriffe ohne Aktualisierungen ist.\\

Ein noch besserer Sch�tzwert kann geliefert werden, falls der Zeitpunkt der letzten Aktualisierung bekannt ist. Dieser ist durch das Attribut $PubDate$ bei RSS-Feeds
gegeben. Cho und Garcia-Molina beschreiben daf�r in \cite{ChGM:2003:ChangeFrequency} folgenden Algorithmus.

\begin{verbatim}
Init() /* initialize variables */ 
  N = 0; /* total number of accesses */ 
  X = 0; /* number of detected changes */ 
  T = 0; /* sum of the times from changes */ 

Update(Ti, Ii) /* update variables */ 
  N = N + 1; 
  /* Has the element changed? */ 
  If (Ti < Ii) then 
  /* The element has changed. */ 
  X = X + 1; 
  T = T + Ti; 
  else 
  /* The element has not changed */ 
  T = T + Ii; 

Estimate() /* return the estimated lambda */ 
  X� = (X-1) - X/(N*log(1-X/N));
  return X�/T;

\end{verbatim} 

Dabei dient {\ttfamily Init()} zur einmaligen Initialisierung der Variablen auf null. Bei jedem Zugriff auf ein Element (Erhalt eines Feeds in unserem Fall) wird
{\ttfamily Update()} aufgerufen. {\ttfamily Ti} ist das Zeitintervall bis zur letzten Aktualisierung beim $i$ten Zugriff, {\ttfamily Ii} das Intervall zwischen
den Zugriffen. 
\input{bestimmung_der_intervallspanne_deltai}



\section{Optimierungsziel: geringe Netzbelastung}
\todo{folgt}
\input{optimierungsziel_bevorzugte_polling_periode}
