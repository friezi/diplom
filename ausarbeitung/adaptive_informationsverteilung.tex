\chapter{Adaptive Informationsverteilung mit RSS und verteiltem Publish/Subscribe}
Im Folgenden wollen wir eine Methode beschreiben, ereignisbasierte Informationen an interessierte Klienten zu verteilen. Konkret orientieren wir
uns dabei an RSS. Mit ``Ereignis'' meinen wir im Folgenden eine Informationseinheit (z. B. eine Nachrichten-Schlagzeile),
mehrere Ereignisse k�nnen in einem Datenpaket (RSS-Feed) gesammelt an den Klienten �bermittelt werden. Dazu betrachten wir zun�chst, nach welchem
Schema die Informationen entsprechend des RSS-Systems verbreitet werden, zeigen vorherrschende Probleme auf und schlagen ein Konzept vor,
wie diese Probleme vermieden bzw. verringert werden k�nnen. Auch wenn wir im Folgenden �berwiegend von RSS sprechen, so l�sst sich das Konzept auch auf
andere Informationssysteme �bertragen, die bestimmte Kriterien erf�llen. Je nachdem, auf welche Ebene wir uns beziehen, werden wir entweder von ``Informationen''
oder ``RSS-Feeds'' sprechen. Das neu entwickelte System wollen wir im Folgenden ``\pubsubrss'' nennen.
 

\subsection{Verteiltes Polling}
Betrachten wir die Gesamtheit der Subscriber und ihr Polling-Verhalten, so erkennen wir, dass aus Sicht eines Servers das Polling-Intervall
dieser Gesamtheit gr��er ist als das Polling-Intervall jedes einzelnen Subscribers. W�nschenswert w�re es, wenn jeder Subscriber von dem
Polling-Intervall der Gesamtheit profitieren k�nnte. Die Idee ist nun, die Subscriber untereinander �ber ein Overlay-Netzwerk zu verbinden,
damit sie sich
gegenseitig die erhaltenen Informationen �bermitteln k�nnen. Das Polling wird also verteilt auf die beteiligten Subscriber. Aber nicht nur das,
wir haben es nun mit einer Kombination aus einem Pull- und einem Push-Ansatz zu tun. Allerdings �bernimmt die Push-Funktion nicht der Server,
sondern die beteiligten Einheiten des Overlay-Netzes, von dem der Server kein Bestandteil ist . Und warum favorisieren wir nicht gleich den
Push-Ansatz bezogen auf den Server? RSS ist ein schon seit l�ngerer Zeit bestehendes Konzept bzw. bestehende Technik.
Dar�ber hinaus ist es weit verbreitet.
Eine Modifikation des Grundkonzeptes w�rde f�r Unternehmen, die es unterst�tzen, bedeuten, bestehende Software austauschen zu m�ssen und
ganz neue Serviceleistungen bereitstellen zu m�ssen. Dies w�rde sicherlich auf Ablehnung sto�en und m�glicherweise nicht die gew�nschte
Verbreitung des neuen Konzeptes mit sich bringen. Ziel ist es, auf dem bestehenden Konzept aufzubauen und es in ein erweitertes Konzept zu
integrieren, um m�glichst f�r den Benutzer als auch f�r den Dienstanbieter ein Minimum an Aufwand zu erreichen. Deolasee et al.
besch�ftigen sich in \cite{bhide02adaptive} mit einer Kombination aus Push-Pull und beschreiben die Probleme, die sich aus reinen Pull- bzw.
Push-Ans�tzen ergeben.

\section{Rolle der Broker}
Broker sind der zentrale Bestandteil des Notifikationssystems. Das Notifikationssystem besteht aus einer Reihe von Brokern,
die untereinander verbunden sind und ein zusammenh�ngendes Netz bilden (Overlay-Broker-Netzwerk, siehe dazu \cite{PietzuchBacon:2003:P2POverlay}).
Ein Broker empf�ngt Feeds von Brokern oder von Klienten/Publishern. Anschlie�end sorgt er f�r eine Verteilung der
Feeds an eine bestimmte Auswahl von mit ihm verbundenen Brokern bzw. Subscribern. Ein Broker kann also eine zentrale Sammelstelle f�r Feeds
unterschiedlicher Anbieter sein. Die einzelnen in den Feeds zusammengefassten Ereignisse k�nnen durch den Broker neu zusammengestellt werden.
Als Kriterien f�r neue Zusammenstellungen bieten sich die Aktualit�t der einzelnen Ereignisse sowie definierbare Filterregeln an. Einzelne 
Ereignisse, die den Broker bereits erreicht haben, brauchen nicht erneut weitergeleitet zu werden und finden daher nach unserem Konzept keinen
Eingang in die neu zusammengestellten Feeds. Dadurch k�nnen Netzressourcen gespart werden. Voraussetzung daf�r ist, dass der Broker einen Cache
unterh�lt, in dem Ereignisse zwischengespeichert werden.\\

Filter k�nnen durch Subscriber definiert und bei Brokern
hinterlegt werden. Aufgrund der Filterregeln k�nnen Ereignisse verschiedener
Anbieter aus den Feeds extrahiert und in einem neuen Feed gesammelt werden. Ein
Subscriber kann also eine individuelle Zusammenstellung der Ereignisse erhalten. Filterregeln und ihre Anwendung wurden schon ausf�hrlich
erforscht und sollen nicht Gegenstand dieser Arbeit sein. Deshalb fanden sie auch keinen Eingang in die weiter unten beschriebene
Simulationsumgebung. Sie erweitern die M�glichkeiten des Systems lediglich, haben aber keinen Einfluss auf das Grundkonzept, um das es hier geht.
Unser System \pubsubrss k�nnte mit bestehenden Pub/Sub-Systemen, welche Filtertechniken unterst�tzen (wie z. B. das System
REBECA \cite{MuFiBu:2001:ArchFrameECommApp}), kombiniert werden.\\

Der folgende Absatz beschreibt Vorg�nge, die eigentlich Teil des jeweiligen Notifikationsdienstes sind, von dem wir abstrahieren. Da jedoch ein
spezifisches Verhalten des Notifikationsdienstes auf eine optimale Funktionsweise des Systems Einfluss nehmen kann, werden wir einige Vorg�nge beschreiben.\\ 
Um sich dem entsprechenden Broker bekannt zu machen und Filter zu hinterlegen, muss sich ein Subscriber bei diesem Broker mit einer speziellen
Registrierungsnachricht registrieren.
Ein Broker braucht nur an diejenigen Subscriber aktuelle Feeds zu �bermitteln, welche auch tats�chlich online sind. Zus�tzlich ist es also notwendig,
dass ein Subscriber in regelm��igen Zeitabst�nden den jeweiligen Broker �ber seinen Online-Status unterrichtet (nennen wir eine solche Nachricht
\glqq KEEPALIVE-Nachricht\grqq{}). Erh�lt ein Broker eine KEEPALIVE-Nachricht eines Subscribers, f�r den er den Zustand \glqq ist offline\grqq{} gespeichert hat,
sollte er diesem eine Zusammenstellung der aktuellsten Ereignisse �bermitteln. Denn kommt es aufgrund von St�rungen im physischen Netzwerk oder
aufgrund von Netz�berlastung zu verloren gegangenen KEEPALIVE-Nachrichten, so gilt der Subscriber f�r seinen Broker als \glqq offline\grqq{}. Dieser wird ihm
daraufhin keine Feeds mehr �bermitteln.
Hat der Subscriber aufgrund von adaptiven Ma�nahmen seine aktuelle Polling-Periode stark angehoben, so wird er zun�chst keine weiteren Feeds beim
entsprechenden RSS-Server selbst�ndig erfragen, so dass ihm eventuell Informationen verloren gehen. Die Menge verloren gegangener Informationen
kann geringer gehalten werden, wenn dem Subscriber die letzten Ereignisse bei Wiedereintritt in das Overlay-Netzwerk als Feed �bermittelt werden.
Entsprechend sollte ein Subscriber seinen jeweiligen Broker dar�ber informieren, wenn er offline geht, um unn�tigen Datentransfer zu vermeiden.\\

Im Zusammenhang mit Brokern werden noch einige weitere Ma�nahmen und Nachrichtentypen notwendig sein, auf die wir jedoch erst in sp�teren Kapiteln zu sprechen
kommen werden, da sie Anpassungen an spezielle Bed�rfnisse darstellen (siehe Kapitel \glqq Churn\grqq{} \ref{cs:churn}) .
%%% Local Variables: 
%%% mode: latex
%%% TeX-master: "diplomarbeit"
%%% End: 

\section{Koordinierung der Subscriber}
\todo{Literaturangaben}\\
\todo{Auflistung der Anforderungen an den Algorithmus}\\
Um das Netz nicht noch zus�tzlich zu belasten, sollte der Overhead, der durch eventuelle Abstimmungsnachrichten entsteht, minimal sein.
Die Konzeption eines Algorithmus sollte unter folgenden Gesichtspunkten geschehen:
\begin{itemize}
  \item Polling durch mehrere bzw. wechselnde Klienten
  \item Anfragen an den RSS-Server sollten nicht gleichzeitig f�r alle Klienten geschehen 
  \item Ausfall von Klienten im Overlaynetzwerk soll Informationsverteilung nicht blockieren
  \item Overhead durch Abstimmungsnachrichten sollte gering gehalten werden
\end{itemize} 
Im Folgenden beschreiben wir einen Algorithmus bzw. eine Technik, die unsere bisher gestellten Anforderungen erf�llt.
\subsubsection{Der Grundlegende Algorithmus}
Die Grundidee ist recht simpel: es sei $t_0$ immer der aktuelle Zeitpunkt. Ausgehend von einem beliebigen Zeitpunkt
$t_x$ mit $t_0\leq t_x$ und einer Intervallspanne $\Delta I$ w�hlt sich jeder Subscriber $i$ innerhalb
des Zeitintervalls $I:=[t_x,t_x+\Delta I]$ einen zuf�lligen Zeitpunkt $TTR_i$ (TimeToRefresh, $TTR$ im allgemeinen), zu dem
er den aktuellen Feed vom RSS-Server erfragt (siehe Abb. \ref{Abb:determine_ttr}).

\begin{picturehere}{3}{1.5}{$TTR$s}{Abb:determine_ttr}
 
\psset{xunit=1cm,yunit=1cm,runit=1cm}
\begin{pspicture}(1.5,-0.5)(7,1)
  \psline{->}(0,0)(7,0)
  \psline{-}(0,0.2)(0,-0.2)
  \uput[0](0,-0.5){$t_0$}
  \psline{-}(3,0.2)(3,-0.2)
  \uput[0](3,-0.5){$t_x$}
  \psline{-}(6,0.2)(6,-0.2)
  \uput[0](6,-0.5){$t_x+\Delta I$}
  \psline{-}(5,0.1)(5,-0.1)
  \uput[0](5,0.4){$TTR_i$}
\end{pspicture}
% \includegraphics{determine_ttr}
\end{picturehere}


Ist $TTR_i$ erreicht, so erfragt Subscriber $i$ den aktuellen Feed vom RSS-Server und setzt nun $TTR_i$ auf einen
Zufallswert innerhalb des
Zeitintervalls $I:=[t_x,t_x+\Delta I]$, wobei $t_x$ ebenfalls neu gew�hlt wird.
Erh�lt Subscriber $i$ vor dem Erreichen des Zeitpunktes $TTR_i$ einen Feed $feed_{new}$ von einem Broker zum 
Zeitpunkt $t_f$ (sei $feed_{old}$ der bisher bei $i$ gespeicherte Feed), so geschieht folgendes:
\pagebreak[3]
\begin{description}
  \item [Fall I:] $feed_{new}$ ist nicht aktueller als $feed_{old}$:
    \begin{description}
      \item keine �nderungen
    \end{description}
  \item[Fall II:] $feed_{new}$ ist aktueller als $feed_{old}$:
    \begin{description}
      \item w�hle $t_x$ neu mit $t_0\leq t_x$
      \item  $TTR_i$ wird gesetzt auf einen Zufallswert innerhalb des Zeitintervalls
        $I:=[t_x,t_x+\Delta I]$
    \end{description}
\end{description}

Die $TTR$s der verschiedenen Subscriber sollten bei der Wahl einer geeigneten Zufallsfunktion �ber $I$ gleichm��ig
verteilt sein. Durch die Wahl eines zuf�lligen Wertes innerhalb von $I$ ist gew�hrleistet, dass nur in extremen Ausnahmef�llen (theoretisch) 
alle Klienten gleichzeitig den RSS-Server kontaktieren.  Nat�rlich kann es vorkommen, dass $TTR$s verschiedener Subscriber auf den gleichen Zeitpunkt fallen
(je nach Gr��e der Intervallspanne $\Delta I$ und der Anzahl der Klienten). Die Verteilung unterliegt jedoch einem kontinuierlichen Wechsel, da die $TTR$s immer
wieder neu berechnet werden. $\Delta I$ bildet eine obere Schranke f�r den erhalt des n�chsten Feeds, da jeder Klient nach sp�testens $\Delta I$ selbst�ndig
den Server kontaktiert, falls in der Zwischenzeit kein aktueller Feed erhalten wurde. Ausf�lle von Klienten k�nnen zwar zu Verz�gerungen beim Erhalt der Feeds
f�hren, sie k�nnen aber die �bermittlung der Feeds zwischen den �brigen Klienten nicht st�ren.\todo{lange �bertragungszeiten}




Gehen wir zun�chst davon aus, es g�be (ausgehend vom aktuellen Zeitpunkt $t_0$) einen Zeitpunkt $nextUpdate$, zu dem der RSS-Server einen neuen Feed bereitstellt.


\section{Optimierungsziel: geringe Netzbelastung}
\todo{folgt}
\input{optimierungsziel_bevorzugte_polling_periode}
