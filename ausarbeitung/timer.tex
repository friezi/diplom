\section{Timer}
\label{css:timer}
Ein wichtiger Bestandteil der im weiteren Verlauf der Arbeit vorgestellten Verfahren sind Timer. Daher stellen wir
zun�chst dar, was unter dem Begriff ``Timer'' zu verstehen ist und welche Problematiken sich im Umgang mit Timern ergeben.\\

Bei der Kommunikation zwischen Computern werden Informationen im allgemeinen in einzelne Datenpakete eingeteilt und zwischen
den beteiligten Computern ausgetauscht. Ein Timer ist ein Mechanismus, um in Computer-Kommunikationsnetzen Fehler erkennen zu k�nnen \cite{18216},
welche bei der Kommunikation zwischen Computern auftreten. Diese Fehler k�nnen im einzelnen Verluste von Datenpaketen, das Zusammenbrechen
von Kommunikationskan�len oder indirekt eine Verst�mmelung der �betragenen Informationen bezeichnen. Dabei arbeitet ein Timer wie eine Alarm-Uhr:
nach einer vordefinierten Zeit l�uft der Timer ab. In der Annahme, dass ein bestimmtes Ereignis innerhalb dieser
vordefinierten Zeit zu geschehen hat (im Regelfall eine Antwort des Kommunikationspartners), deutet ein abgelaufener Timer auf
einen aufgetretenen Fehler hin. Hierbei ergeben sich erste Probleme: ist die Zeitspanne bis zum Ablauf eines Timers zu lang,
so wird ein eventuell aufgetretener Fehler zu sp�t erkannt. Ist die Zeitspanne dagegen zu kurz, so kann dies einen falschen Alarm zur Folge haben.
Um ein optimales Leistungsverhalten zu erzielen, wird ein Gleichgewicht zwischen diesen gegens�tzlichen Zielen angestrebt.\\

Verschiedene Timer werden bei Kommunikationsprotokollen (Bsp. TCP, siehe Abschnitt \ref{staukontrolle_tcp}) verwendet, um verschiedenartige
Formen der Kommunikationsst�rung zu erkennen. So sorgt ein ``retransmission-timer'' f�r das wiederholte Aussenden von eventuell verloren gegangenen
Datenpaketen. Dagegen gibt ein ``death-timer'' Auskunft �ber den Zusammenbruch einer Datenverbindung \cite{18216}. Die Zeitspannen
(``Timeout-Intervalle'') werden dabei je nach Timer unterschiedlich gesetzt. Um ein Gleichgewicht zwischen der Reduzierung falscher Alarme
und der rechtzeitigen Erkennung aufgetretener St�rungen zu erreichen, werden die Timeout-Intervalle meistens dynamisch und situationsabh�ngig
bestimmt. Folgende Situationen k�nnen nach Zhang \cite{18216} beispielsweise f�r das Ablaufen von retransmission-timern urs�chlich sein:
\begin{enumerate}
\item Ein Datenpaket hat den den Sender noch nicht verlassen (aufgrund eines beispielsweise blockierten Netzwerks), befindet sich aber bereits
  in einer tieferen Anwendungsschicht.
\item Das Timeout-Intervall ist k�rzer als die Zeit bis zu einer positiven Antwort vom Empf�nger.
\item Ein Datenpaket hat den Empf�nger erreicht, die Best�tigungsnachricht ging jedoch verloren.
\item \label{En:Congestion} Ein Datenpaket wurde an einem Vermittlungsknoten (Gateway) aufgrund von Datenstau verworfen.
\item \label{En:Channelerror}Ein Datenpaket wurde augrund eines Fehlers im �bertragungskanal verworfen.
\item Das Netzwerk wurde partitioniert oder der Empf�nger-Knoten ist ausgefallen.
\end{enumerate}

Nur bei Punkt \ref{En:Channelerror} ist ein unmittelbares und wiederholtes Aussenden notwendig. Punkt \ref{En:Congestion} erfordert zwar
auch ein wiederholtes Aussenden eines Datenpakets, das Setzen des retransmission-timers muss aber mit Bedacht geschehen,
um den urs�chlichen Datenstau nicht noch zu vergr��ern. Zhang beschreibt in \cite{18216} die Problematik bei dynamischer Bestimmung der
Timeout-Intervalle. Techniken zur dynamischen Bestimmung von Timeout-Intervallen werden wir in Abschnitt \ref{staukontrolle_tcp} vorstellen.


 
%%% Local Variables: 
%%% mode: latex
%%% TeX-master: "diplomarbeit"
%%% End: 
