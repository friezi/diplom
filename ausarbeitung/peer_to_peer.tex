\section{Peer-To-Peer-Systeme}
\label{Abschnitt:peer_to_peer}
Der klassische Ansatz, auf Daten oder Dienste innerhalb des Internets zuzugreifen, basiert auf dem Client/Server-Modell. Ein Server stellt als zentrale Einheit
Dienste zur Verf�gung, die einzelne, voneinander unabh�ngige Klienten anfordern k�nnen. In zunehmendem Ma�e spielen Peer-To-Peer (kurz P2P) -Systeme eine wichtige
Rolle bei der Datenkommunikation im Internet. Ein $Peer$ ist eine Einheit, die sowohl Dienste zur Verf�gung stellen als auch in Anspruch nehmen kann. So k�nnen
Peers beispielsweise Ressourcen teilen (Drucker, Speicherkapazit�t, ...) oder Daten direkt miteinander austauschen. Ein Vorteil von P2P gegen�ber dem Client/Server-Modell
liegt in der Dezentralisierung von Ressourcen, so dass Ausf�lle einzelner Einheiten  nicht unmittelbar den Zusammenbruch des Systems mit sich f�hren m�ssen.
Schollmeier unterscheidet in \cite{Schollmeier:P2PDefinition} $reine$ von $hybriden$ P2P-Systemen. Er definiert ein reines P2P-System als ein P2P-System,
bei dem eine beliebige terminale Einheit
aus dem Netzwerk entfernt werden kann, ohne dass das Netzwerk in seinen Diensten eingeschr�nkt wird. Bei hybriden P2P-Systemen sind zentrale Einheiten notwendig,
um Teile der angebotenen Netzwerkdienste zur Verf�gung zu stellen bzw. verf�gbar zu machen. Nichtsdestoweniger ist das Teilen von Netzwerkressourcen der Peers ein
substantieller Bestandteil eines hybriden P2P-Systems.\\

Verschiedene Netzwerkdienste arbeiten auf Basis von P2P-Systemen. Zu nennen w�ren beispielsweise Gnutella, Napster, Freenet oder KaZaA
(siehe \cite{Aberer:P2P,KaZaA}). Diese sind sogenannte {\itshape File-Sharing-Dienste} und dienen zum Austausch von allgemeinen Dateien oder
Musikdateien. Ein allgemeines P2P-System, was als Basistechnologie f�r andere P2P-Systeme dienen kann, ist Pastry \cite{Pastry}.
%%% Local Variables: 
%%% mode: latex
%%% TeX-master: "diplomarbeit"
%%% End: 
