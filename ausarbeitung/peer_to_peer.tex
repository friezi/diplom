\section{Peer-To-Peer-Systeme}
\label{Abschnitt:peer_to_peer}
Der klassische Ansatz, auf Daten oder Dienste innerhalb des Internets zuzugreifen, basiert auf dem Client/Server-Modell. Ein Server stellt als zentrale Einheit
Dienste zur Verf�gung, die einzelne, voneinander unabh�ngige Klienten anfordern k�nnen. In zunehmendem Ma�e spielen Peer-To-Peer (kurz P2P) -Systeme eine wichtige
Rolle bei der Datenkommunikation im Internet. Ein $Peer$ ist eine Einheit, die sowohl Dienste zur Verf�gung stellen als auch in Anspruch nehmen kann. So k�nnen
Peers beispielsweise Ressourcen teilen (Drucker, Speicherkapazit�t, ...) oder Daten direkt miteinander austauschen. Ein Vorteil von P2P gegen�ber dem Client/Server-Modell
liegt in der Dezentralisierung von Ressourcen, so dass Ausf�lle einzelner Einheiten  nicht unmittelbar den Zusammenbruch des Systems mit sich f�hren m�ssen.
Schollmeier unterscheidet in \cite{Schollmeier:P2PDefinition} $reine$ von $hybriden$ P2P-Systemen. Er definiert ein reines P2P-System als ein P2P-System,
bei dem eine beliebige terminale Einheit
aus dem Netzwerk entfernt werden kann, ohne dass das Netzwerk in seinen Diensten eingeschr�nkt wird. Bei hybriden P2P-Systemen sind zentrale Einheiten notwendig,
um Teile der angebotenen Netzwerkdienste zur Verf�gung zu stellen bzw. verf�gbar zu machen. Nichtsdestoweniger ist das Teilen von Netzwerkressourcen der Peers ein
substantieller Bestandteil eines hybriden P2P-Systems.\\

Verschiedene Netzwerkdienste arbeiten auf Basis von P2P-Systemen. Zu nennen w�ren beispielsweise Gnutella, Napster, Freenet oder KaZaA
(siehe \cite{Aberer:P2P,KaZaA}). Diese sind sogenannte {\itshape File-Sharing-Dienste} und dienen zum Austausch von allgemeinen Dateien oder
Musikdateien. Ein allgemeines P2P-System, was als Basistechnologie f�r andere P2P-Systeme dienen kann, ist Pastry \cite{Pastry}. Im Folgenden wollen wir
Pastry kurz in den Grundz�gen vorstellen, da es als Basis f�r weitere in dieser Arbeit angesprochene Verfahren dient.

\subsection{Pastry}
Das von A. Rowstron und P. Druschel entwickelte P2P-System Pastry \cite{Pastry} ist ein skalierbares P2P-Grundger�st, welches die Verteilung und Auffindung
von Objekten in Netzwerken erm�glicht und als Basis f�r gro� angelegte P2P-Anwendungen dienen kann. Um eine gleichm��ige Verteilung und ein schnelles Routing zu erreichen,
wird jedem Knoten bzw. Peer innerhalb des Netzwerks ein eindeutiger numerischer 128-Bit Bezeichner ({\itshape Id}) zugewiesen. Erh�lt ein Knoten eine Nachricht und einen Schl�ssel,
so wird die Nachricht zu dem Knoten weitergeleitet, dessen Id dem Schl�ssel numerisch am n�chsten liegt. Ist die Anzahl der Knoten im Pastry-Netz $N$,
so liegt die erwartete Zahl an Routing-Schritten in $O$$(\log N)$. Jeder Knoten innerhalb des Pastry-Netzes unterh�lt eine Routing-Tabelle, eine Nachbarschafts-Menge
({\itshape neighborhood-set}) und eine Blatt-Menge ({\itshape leaf-set}). In einer Spalte $n$ der Routing-Tabelle sind IP-Adressen von Knoten eingetragen,
die im Prefix der L�nge $n$ mit der Knoten-Id des Ausgangsknotens �bereinstimmen. Die Nachbarschafts-Menge unterh�lt Knoten-Ids und IP-Adressen von Knoten,
die dem Ausgangsknoten entsprechend einer Nachbarschaftsmetrik am n�chsten liegen. Die Blatt-Menge enth�lt Knoten-Ids von Knoten, die der Knoten-Id des Ausgangsknotens am
n�chsten liegen und wird f�r die Weiterleitung von Nachrichten herangezogen. F�r eine ausf�hrliche Beschreibung verweisen wir auf \cite{Pastry}.\\
Dar�ber hinaus ist Pastry ein selbstorganisierendes und adaptierendes System. Es kann selbst�ndig die notwendigen Initialisierungsschritte bei Zu- und Abwanderung von Knoten
vornehmen (zu Selbstorganisation siehe \cite{HeMueGei:2005:SelfMa}).
%%% Local Variables: 
%%% mode: latex
%%% TeX-master: "diplomarbeit"
%%% End: 
