\subsection{Churn}
\label{cs:churn}
Bei allen Peer-To-Peer-Systemen tritt ein Ph�nomen auf, welches als ``Churn'' bezeichnet wird: das dynamische Zu- und Abwandern von Klienten bzw. Knoten. In
Peer-To-Peer-Systemen spielen die Klienten eine entscheidende Rolle. In fast allen diesen Systemen kommunizieren die Klienten direkt miteinander, um Daten
auszutauschen oder wichtige Informationen zu liefern (z. B. Dateien oder Routing-Informationen). Je nach Struktur des Peer-To-Peer-Netzes kann die pl�tzliche
Abwesenheit von beteiligten Knoten zu Beeintr�chtigungen oder gar Fehlfunktionen des Systems f�hren. W�hrend unstrukturierte Peer-To-Peer-Netze Churn zum Teil
gut verkraften k�nnen, k�nnen strukturierte Peer-To-Peer-Netze (z. B. DHTs) mit Churn nicht anstandslos umgehen, bzw. ben�tigen spezielle Mechanismen, um den
Einfl�ssen von Churn entgegenzuwirken \cite{Stutzbach2004}.\\

\paragraph{Auswirkungen:}
Die Auswirkungen von Churn k�nnen verschiedenartig sein. So kann Churn beispielsweise bei BitTorrent dazu f�hren, dass sich Downloadzeiten verl�ngern, falls
Klienten das System verlassen, oder dass bestimmte Dateien nicht zugreifbar sind, falls ein Tracker ausf�llt. Bei DHT-basierten Netzen (neuere BitTorrent-Versionen
sind mittlerweile DHT-basiert) kann schon ein
vor�ber\-gehen\-der Verlust eines Nachbarknoten zu Performanzeinbu�en f�hren (Effizienz ist ein Designziel bei DHT), da der Ausgangsknoten gezwungen ist,
suboptimale Routen zu w�hlen \cite{Rhea2004}. Dabei erh�ht sich die Wahrscheinlichkeit zuk�nftiger Ausf�lle des Systems.

\paragraph{Messung:}
Um Churn messen zu k�nnen, ist eine Metrik erforderlich, die �ber Zu- und Abwanderung Auskunft gibt. Es bietet sich an, die Zeit zwischen Betreten und Verlassen
des Systems durch einen Knoten zu messen. Beobachtungen haben gezeigt, dass sich die durchschnittlichen Zeiten zwischen einer Stunde und einigen Minuten bewegen
k�nnen \cite{Rhea2004}. Stutzbach und Rejaie \cite{Stutzbach2004} haben eine Reihe von Techniken entwickelt und in einem von ihnen entwickelten Tool ($Cruiser$)
vereint, um das Gnutella-Netzwerk in relativ kurzer Zeit zu durchforsten und einen aktuellen Schnappschuss (``Snapshot'') der Gnutella-Population zu erhalten. Wie
Stutzbach und Rejaie feststellen, ist das Gnutella-Netzwerk ist ein sehr gro�es Peer-To-Peer-Netzwerk bestehend aus hunderttausenden von heterogenen und
geographisch verteilten Peers. Dar�ber hinaus wird jeder User-Client-Prozess direkt durch einen Benutzer gesteuert. Das hei�t das ``Verhalten der User-Clients
repr�sentiert vollst�ndig Benutzer-gesteuerte Aspekte dynamischer Mitgliedschaft'' im System \cite{Stutzbach2004}. Ergebnisse einer solchen Untersuchung k�nnen
damit f�r Peer-To-Peer-Systeme auf gleicher Funktionsbasis herangezogen werden.
