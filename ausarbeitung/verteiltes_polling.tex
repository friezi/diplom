
\subsection{Verteiltes Polling}
Betrachten wir die Gesamtheit der Subscriber und ihr Polling-Verhalten, so erkennen wir, dass aus Sicht eines Servers das Polling-Intervall
dieser Gesamtheit gr��er ist als das Polling-Intervall jedes einzelnen Subscribers. W�nschenswert w�re es, wenn jeder Subscriber von dem
Polling-Intervall der Gesamtheit profitieren k�nnte. Die Idee ist nun, die Subscriber untereinander �ber ein Overlay-Netzwerk zu verbinden,
damit sie sich
gegenseitig die erhaltenen Informationen �bermitteln k�nnen. Das Polling wird also verteilt auf die beteiligten Subscriber. Aber nicht nur das,
wir haben es nun mit einer Kombination aus einem Pull- und einem Push-Ansatz zu tun. Allerdings �bernimmt die Push-Funktion nicht der Server,
sondern die beteiligten Einheiten des Overlay-Netzes, von dem der Server kein Bestandteil ist . Und warum favorisieren wir nicht gleich den
Push-Ansatz bezogen auf den Server? RSS ist ein schon seit l�ngerer Zeit bestehendes Konzept bzw. bestehende Technik.
Dar�ber hinaus ist es weit verbreitet.
Eine Modifikation des Grundkonzeptes w�rde f�r Unternehmen, die es unterst�tzen, bedeuten, bestehende Software austauschen zu m�ssen und
ganz neue Serviceleistungen bereitstellen zu m�ssen. Dies w�rde sicherlich auf Ablehnung sto�en und m�glicherweise nicht die gew�nschte
Verbreitung des neuen Konzeptes mit sich bringen. Ziel ist es, auf dem bestehenden Konzept aufzubauen und es in ein erweitertes Konzept zu
integrieren, um m�glichst f�r den Benutzer als auch f�r den Dienstanbieter ein Minimum an Aufwand zu erreichen (Deolasee et al.
besch�ftigen sich in \cite{bhide02adaptive} mit einer Kombination aus Push-Pull und beschreiben die Probleme, die sich aus reinen Pull- bzw.
Push-Ans�tzen ergeben).\\
Es ist naheliegend, das Publish-Subscribe-Paradigma auf unser Problem anzuwenden: Die Informationen, konkret also die RSS-Feeds, sollen �ber ein
Notifikationssystem an die Interessenten, sprich Subscriber, ausgeliefert werden. Da sich die Funktion bzw. Rolle des RSS-Servers nicht �ndern
soll, muss die Funktion des Publishers eine andere Einheit �bernehmen. Es bietet sich an, die Rolle des Publishers ebenfalls den Klienten
zuzuweisen. Ein Klient erh�lt in der Rolle des Subscribers auf eine Anfrage hin den RSS-Feed von einem Server. Der Klient kann nun diesen Feed in
der Rolle des Publishers in das Notifikationssystem einspeisen. Das Notifikationssystem soll aus einem System von vernetzten Brokern
bestehen. Ein Broker, welcher einen Feed erh�lt, liefert diesen an die �brigen mit ihm verbundenen Subscriber bzw. Broker aus. F�r einen
Subscriber gibt es also zwei M�glichkeiten, einen Feed zu erhalten: entweder auf direkte Anfrage von einem Server (Pull) oder von einem
Broker (Push). Um einen neuen Feed zu erhalten, kann ein Subscriber selbst aktiv werden und den Feed vom Server anfordern, oder er kann warten,
bis ihm ein neuer Feed durch das Netzwerk �ber einen Broker �bermittelt wird. Es ergibt sich dabei folgende Fragestellung: wann soll ein
Klient aktiv werden und den Server kontaktieren und wann soll ein Klient inaktiv bleiben, um den Feed indirekt zu erhalten? Denn
folgende Problemsituation ist denkbar: kontaktieren alle Klienten gleichzeitig den entsprechenden Server, ist nichts gewonnen; erreicht
einen Klienten der neue Feed �ber einen Broker, so besitzt der entsprechende Klient diesen neuen Feed bereits. Es muss also zu einer Abstimmung
bzw. Koordinierung der Klienten unter einander kommmen, die daf�r sorgt, dass nur ein Teil der Klienten den Server kontaktiert.

