
\subsection{Verteiltes Polling}
Betrachten wir die Gesamtheit der Subscriber und ihr Polling-Verhalten, so erkennen wir, dass aus Sicht eines Servers das Polling-Intervall
dieser Gesamtheit gr��er ist als das Polling-Intervall jedes einzelnen Subscribers. W�nschenswert w�re es, wenn jeder Subscriber von dem
Polling-Intervall der Gesamtheit profitieren k�nnte. Die Idee ist nun, die Subscriber untereinander �ber ein Overlay-Netzwerk zu verbinden,
damit sie sich
gegenseitig die erhaltenen Informationen �bermitteln k�nnen. Das Polling wird also verteilt auf die beteiligten Subscriber. Aber nicht nur das,
wir haben es nun mit einer Kombination aus einem Pull- und einem Push-Ansatz zu tun. Allerdings �bernimmt die Push-Funktion nicht der Server,
sondern die beteiligten Einheiten des Overlay-Netzes, von dem der Server kein Bestandteil ist . Und warum favorisieren wir nicht gleich den
Push-Ansatz bezogen auf den Server? RSS ist ein schon seit l�ngerer Zeit bestehendes Konzept bzw. bestehende Technik.
Dar�ber hinaus ist es weit verbreitet.
Eine Modifikation des Grundkonzeptes w�rde f�r Unternehmen, die es unterst�tzen, bedeuten, bestehende Software austauschen zu m�ssen und
ganz neue Serviceleistungen bereitstellen zu m�ssen. Dies w�rde sicherlich auf Ablehnung sto�en und m�glicherweise nicht die gew�nschte
Verbreitung des neuen Konzeptes mit sich bringen. Ziel ist es, auf dem bestehenden Konzept aufzubauen und es in ein erweitertes Konzept zu
integrieren, um m�glichst f�r den Benutzer als auch f�r den Dienstanbieter ein Minimum an Aufwand zu erreichen. Deolasee et al.
besch�ftigen sich in \cite{bhide02adaptive} mit einer Kombination aus Push-Pull und beschreiben die Probleme, die sich aus reinen Pull- bzw.
Push-Ans�tzen ergeben.
