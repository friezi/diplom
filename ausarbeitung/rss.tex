\section{Really Simple Syndication (RSS)}
\label{ch_rss}
RSS ist eine Technik, um Kurznachrichten oder andere Inhalte des WorldWideWeb an Nutzer, die sich daf�r interessieren, zu verbreiten. Anbieter k�nnen
die von ihnen angebotenen Informationen oder auf sie bezogene Kurzbeschreibungen in einer Datei (RSS-Feed) zusammenfassen und �ber einen Link
auf einer Webseite zur Verf�gung stellen. Nutzer k�nnen RSS-Feeds abonnieren und ben�tigen daf�r ein Programm (RSS-Reader), welches die abonnierten
RSS-Feeds automatisch �ber das Netz l�d und anzeigt. Jeder RSS-Feed ist dabei einem Thema (Channel bzw. Kanal) zugeordnet. H�ufig werden RSS-Feeds dazu benutzt,
Nachrichten-Schlagzeilen zu sammeln und bereitzustellen. Es k�nnen aber auch aktuelle Blog-Eintr�ge oder Video- bzw. Audiodaten (als Podcasts) zur
Verf�gung gestellt werden.\\

RSS z�hlt zwar zu den Pub/Sub-Systemen jedoch nicht zu den verteilten push-basierten, wie sie in Abschnitt \ref{publishsubscribe} beschrieben wurden,
denn es folgt dem klassischen Client/Server-Ansatz. Es handelt sich um zentralisiertes Polling, Message-Broker sind hierbei nicht vorgesehen.\\

Die zugrunde liegende Beschreibungssprache der RSS-Feeds ist XML. Urspr�nglich bediente man sich dabei
der RDF-Syntax. RDF steht f�r \glqq Resource Description Framework\grqq{} und bezeichnet eine formale Sprache, mit der Metadaten definiert werden, um Webinhalte
darzustellen (siehe \cite{RDF}). F�r RSS gibt es verschiedene �bersetzungen: \glqq Rich Site Summary\grqq{}, \glqq RDF Site Summary\grqq{} und ab RSS der Version 2.0
\glqq Really Simple Syndication\grqq{}. RSS 2.0 macht keinen Gebrauch mehr von RDF.

\paragraph{Elemente eines RSS-Feeds:}
Ein RSS-Feed besteht aus einigen erforderlichen und optionalen Elementen (siehe \cite{RSSSpecWi2004}), die hier in tabellarischer Form aufgelistet werden
sollen (Tabellen \ref{Tab:erf_Elemente_RSS} und \ref{Tab:opt_Elemente_RSS}):
\begin{table*}[h]
  \begin{center}
    \begin{tabular}{|ll|}
      \hline
      Element & Beschreibung\\
      \hline\hline
      title & Name des Kanals \\
      link & URL der Webseite, zu der dieser Kanal geh�rt\\
      description & textuelle Beschreibung des Kanals\\
      \hline
    \end{tabular}
  \end{center}
  \caption{erforderliche Elemente eines RSS-Feeds}
  \label{Tab:erf_Elemente_RSS}
\end{table*}

\begin{table*}[h]
  \begin{center}
    \begin{tabular}{|lp{11cm}|}
      \hline
      Element & Beschreibung\\
      \hline\hline
       language & Sprache, in der der Kanal verfasst wurde\\
       copyright & Copyright-Notiz\\
       managingEditor & email-Adresse des f�r den Inhalt verantwortlichen Herausgebers\\
       webMaster & email-Adresse der f�r den Kanal technisch verantwortlichen Person\\
       pubDate & Ver�ffentlichungszeitpunkt des Inhalts des Kanals\\
       lastBuildDate & Zeitpunkt, zu dem sich der Inhalt des Kanals das letzte Mal ver�ndert hat\\
       category & spezifiziert eine oder mehrere Kategorien, zu der der Kanal geh�rt\\
       generator & bezeichnet Programm, welches den Kanal generiert hat\\
       docs & URL, verweist auf Dokumentation des verwendeteten Formats\\
       cloud & Adresse, bei der sich Programme registrieren k�nnen, um �ber Aktualisierungen des Kanals unterrichtet zu werden (PubSub)\\
       ttl & time to live: gibt Auskunft dar�ber, wie lange Feed aktuell bleibt\\
       image & Adresse einer Bilddatei\\
       rating & Bewertung f�r diesen Kanal (PICS)\\
       textInput & Spezifiziert Texteingabe-Box\\
       skipHours & Uhrzeiten, zu denen der Kanal nicht abgefragt werden soll\\
       skipDays & Tage, zu denen der Kanal nicht abgefragt werden soll\\
      \hline
    \end{tabular}
  \end{center}
  \caption{optionale Elemente eines RSS-Feeds}
  \label{Tab:opt_Elemente_RSS}
\end{table*}

Jeder Eintrag eines RSS-Feeds wird �ber das Element \texttt{item} gekennzeichnet. Eintr�ge in einem Feed sind optional. Ein RSS-Feed kann beliebig viele Eintr�ge besitzen.
Das Element \texttt{item} besitzt ebenfalls eine Reihe von optionalen Elementen, welche in Tabelle \ref{Tab:item_Elemente_RSS} zusammengetragen sind.

\begin{table*}[h]
  \begin{center}
    \begin{tabular}{|ll|}
      \hline
      Element & Beschreibung\\
      \hline\hline
      title & Titel des Eintrags \\
      link & URL des Eintrags\\
      description & textuelle Beschreibung des Eintrags\\
      author & email-Adresse des Autors des Eintrags\\
      category & kategorielle Zuordnung des Eintrags\\
      comments & URL f�r Kommentare bzgl. des Eintrags\\
      enclosure & Beschreibt ein zugeh�riges Media-Objekt\\
      guid & eindeutige Zeichenkette zur Identifizierung des Eintrags\\
      pubDate & zeigt an, wann der Eintrag erstellt wurde\\
      source & Ursprungskanal, zu dem der Eintrag geh�rt\\
      \hline
    \end{tabular}
  \end{center}
  \caption{Elemente eines \texttt{item}-Elements eines RSS-Feeds}
  \label{Tab:item_Elemente_RSS}
\end{table*}

Abbildung \ref{Abb:beisp_RSS-Feed} zeigt ein Beispiel eines RSS-Feeds.

\begin{figure*}
  \lstset{language=XML}
  \begin{lstlisting}
<?xml version="1.0" encoding="UTF-8"?>
<rss version="2.0">
        <channel>
                <title>Front page news</title>
                <link>http://www.my-web-page.org/</link>
                <description>Some Infos</description>
                <copyright>Copyright by me</copyright>
                <lastBuildDate>Wed, 01 Nov 2006 20:19:08 GMT
                </lastBuildDate>
                <generator>RSS generator</generator>
                <item>
                        <title>new Songs produced</title>
                        <description>next-generation Songs
                        </description>
                        <author>justme@my-web-page.org</author>
                        <pubDate>Wed, 01 Nov 2006 20:19:08 GMT
                        </pubDate>
                </item>
                <item>
                        <title>new Videos produced</title>
                        <description>next-generation Videos
                        </description>
                        <author>justme@my-web-page.org</author>
                        <pubDate>Wed, 01 Nov 2006 20:12:34 GMT
                        </pubDate>
                </item>
        </channel>

</rss>
  \end{lstlisting}
  \caption{Beispiel eines RSS-Feeds}
  \label{Abb:beisp_RSS-Feed}
\end{figure*}

%\subsection{Optionale Parameter}
\label{op_rss}

%%% Local Variables: 
%%% mode: latex
%%% TeX-master: "diplomarbeit"
%%% End: 
