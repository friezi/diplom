\section{Koordinierung der Subscriber}
\todo{Literaturangaben}\\

Um das Netz nicht noch zus�tzlich zu belasten, sollte die Netzbelastung, die durch eventuelle Abstimmungsnachrichten entsteht, minimal sein.
Die Konzeption eines Algorithmus sollte unter folgenden Gesichtspunkten erfolgen:
\begin{itemize}
  \item Polling durch mehrere bzw. wechselnde Klienten
  \item Anfragen an den RSS-Server sollten nicht gleichzeitig f�r alle Klienten geschehen 
  \item Ausfall von Klienten im Overlaynetzwerk soll Informationsverteilung nicht blockieren
  \item Netzbelastung durch Abstimmungsnachrichten sollte gering gehalten werden
\end{itemize} 
Im Folgenden beschreiben wir einen Algorithmus bzw. eine Technik, die unsere bisher gestellten Anforderungen erf�llt.
\subsection{Der Grundlegende Algorithmus}
Es sei $t_0$ immer der aktuelle Zeitpunkt. Ausgehend von einem beliebigen Zeitpunkt
$t_x$ mit $t_0\leq t_x$ und einer Intervallspanne $\Delta I$ w�hlt sich jeder Subscriber $i$ innerhalb
des Zeitintervalls $I:=[t_x,t_x+\Delta I]$ einen zuf�lligen Zeitpunkt $ttr_i$ (TimeToRefresh, $ttr$ im allgemeinen), zu dem
er den aktuellen Feed vom RSS-Server erfragt (siehe Abb. \ref{Abb:determine_ttr}).

\begin{picturehere}{3}{1.5}{$ttr$s}{Abb:determine_ttr}
 
%\psset{xunit=1cm,yunit=1cm,runit=1cm}
%\begin{picture}(1.5,-0.5)(7,1)
\begin{picture}(7,1)(1.5,-0.5)
  \put(0,0){\vector(1,0){7}}
  \put(0,-0.2){\line(0,1){0.4}}
  \put(0,-0.5){$t_0$}
  \put(3,-0.2){\line(0,1){0.4}}
  \put(3,-0.5){$t_x$}
  \put(6,-0.2){\line(0,1){0.4}}
  \put(6,-0.5){$t_x+\Delta I$}
  \put(5,-0.1){\line(0,1){0.2}}
  \put(4.5,0.4){$ttr_i$}
  \put(7.8,0){$time$}
\end{picture}
% \includegraphics{determine_ttr}
\end{picturehere}


Ist $ttr_i$ erreicht, so erfragt Subscriber $i$ den aktuellen Feed vom RSS-Server und setzt nun $ttr_i$ auf einen
Zufallswert innerhalb des
Zeitintervalls $I:=[t_x,t_x+\Delta I]$, wobei $t_x$ ebenfalls neu gew�hlt wird.
Erh�lt Subscriber $i$ vor dem Erreichen des Zeitpunktes $ttr_i$ einen Feed $feed_{new}$ von einem Broker zum 
Zeitpunkt $t_f$ (sei $feed_{old}$ der bisher bei $i$ gespeicherte Feed), so geschieht folgendes:
\pagebreak[3]
\begin{description}[\compact]
  \item [Fall I:] $feed_{new}$ ist nicht aktueller als $feed_{old}$:
    \begin{description}[\breaklabel\compact]
      \item keine �nderungen
    \end{description}
  \item[Fall II:] $feed_{new}$ ist aktueller als $feed_{old}$:
    \begin{description}[\breaklabel\compact]
      \item w�hle $t_x$ neu mit $t_0\leq t_x$
      \item  $ttr_i$ wird auf einen Zufallswert gesetzt innerhalb des Zeitintervalls\\
        \mbox{$I:=[t_x,t_x+\Delta I]$}
    \end{description}
\end{description}

Die $ttr$s der verschiedenen Subscriber sollten bei der Wahl einer geeigneten Zufallsfunktion �ber $I$ gleichm��ig
verteilt sein. Durch die Wahl eines zuf�lligen Wertes innerhalb von $I$ ist gew�hrleistet, dass nur in extremen Ausnahmef�llen (theoretisch) 
alle Klienten gleichzeitig den RSS-Server kontaktieren.  Nat�rlich kann es vorkommen, dass $ttr$s verschiedener
Subscriber auf den gleichen Zeitpunkt fallen (je nach Gr��e der Intervallspanne $\Delta I$ und der Anzahl der Klienten).
Die Verteilung unterliegt jedoch einem kontinuierlichen Wechsel, da die $ttr$s immer
wieder neu berechnet werden. $\Delta I$ bildet eine obere Schranke f�r den Erhalt des n�chsten Feeds, da jeder Klient nach
sp�testens $\Delta I$ selbst�ndig den Server kontaktiert, falls in der Zwischenzeit kein aktueller Feed erhalten wurde. Dadurch k�nnen lange
�bertragungszeiten zwischen den Klienten ausgeglichen werden.
Ausf�lle von Klienten k�nnen zwar zu Verz�gerungen beim
Erhalt der Feeds f�hren, sie k�nnen aber die �bermittlung der Feeds zwischen den �brigen Klienten nicht st�ren,
solange nicht das physikalische Netz betroffen ist und das Brokernetz intakt ist.
\subsection{Konkrete Anpassung an RSS -- Bestimmung relevanter Parameter}
Im Folgenden betrachten wir, wie sich die relevanten Parameter in Zusammenhang mit RSS bestimmen lassen.
\subsubsection{Bestimmung von $t_x$}
Die Frage ist, wie sich $t_x$ bestimmt. L�sst sich der Zeitpunkt $nextBuild$, zu dem der RSS-Server einen neuen Feed
bereitstellt, innerhalb eines gewissen Toleranzbereiches genau bestimmen, dann k�nnen wir $t_x:=nextBuild$ setzen. Kann
$nextBuild$ innerhalb des gew�nschten Toleranzbereiches nicht genau bestimmt werden, kann es n�tig sein $t_x:=t_0$ zu setzen. Unter welchen
Umst�nden welche Variante vorzuziehen ist, werden wir sp�ter noch er�rtern.
\subsubsection{Bestimmung von $nextBuild$}
Um $nextBuild$ zu bestimmen, definieren wir zwei weitere Parameter: $ttl$ und $lastBuildDate$. $ttl$ steht f�r ``time to live'' und bezeichnet
die Anzahl Minuten, die ein Feed aktuell bleibt, bevor er server-seitig aktualisiert wird. $lastBuildDate$ steht f�r den Zeitpunkt, zu dem ein
Feed vom Server aktualisiert wurde.
Der RSS 2.0 Standard\cite{RSSSpecWi2004} sieht unter anderem die optionalen Parameter $lastBuildDate$ und $pubDate$ vor. Setzen wir voraus,
dass mindestens der Parameter $lastBuildDate$ vom Server bereitgestellt wird.
(Beschreibung siehe Kapitel \ref{ch_rss} auf Seite \pageref{op_rss}). $nextBuild$ l�sst sich
aufgrund des letzten aktuellen Feeds wie folgt berechnen:
\pagebreak[3]
\[nextBuild:=t_0+\Delta t\] mit \[\Delta t:=\left\{\begin{array}{r@{\quad:\quad}l}
    0 & (t_0-lastBuildDate)>ttl \\ttl-(t_0-lastBuildDate) & sonst
  \end{array}\right. \]

Alternativ k�nnte statt $lastBuilddate$ auch $pubDate$ zur Berechnung genommen werden.
\subsubsection{Bestimmung des $ttl$}
Um $ttl$ zu bestimmen, gibt es zwei M�glichkeiten.

\paragraph{Bereitstellung des $ttl$ durch den Informationsanbieter:}
RSS 2.0\cite{RSSSpecWi2004} sieht ebenfalls den optionalen Parameter $ttl$ vor.
Ob es aus Sicht des Informationsanbieters Sinn macht, diesen optionalen Parameter bereitzustellen, h�ngt von der Vorhersagbarkeit des Auftretens
neuer Daten und der zeitlichen M�glichkeit, diese Daten bereit zu stellen, ab. Beispielsweise ist das Auftreten von Ereignissen des
aktuellen Tagesgeschehens mit Sicherheit nicht vorhersagbar. Werden diese zum Zeitpunkt der Berichterstattung bereitgestellt, so wird
es sicherlich wenig Sinn machen, den Parameter $ttl$ zu definieren, da nicht vorherbestimmt werden kann, wann neue Ereignisse eintreten. Ein
Betreiber einer Webseite jedoch, der t�glich sein Tagebuch ver�ffentlicht, k�nnte diesen Parameter mitliefern, wenn er z.B. immer um Punkt
10 Uhr seine Webseite aktualisiert.

\paragraph{Bestimmung des $ttl$ durch den Klienten:}
Wird der Parameter $ttl$ vom Informationsanbieter nicht unterst�tzt, so kann $ttl$ heuristisch durch den Klienten bestimmt werden.
Hierzu gibt es verschiedene Verfahren, von denen wir eines vorstellen wollen.
Cho und Garcia-Molina beschreiben in \cite{ChGM:2003:ChangeFrequency} eine Methode zur Bestimmung der Update-Frequenz mit besseren
Eigenschaften als die naive Methode.

\subsubsection{Heuristische Bestimmung von Time-To-Live}
Hierzu gibt es verschiedene Verfahren. Um $ttl$
berechnen zu k�nnen, muss zun�chst die Rate gesch�tzt werden, mit der Feeds Server-seitig aktualisiert werden. Daf�r misst ein Subscriber innerhalb eines
Zeitintervalles $T$ die Anzahl $X$ der aufgetretenen Aktualisierungen eines Feeds. Eine Aktualisierung wird dann festgestellt, wenn ein Subscriber eine neuen Feed
erh�lt. Dabei kann das Attribut $PubDate$ der einzelnen Ereignisse (Items) eines RSS-Feeds herangezogen werden, um eine feinere Bestimmung der Aktualisierungen
vorzunehmen. Jedes neue Event steht dabei f�r eine Aktualisierung. Bei Eintritt eines Subscribers in das Netzwerk sollte der $ttl$ zun�chst auf $0$ gesetzt werden,
er wird dann w�hrend der Zeit, die sich ein Subscriber aktiv im Overlay-Netzwerk befindet, angepasst. Nat�rlich kann der errechnet Wert bei Verlassen des
Systems zwischengespeichert werden, damit er beim n�chsten Eintritt in das System wieder zur Verf�gung steht.\\

Zun�chst beschreiben wir eine simple und intuitive Methode, welche jedoch starke Verzerrungen aufweisen kann. Im Anschluss daran werden wir ein verbessertes
Verfahren vorstellen, welches von Cho und Garcia-Molina entwickelt wurde.
\paragraph{IntuitiveMethode:}
$\hat\mu_r:=\frac{X}{T}$ liefert eine gesch�tzte Aktualisierungsrate der Feeds. Das Verh�ltnis zwischen der tats�chlichen Aktualisierungsrate $\mu$ und der
Abtastrate $f$ (Anzahl der erhaltenen RSS-Feeds bzw. Ereignisse pro Zeiteinheit) $r:=\frac{\mu}{f}$ kann �ber die G�te von $\hat\mu$ Auskunft geben: gilt $r>1$,
so hat es mehr Aktualisierungen als Zugriffe (Feeds) gegeben, und der berechnete Wert $\hat\mu$ weist eine gewisse Ungenauigkeit auf. Liegt die gesamte Historie der
Akzualisierungen vor, so ist $\frac{X}{T}$ ein guter Sch�tzwert \cite{ChGM:2003:ChangeFrequency}. Da innerhalb eines Feeds mehrere Ereignisse (Items)
zusammengefasst sind, ist die Wahrscheinlichkeit geringer, dass Aktualisierungen verloren gehen, als wenn ein Feed nur ein Ereignis beinhalten w�rde.
Falls jedoch $\varDelta Z$ und $cpp$ sehr gro� gew�hlt sind bei einer gleichzeitig geringen Anzahl von Subscribern im Netzwerk, k�nnen neue Ereignisse
verloren gehen.

\paragraph{Verbesserte Methode:}
Um eine bessere Ann�herung von $\hat\mu$ an $\mu$ zu erreichen, haben Cho und Garcia-Molina in \cite{ChGM:2003:ChangeFrequency} ein anderes Verfahren zu Bestimmung
von Aktualisierungsraten entwickelt (entgegen der Berechnung bei Cho und Garcia-Molina haben wir die Aktualisierungsrate statt $\lambda$ mit $\mu$ bezeichnet, da
$\lambda$ in unserem Kontext schon belegt ist). Dabei gehen sie von der Annahme bzw. Beobachtung aus, dass die Aktualisierungsrate von Web-Inhalten durch einen
Poisson-Prozess bestimmt wird. Diese Beobachtung l�sst sich auf die von uns betrachteten RSS-Feeds �bertragen, da es sich bei diesen technisch gesehen ebenfalls
um Web-Inhalte handelt. Eine genaue Herleitung und Beschreibung des Verfahrens geht �ber den Rahmen dieser Arbeit hinaus und findet sich
in \cite{ChGM:2003:ChangeFrequency}.\\
Innerhalb des Zeitintervalls $[t;t+1]$ wird $\mu$ geliefert durch den Erwartungswert
\[E[X(t+1)-X(t)]=\sum^\infty_{k=0}k\frac{\mu^k e^{-\mu}}{k!}=\mu.\]
Dann wird bei einer unvollst�ndigen Historie der Aktualisierungen ein besserer Sch�tzwert geliefert durch:
\[\hat\mu:=-log\left(\frac{\bar X-0.5}{n-0.5}\right)\]
wobei $n$ die Anzahl der Zugriffe (also Feeds bzw. Ereignisse innerhalb eines Feeds) und $\bar X:=n-X$ die Anzahl der Zugriffe ohne Aktualisierungen ist.\\

Ein noch besserer Sch�tzwert kann geliefert werden, falls der Zeitpunkt der letzten Aktualisierung bekannt ist. Dieser ist durch das Attribut $PubDate$ bei RSS-Feeds
gegeben. Cho und Garcia-Molina beschreiben daf�r in \cite{ChGM:2003:ChangeFrequency} folgenden Algorithmus.

\begin{verbatim}
Init() /* initialize variables */ 
  N = 0; /* total number of accesses */ 
  X = 0; /* number of detected changes */ 
  T = 0; /* sum of the times from changes */ 

Update(Ti, Ii) /* update variables */ 
  N = N + 1; 
  /* Has the element changed? */ 
  If (Ti < Ii) then 
  /* The element has changed. */ 
  X = X + 1; 
  T = T + Ti; 
  else 
  /* The element has not changed */ 
  T = T + Ii; 

Estimate() /* return the estimated lambda */ 
  X� = (X-1) - X/(N*log(1-X/N));
  return X�/T;

\end{verbatim} 

Dabei dient {\ttfamily Init()} zur einmaligen Initialisierung der Variablen auf null. Bei jedem Zugriff auf ein Element (Erhalt eines Feeds in unserem Fall) wird
{\ttfamily Update()} aufgerufen. {\ttfamily Ti} ist das Zeitintervall bis zur letzten Aktualisierung beim $i$ten Zugriff, {\ttfamily Ii} das Intervall zwischen
den Zugriffen. 
\input{bestimmung_der_intervallspanne_deltai}

