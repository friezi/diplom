\section{Koordinierung der Subscriber}
\todo{Literaturangaben}\\
\todo{Auflistung der Anforderungen an den Algorithmus}\\
Um das Netz nicht noch zus�tzlich zu belasten, sollte der Overhead, der durch eventuelle Abstimmungsnachrichten entsteht, minimal sein.
Die Konzeption eines Algorithmus sollte unter folgenden Gesichtspunkten erfolgen:
\begin{itemize}
  \item Polling durch mehrere bzw. wechselnde Klienten
  \item Anfragen an den RSS-Server sollten nicht gleichzeitig f�r alle Klienten geschehen 
  \item Ausfall von Klienten im Overlaynetzwerk soll Informationsverteilung nicht blockieren
  \item Overhead durch Abstimmungsnachrichten sollte gering gehalten werden
\end{itemize} 
Im Folgenden beschreiben wir einen Algorithmus bzw. eine Technik, die unsere bisher gestellten Anforderungen erf�llt.
\subsubsection*{Der Grundlegende Algorithmus}
Es sei $t_0$ immer der aktuelle Zeitpunkt. Ausgehend von einem beliebigen Zeitpunkt
$t_x$ mit $t_0\leq t_x$ und einer Intervallspanne $\Delta I$ w�hlt sich jeder Subscriber $i$ innerhalb
des Zeitintervalls $I:=[t_x,t_x+\Delta I]$ einen zuf�lligen Zeitpunkt $TTR_i$ (TimeToRefresh, $TTR$ im allgemeinen), zu dem
er den aktuellen Feed vom RSS-Server erfragt (siehe Abb. \ref{Abb:determine_ttr}).

\begin{picturehere}{3}{1.5}{$TTR$s}{Abb:determine_ttr}
 
%\psset{xunit=1cm,yunit=1cm,runit=1cm}
%\begin{picture}(1.5,-0.5)(7,1)
\begin{picture}(7,1)(1.5,-0.5)
  \put(0,0){\vector(1,0){7}}
  \put(0,-0.2){\line(0,1){0.4}}
  \put(0,-0.5){$t_0$}
  \put(3,-0.2){\line(0,1){0.4}}
  \put(3,-0.5){$t_x$}
  \put(6,-0.2){\line(0,1){0.4}}
  \put(6,-0.5){$t_x+\Delta I$}
  \put(5,-0.1){\line(0,1){0.2}}
  \put(4.5,0.4){$TTR_i$}
  \put(7.8,0){$time$}
\end{picture}
% \includegraphics{determine_ttr}
\end{picturehere}


Ist $TTR_i$ erreicht, so erfragt Subscriber $i$ den aktuellen Feed vom RSS-Server und setzt nun $TTR_i$ auf einen
Zufallswert innerhalb des
Zeitintervalls $I:=[t_x,t_x+\Delta I]$, wobei $t_x$ ebenfalls neu gew�hlt wird.
Erh�lt Subscriber $i$ vor dem Erreichen des Zeitpunktes $TTR_i$ einen Feed $feed_{new}$ von einem Broker zum 
Zeitpunkt $t_f$ (sei $feed_{old}$ der bisher bei $i$ gespeicherte Feed), so geschieht folgendes:
\pagebreak[3]
\begin{description}
  \item [Fall I:] $feed_{new}$ ist nicht aktueller als $feed_{old}$:
    \begin{description}
      \item keine �nderungen
    \end{description}
  \item[Fall II:] $feed_{new}$ ist aktueller als $feed_{old}$:
    \begin{description}
      \item w�hle $t_x$ neu mit $t_0\leq t_x$
      \item  $TTR_i$ wird auf einen Zufallswert gesetzt innerhalb des Zeitintervalls
        $I:=[t_x,t_x+\Delta I]$
    \end{description}
\end{description}

Die $TTR$s der verschiedenen Subscriber sollten bei der Wahl einer geeigneten Zufallsfunktion �ber $I$ gleichm��ig
verteilt sein. Durch die Wahl eines zuf�lligen Wertes innerhalb von $I$ ist gew�hrleistet, dass nur in extremen Ausnahmef�llen (theoretisch) 
alle Klienten gleichzeitig den RSS-Server kontaktieren.  Nat�rlich kann es vorkommen, dass $TTR$s verschiedener
Subscriber auf den gleichen Zeitpunkt fallen (je nach Gr��e der Intervallspanne $\Delta I$ und der Anzahl der Klienten).
Die Verteilung unterliegt jedoch einem kontinuierlichen Wechsel, da die $TTR$s immer
wieder neu berechnet werden. $\Delta I$ bildet eine obere Schranke f�r den Erhalt des n�chsten Feeds, da jeder Klient nach
sp�testens $\Delta I$ selbst�ndig den Server kontaktiert, falls in der Zwischenzeit kein aktueller Feed erhalten wurde. Dadurch k�nnen lange
�bertragungszeiten zwischen den Klienten ausgeglichen werden.
Ausf�lle von Klienten k�nnen zwar zu Verz�gerungen beim
Erhalt der Feeds f�hren, sie k�nnen aber die �bermittlung der Feeds zwischen den �brigen Klienten nicht st�ren,
solange nicht das Sublayer betroffen ist und das Brokernetz intakt ist.
\subsubsection*{Konkrete Anpassung an RSS}
Die Frage ist, wie sich $t_x$ bestimmt. L�sst sich der Zeitpunkt $nextBuild$, zu dem der RSS-Server einen neuen Feed
bereitstellt, innerhalb eines gewissen Toleranzbereiches genau bestimmen, dann k�nnen wir $t_x:=nextBuild$ setzen. Kann
$nextBuild$ innerhalb des gew�nschten Toleranzbereiches nicht genau bestimmt werden, kann es n�tig sein $t_x:=t_0$ zu setzen. Unter welchen
Umst�nden welche Variante vorzuziehen ist, werden wir sp�ter noch er�rtern.
\subsubsection*{Bestimmung von $nextBuild$}
Es gibt verschiedene M�glichkeiten $nextBuild$ zu bestimmen:
\paragraph{Server-seitig unterst�tzte Bestimmung:}
Der RSS 2.0 Standard\cite{RSSSpecWi2004} sieht einige sehr hilfreiche optionale Parameter vor: $TTL$, $lastBuildDate$ und $pubDate$
(Beschreibung siehe Kapitel\ref{ch_rss} auf Seite \pageref{op_rss}). Falls der RSS-Server diese Parameter unterst�tzt, l�sst sich
$nextBuild$ aufgrund des letzten aktuellen Feeds wie folgt berechnen:
\[nextBuild:=t_0+\Delta t\] mit \[\Delta t:=\left\{\begin{array}{r@{\quad:\quad}l}
    0 & (t_0-lastBuildDate)>TTL \\TTL-(t_0-lastBuildDate) & sonst
  \end{array}\right. \]

Ob es aus Sicht des Informationsanbieters Sinn macht, diese optionalen Parameter bereitzustellen, h�ngt von der Vorhersagbarkeit des Auftretens
neuer Daten und der zeiltichen M�glichkeit, diese Daten bereit zu stellen, ab. Beispielsweise ist das Auftreten von Ereignissen des
aktuellen Tagesgeschehens mit Sicherheit nicht vorhersagbar. Werden diese zum Zeitpunkt der Berichterstattung bereitgestellt, so wird
es sicherlich wenig Sinn machen, den Parameter $TTL$ zu definieren, da nicht vorherbestimmt werden kann, wann neue Ereignisse eintreten. Ein
Betreiber einer Webseite, der t�glich sein Tagebuch ver�ffentlicht, k�nnte diesen Parameter jedoch mitliefern, wenn er z.B. immer um Punkt
10 Uhr seine Webseite aktualisiert.

\paragraph{Bestimmung durch den Klienten}:
Wird der Parameter $TTL$ vom Betreiber nicht unterst�tzt, so kann $nextBuild$ heuristisch bestimmt werden. Hierzu gibt es verschiedene
Verfahren, von denen wir zwei vorstellen wollen.\todo{Bestimmung des TTL}


