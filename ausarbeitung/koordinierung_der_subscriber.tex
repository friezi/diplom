\section{Koordinierung der Subscriber}
\todo{Literaturangaben}\\
\todo{Auflistung der Anforderungen an den Algorithmus}\\
Um das Netz nicht noch zus�tzlich zu belasten, sollte der Overhead, der durch eventuelle Abstimmungsnachrichten entsteht, minimal sein.
Die Konzeption eines Algorithmus sollte unter folgenden Gesichtspunkten erfolgen:
\begin{itemize}
  \item Polling durch mehrere bzw. wechselnde Klienten
  \item Anfragen an den RSS-Server sollten nicht gleichzeitig f�r alle Klienten geschehen 
  \item Ausfall von Klienten im Overlaynetzwerk soll Informationsverteilung nicht blockieren
  \item Overhead durch Abstimmungsnachrichten sollte gering gehalten werden
\end{itemize} 
Im Folgenden beschreiben wir einen Algorithmus bzw. eine Technik, die unsere bisher gestellten Anforderungen erf�llt.
\subsubsection*{Der Grundlegende Algorithmus}
Es sei $t_0$ immer der aktuelle Zeitpunkt. Ausgehend von einem beliebigen Zeitpunkt
$t_x$ mit $t_0\leq t_x$ und einer Intervallspanne $\Delta I$ w�hlt sich jeder Subscriber $i$ innerhalb
des Zeitintervalls $I:=[t_x,t_x+\Delta I]$ einen zuf�lligen Zeitpunkt $TTR_i$ (TimeToRefresh, $TTR$ im allgemeinen), zu dem
er den aktuellen Feed vom RSS-Server erfragt (siehe Abb. \ref{Abb:determine_ttr}).

\begin{picturehere}{3}{1.5}{$TTR$s}{Abb:determine_ttr}
 
%\psset{xunit=1cm,yunit=1cm,runit=1cm}
%\begin{picture}(1.5,-0.5)(7,1)
\begin{picture}(7,1)(1.5,-0.5)
  \put(0,0){\vector(1,0){7}}
  \put(0,-0.2){\line(0,1){0.4}}
  \put(0,-0.5){$t_0$}
  \put(3,-0.2){\line(0,1){0.4}}
  \put(3,-0.5){$t_x$}
  \put(6,-0.2){\line(0,1){0.4}}
  \put(6,-0.5){$t_x+\Delta I$}
  \put(5,-0.1){\line(0,1){0.2}}
  \put(4.5,0.4){$TTR_i$}
  \put(7.8,0){$time$}
\end{picture}
% \includegraphics{determine_ttr}
\end{picturehere}


Ist $TTR_i$ erreicht, so erfragt Subscriber $i$ den aktuellen Feed vom RSS-Server und setzt nun $TTR_i$ auf einen
Zufallswert innerhalb des
Zeitintervalls $I:=[t_x,t_x+\Delta I]$, wobei $t_x$ ebenfalls neu gew�hlt wird.
Erh�lt Subscriber $i$ vor dem Erreichen des Zeitpunktes $TTR_i$ einen Feed $feed_{new}$ von einem Broker zum 
Zeitpunkt $t_f$ (sei $feed_{old}$ der bisher bei $i$ gespeicherte Feed), so geschieht folgendes:
\pagebreak[3]
\begin{description}
  \item [Fall I:] $feed_{new}$ ist nicht aktueller als $feed_{old}$:
    \begin{description}
      \item keine �nderungen
    \end{description}
  \item[Fall II:] $feed_{new}$ ist aktueller als $feed_{old}$:
    \begin{description}
      \item w�hle $t_x$ neu mit $t_0\leq t_x$
      \item  $TTR_i$ wird auf einen Zufallswert gesetzt innerhalb des Zeitintervalls
        $I:=[t_x,t_x+\Delta I]$
    \end{description}
\end{description}

Die $TTR$s der verschiedenen Subscriber sollten bei der Wahl einer geeigneten Zufallsfunktion �ber $I$ gleichm��ig
verteilt sein. Durch die Wahl eines zuf�lligen Wertes innerhalb von $I$ ist gew�hrleistet, dass nur in extremen Ausnahmef�llen (theoretisch) 
alle Klienten gleichzeitig den RSS-Server kontaktieren.  Nat�rlich kann es vorkommen, dass $TTR$s verschiedener
Subscriber auf den gleichen Zeitpunkt fallen (je nach Gr��e der Intervallspanne $\Delta I$ und der Anzahl der Klienten).
Die Verteilung unterliegt jedoch einem kontinuierlichen Wechsel, da die $TTR$s immer
wieder neu berechnet werden. $\Delta I$ bildet eine obere Schranke f�r den Erhalt des n�chsten Feeds, da jeder Klient nach
sp�testens $\Delta I$ selbst�ndig den Server kontaktiert, falls in der Zwischenzeit kein aktueller Feed erhalten wurde. Dadurch k�nnen lange
�bertragungszeiten zwischen den Klienten ausgeglichen werden.
Ausf�lle von Klienten k�nnen zwar zu Verz�gerungen beim
Erhalt der Feeds f�hren, sie k�nnen aber die �bermittlung der Feeds zwischen den �brigen Klienten nicht st�ren,
solange nicht das Sublayer betroffen ist und das Brokernetz intakt ist.
\subsubsection*{�bertragung auf RSS}
 Nach \cite{RSSSpecWi2004} existiert




Gehen wir zun�chst davon aus, es g�be (ausgehend vom aktuellen Zeitpunkt $t_0$) einen Zeitpunkt $nextUpdate$, zu dem der RSS-Server einen neuen Feed bereitstellt.
