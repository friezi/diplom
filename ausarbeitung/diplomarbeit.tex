\documentclass{scrbook}
\usepackage[latin1]{inputenc}
\usepackage{latexsym}
\usepackage[german]{babel}
%\usepackage{ngerman}
\usepackage[a4paper]{geometry}
\usepackage{graphicx}
\usepackage{pst-all}
\usepackage{multido}
\usepackage{glossar}
\usepackage{expdlist}
%\usepackage{nomencl}
\pagestyle{plain}
\usepackage{color}
\usepackage{keyval}
%\usepackage{charter}
%\geometry{textwidth=17cm, textheight=22cm} 
\parindent0em
\pagenumbering{arabic}

\renewcommand{\glshead}{\chapter*{Glossar}}


\newcommand{\todo}[1]{\hspace{0.5cm}\textbf{\textcolor{red}{TODO}: #1}\hspace{0.5cm}}

\newenvironment{notizen}
{%


\textbf{\textcolor{blue}{Notizen:}}\newline\textbf{\textcolor{blue}{$|^{---}$}}\begin{quote}
}%
{%
  \end{quote}
  \textbf{\textcolor{blue}{$|_{---}$}} \newline
}%

\makeatletter
% args are: width,height,caption,label
\newcommand{\picturehere@caption}{false}
\newcommand{\picturehere@label}{false}
\define@key{picturehere}{caption}[false]{
  \renewcommand{\picturehere@caption}{#1}}
\define@key{picturehere}{label}[false]{
  \renewcommand{\picturehere@label}{#1}}
\newenvironment{picturehere}[4]%
{%
 \setkeys{picturehere}{caption = #3, label = #4}
  \begin{figure*}[h]%
    \begin{center}%
      \begin{minipage}[b]{.3\linewidth}%
        \psset{xunit=1cm,yunit=1cm,runit=1cm}%
        \begin{pspicture}(#1,#2)%
}%
{%
        \end{pspicture}%
      \end{minipage}%
    \end{center}%
    \caption{\picturehere@caption}%
    \label{\picturehere@label}%
  \end{figure*}%
}
\makeatother

%parameters: Bezeichnung, Referenz, ps-Datei
\newcommand{\importgnuplotps}[3]{
  \begin{picturehere}{1}{10}{\mbox{#1}}{#2}
    \includegraphics[bb=530 310 730 600,scale=0.55,angle=-90]{#3}
  \end{picturehere}
}

%parameters: Bezeichnung, Referenz, ps-Datei
\newcommand{\importsmallgnuplotps}[3]{
  \begin{picturehere}{1}{7}{\mbox{#1}}{#2}
    \includegraphics[bb=530 310 730 600,scale=0.4,angle=-90]{#3}
  \end{picturehere}
}

\newcommand{\pubsubrss}{Pub/Sub-RSS }


%%% Local Variables: 
%%% mode: latex
%%% TeX-master: "diplomarbeit"
%%% End: 


\includeonly{einleitung,publish_subscribe,rss_mittels_verteiltem_pubsub,adaptive_informationsverteilung,glossar}

\title{Adaptive Informationsverbreitung mittels RSS und Publish/Subscribe \\[2cm]\textnormal{Diplomarbeit}}
\author{Friedemann Zintel}
\date{\today}

\begin{document}
%\sffamily

\maketitle

\bibliographystyle{alpha}
%\bibliographystyle{plain}

\frontmatter

\setcounter{tocdepth}{3}
\setcounter{secnumdepth}{3}
\tableofcontents
\listoffigures

\mainmatter

\chapter{Einleitung}
Der Austausch von Informationen hat seit jeher eine gro�e Bedeutung in menschlichen Gemeinschaften. Das Internet stellt ein Medium dar,
Informationen relativ schnell und leicht zur Verf�gung zu stellen, auszutauschen und zu verbreiten. Zu diesem Zweck sind im Laufe der
Zeit verschiedene Internet-Dienste entwickelt worden. Private Informationen k�nnen als Email (elektronische Post) versendet werden.
Um Informationen einer breiten �ffentlichkeit zug�nglich zu machen, findet die M�glichkeit, diese auf Webseiten zu pr�sentieren, h�ufig Anwendung.
F�r statische Informationen, die keinem stetigen Wechsel unterliegen, ist dies ein geeignetes Verfahren.
Da im Internet pr�sentierte Informationen in zunehmendem Ma�e einen stark dynamischen Charakter aufweisen (z. B. Nachrichten-Schlagzeilen, B�rsennachrichten), 
sind Konzepte entwickelt worden, Interessenten �ber deren inhaltliche �nderungen zu unterrichten. Ein allgemeines Kommunikationsmodell, welches diesem Zweck gerecht
wird, ist Publish/Subscribe.

\section{Motivation}
Im Kommunikationsmodell von verteiltem Publish/Subscribe (kurz: Pub/Sub bzw. Pub/\-Sub-System) finden sich drei
Parteien: \glqq Publisher\grqq{} stellen Informationen �ffentlich zur Verf�gung; \glqq Subscriber\grqq{} interessieren
sich f�r bestimmte Informationen und k�nnen diese abonnieren (subskribieren);
\glqq Message-Broker\grqq{} (kurz: \glqq Broker\grqq) sorgen f�r die Sammlung und Weiterleitung von Notifikationen (Benachrichtigungen) an die Subscriber.
Aufgrund der indirekten Verbindung zwischen Subscribern und Publishern �ber die Broker brauchen sich Publisher nicht um eine Vielzahl von
Interessenten zu k�mmern.
Zudem l�sst die broker-seitige Ansammlung einer Reihe von Informationen die Definition von Filtern zu, durch die
anbieter�bergreifend Informationen seitens der Subscriber abgefragt werden k�nnen.\\

RSS (Really Simple Syndication) z�hlt zwar zu den Pub/Sub-Systemen jedoch nicht zu den verteilten push-basierten, wie sie oben beschrieben wurden,
denn es folgt dem klassischen Client/Server-Ansatz.
Es handelt sich um zentralisiertes Polling, Message-Broker sind hierbei nicht vorgesehen:
auf einer Webseite wird ein RSS-Feed (aktueller Beitrag) vom RSS-Server als XML-Datei abgelegt.
Ein Feed ist einem Thema (Kanal bzw. Channel) zugeordnet und beinhaltet
verschiedene Eintr�ge (z.B. Nachrichten-Schlagzeilen). Interessenten bzw. Nutzer k�nnen nun diese Feeds
herunter laden. Da nicht vorhersehbar ist, zu welchem Zeitpunkt eine Aktualisierung der Feeds seitens des Servers erfolgen wird,
m�ssen die Nutzer in regelm��igen
Zeitabst�nden beim Server nachfragen, um �ber Neuigkeiten informiert zu sein (Polling). Die Definition von Filtern ist nicht vorgesehen,
d.h. Nutzer erhalten den kompletten Feed und m�ssen
sich notfalls bei vielen verschiedenen Servern subskribieren, um eine gro�e Auswahl an Informationen zu einem bestimmten Thema zu erhalten.
RSS-Feeds sind �ber die Channel zwar themenbasiert organisiert (vgl. \cite{LiuVenSirer:2005:MeasureRSSPubSub}), jedoch wird die inhaltlich thematische
Zuordnung auf Server-Seite vorgenommen. Eine thematische Filterung 
aus Nutzersicht auf h�herer Ebene kann lediglich lokal auf Nutzerseite erfolgen.\\
Das st�ndige Abfragen des Servers durch m�glicherweise hunderttausende von Abonnenten f�hrt zu einer
starken server-seitigen Last (vgl. \cite{SandlerEtAl:2005:FeedTree},
\cite{Hicks:2004:RSSBandwith}).
M�chte ein Nutzer eine mit dem Publisher m�glichst synchronisierte Aktualisierung der Feeds erreichen, so muss er die Polling-Rate
hoch setzen. Dies bedeutet wiederum eine h�here Server-Belastung. Um eine hohe Belastung durch hohe Polling-Raten zu unterbinden, haben Server die
M�glichkeit, einen Nutzer (abgrenzbar durch seine IP-Adresse), dessen Polling-Rate zu hoch ist, vor�bergehend zu blocken.
Dadurch ist der Grad der Aktualit�t der RSS-Feeds begrenzt.
Beispiele f�r die (RSS-basierten) Datenmengen, die pro Tag von einzelnen Servern �bermittelt werden m�ssen, finden sich
in \cite{SandlerEtAl:2005:FeedTree}.

\section{Ziele der Arbeit}
Um eine Lastverteilung, besonders auch (und damit) eine potentiell gr��ere Aktualit�t der Informationen und eine h�here Flexibilit�t in Bezug
auf die Auswahl der Informationen seitens der Nutzer zu erreichen, bietet es
sich an, ein push-basiertes Publish/Subscribe-System einzusetzen. RSS-Server k�nnten als Publisher fungieren und neue Feeds eigenm�chtig an ihre lokalen Broker
�bermitteln, welche die Feeds ihrerseits an Nutzer (Subscriber) weiterleiten. Dies w�rde jedoch bedeuten,
dass sowohl auf Nutzer- als auch auf RSS-Server-Seite die schon bestehenden Anwendungen durch neue ersetzt werden m�ssten oder entsprechende Software
installiert werden m�sste.
Solch ein Ansatz w�rde sicherlich auf
Akzeptanzschwierigkeiten sto�en. Au�erdem w�rde der Aktualit�tsgrad hier wie auch bei dem Ansatz unten mit nur einem Poller von dem Overlay-Netz abh�ngen
und k�nnte ung�nstig ausfallen! Um dies zu verhindern, bietet sich folgendes Konzept an: Die Rolle der RSS-Server bleibt bestehen und �ndert sich nicht.
Nutzer entsprechen sowohl den Publishern als auch den Subscribern.
In der Rolle des Publishers erfragt ein Nutzer den aktuellen Feed beim
RSS-Server und speist ihn in das Netz ein bzw. �bermittelt ihn an seinen lokalen Broker.
Dieser sorgt daf�r, dass der Feed an alle Broker weitergeleitet wird, zu denen Subscriber verbunden sind, die sich f�r diesen interessieren.
In der Rolle eines Subscribers erh�lt ein Nutzer einen Feed von
einem Broker. F�r einen Nutzer gibt es also zwei M�glichkeiten einen Feed zu erhalten: direkt vom RSS-Server oder �ber das Netzwerk.
Um eine Server-Entlastung zu erreichen, w�rde es ausreichen, nur
einen Publisher zu definieren, welcher die Feeds in das System einspeist. Alle �brigen Nutzer/Subscriber w�rden
den Feed �ber das Netzwerk erhalten,
wodurch weniger Anfragen beim RSS-Server auftreten w�rden. Doch
dieses Konzept ginge auf Kosten der Aktualit�t der Feeds, da es m�glicherweise sehr lange dauert, bis ein Feed einen Subscriber erreicht (abh�ngig von der
Struktur des Overlay-Netzes).
Es sollte also verschiedene Publisher geben, welche Feeds in das Netzwerk einspeisen und die geeignet im Overlay-Netzwerk positioniert sind.
Die Publisher m�ssen sich untereinander koordinieren, welche von
ihnen den n�chsten Feed herunter laden. Sind es zu viele Publisher, so f�hrt das wiederum zu einer Mehrbelastung des Servers. Es gilt also, ein Optimum
zwischen Aktualit�tsgrad und Server-Belastung zu finden.\\
Ziel dieser Arbeit ist es, ein hohes Update-Intervall ohne Blockierung zu erreichen, indem sich die Publisher beim Polling abwechseln.
Dies kann den Grad an Aktualit�t der Feeds erh�hen.\\

Die Ausarbeitung eines solchen Systems soll folgende Weiterentwicklung erm�glichen: RSS-Feeds k�nnen mit Hilfe von Filtern, welche durch Subscriber
definiert werden,
schon auf der Broker-Ebene gefiltert werden (Filter werden von Brokern an
Nachbarbroker weitergeleitet). Die Filterung kann anbieter�bergreifend wirken, da sich Feeds von unterschiedlichen
RSS-Servern an einem Broker sammeln. Um eine genaue Analyse der Feeds bez�glich der angegebenen Filter zu erm�glichen, wird die inhaltsbasierte Filterung
(content-based filtering, \cite{Muehl:2001:GenericConstraints}) favorisiert. Hierf�r ist es notwendig, nicht nur die Feeds selbst, sondern
auch die durch sie referenzierten Daten herunterzuladen, um sie in den Filtervorgang mit einzubeziehen.
Die Filterung der Feeds auf Broker-Ebene erm�glicht den Subscribern eine
inhaltsorientierte Sichtweise auf Subskriptionen im Gegensatz zur bisherigen anbieterorientierten Sichtweise.\\

Um diese Arbeit in angemessenem Rahmen zu halten, werden wir auf die Umsetzung von Filtertechniken nicht weiter eingehen, schaffen jedoch eine Basis f�r
weitere Entwicklungen in dieser Richtung. Als Ausgangspunkt f�r Erweiterungen verwenden wir die
Filtertechnik des \glqq floodings\grqq{} (siehe \cite{MuFiBu:2002:FilterSimilarities}).\\

Vorteile des genannten Ansatzes gegen�ber anderen Ans�tzen (wie z.B. FeedTree, siehe \cite{SandlerEtAl:2005:FeedTree}) sind:
\begin{itemize}
  \item Die L�sung kann sich problemlos in ein bestehendes RSS-System integrieren, d.h. es ist keine neue Anbieter-Software
        f�r RSS-Server n�tig.
  \item Bei geschickter Umsetzung kann auch die Client-Software weiter verwendet werden (z.B. Abfangen der Anfragen durch einen Proxy).
  \item Eine komplette Neukonstruktion eines Pub/Sub-Systems ist nicht notwendig, auf schon bestehende Systeme kann zur�ckgegriffen und aufgebaut werden
        (z.B. RE\-BE\-CA \cite{MuFiBu:2001:ArchFrameECommApp}).
\end{itemize}

Im Verlauf der Arbeit soll eine angemessene L�sung f�r die oben beschriebene Zielsetzung gefunden werden. Des Weiteren soll eine Simulationsumgebung implementiert
werden, um die entwickelten Algorithmen umzusetzen, zu evaluieren und zu demonstrieren.

\section{wissenschaftlicher Beitrag}
Im Lauf der Arbeit entwickeln wir zwei voneinander relativ unabh�ngige Verfahren:
\begin{itemize}
\item ein Abstimmungsverfahren f�r Klienten bez�glich des n�chsten Zugriffs auf einen Server, welches gegen�ber hohen Nachrichtenlaufzeiten und Knotenausf�llen
  tolerant ist, ohne zu einem stark erh�hten Netzwerkverkehr durch zus�tzliche Abstimmungsnachrichten zu f�hren
\item ein Verfahren zur Adaption von Polling-Raten verschiedener Klienten an den Leistungsgrad eines Servers, das ohne Kenntnis der �brigen Klienten und der
  Gesamtstruktur und -gr��e des Netzwerks auskommt
\end{itemize}
Diese Verfahren wurden zwar in Hinsicht auf RSS entwickelt, sie lassen sich aber auch auf andere �hnlich gestaltete bzw. ereignisbasierte Systeme �bertragen.
Zudem lassen sich die Algorithmen durch geringf�gige Modifikationen auch an
paketorientierte �bertragungstechniken (z. B. UDP) anpassen, obwohl wir diese Verfahren f�r eine datenstrombasierte �bertragungsmethode (TCP) entworfen haben.

\section{Aufbau der Arbeit}
In Kapitel \ref{Abschnitt:Grundlagen} geben wir zun�chst einen �berblick �ber ausgew�hlte Grundbegriffe und Verfahren, soweit sie f�r diese Arbeit von Bedeutung
sind. Im weiteren Verlauf der Arbeit werden wir Begriffe aus diesem Kapitel referenzieren.\\

Kapitel \ref{Abschnitt:RSS_mittels_verteiltem_pubsub} befasst sich mit RSS und der designbezogenen Problematik im Kontext einer gro�en Nutzergemeinde. Wir werden
die schon angesprochenen Ziele unserer Arbeit genauer formulieren und problematische Aspekte bei ihrer Umsetzung beleuchten. Zuletzt geben wir einen kurzen
�berblick �ber Arbeiten, die sich auf vergleichbare oder andere Weise diesem Thema zugewandt haben.\\

In Kapitel \ref{adapt_informationsverteilung} werden wir ein auf unsere Zielsetzung abgestimmtes L�sungskonzept entwickeln. Die Entwicklung erfolgt dabei
schrittweise, und wir werden sie durch die Analyse der gegebenen Begleitumst�nde motivieren.\\

Kapitel \ref{c:implementierung} besch�ftigt sich mit der Implementierung der entwickelten Simulationsumgebung. Wir stellen die wichtigsten Klassen vor, geben einen
�berblick �ber die wichtigsten Methoden und ihr Wechselspiel und erl�utern die Parametrisierung der Software.\\


In Kapitel \ref{c:experimente} stellen wir Experimente vor, mit deren Hilfe wir die entwickelten Algorithmen getestet und evaluiert haben. Die Ergebnisse werden
durch geeignete Graphiken und Diagramme erl�utert.\\

Kapitel \ref{Abschnitt:Zusammenfassung} fasst sowohl die geleistete Arbeit als auch die Ergebnisse zusammen und f�hrt eine abschlie�ende Bewertung durch.
Dar�ber hinaus werden wir einen Ausblick auf m�gliche Erweiterungen zu den entwickelten Verfahren geben.


%%% Local Variables: 
%%% mode: latex
%%% TeX-master: "diplomarbeit"
%%% End: 

\chapter{Grundlagen}
\chapter{Publish/Subscribe}
%\section{Andere Arbeiten}
\section{Peer-To-Peer-Systeme}
\chapter{RSS mittels verteiltem Publish/Subscribe}
\chapter{RSS}

\section{Das Verteilungsschema bei RSS und seine Problematik}
Ein Problem bei Dienstapplikationen im Internet ist eine starke Serverbelastung zu Sto�zeiten.
Die Masse der Anfragen an einen Server kann dazu f�hren, dass dieser der zeitgerechten Beantwortung nicht mehr nachkommen kann.
Im schlimmsten Fall
k�nnen einige Anfragen gar nicht beantwortet werden, da die Masse der Anfragen die zur Verf�gung gestellte Verarbeitungsapazit�t des Servers
�bersteigt. Ein Server unterh�lt im Regelfall eine Warteschlange (Queue), in der diejenigen Anfragen vorgehalten werden, welche aufgrund
anderer in Abfertigung befindlicher Anfragen momentan nicht bearbeitet werden k�nnen. Wird die Kapazit�tsgrenze dieser Warteschlange erreicht,
k�nnen weitere �bersch�ssige Anfragen nicht bearbeitet werden, sie werden verworfen.\\

Beim bestehenden Konzept zur Verteilung von RSS-Feeds kann es
vorkommen, dass Subscriber, deren Anfragen sich sehr weit hinten in der Warteschlange eines RSS-Servers befinden, erst sehr sp�t bzw. zu sp�t die gew�nschten
Informationen erhalten. Man denke z.B. an aktuelle B�rsennachrichten, welche im Sekundenbereich aktualisiert werden. Hier kann eine Nachricht
schon als veraltet gelten, erreicht sie den Interessenten einige Sekunden zu sp�t. Subscriber, deren Anfragen den RSS-Server bei bereits
ausgesch�pfter Kapazit�t der Warteschlange erreichen, kommen gar nicht an die gew�nschte Information.\\

Ein weiteres Problem im Zusammenhang mir RSS ist der erh�hte Verbrauch von Bandbreiten durch die �bertragung redundanter Daten. Die klientenseitige
Einstellung hoher Polling-Raten kann die wiederholte �bertragung unver�nderter RSS-Feeds provozieren. Diese zus�tzliche Belastung eines Web-Servers kann dazu
f�hren, dass dieser seiner normalen T�tigkeit nicht mehr zufriedenstellend nachkommen kann \cite{Hicks:2004:RSSBandwith}.
Zudem werden RSS-Feeds an jeden Klienten separat �bermittelt. Gleiche Interessen und nahegelegene Positionierung von Klienten im Internet k�nnen nicht ausgenutzt
werden, bzw. es obliegt Subnetzen durch die Einrichtung von Proxies und/oder Caches diese Vorteile auszunutzen. Um einige der oben genannten Probleme einzud�mmen,
wird ab der RSS-Version 0.92 der Parameter \texttt{cloud} unterst�tzt (siehe Abschnitt \ref{ch_rss}). Mit diesem Parameter kann ein Web-Service definiert werden,
bei dem sich Subscriber registrieren k�nnen, um �ber Aktualisierungen eines RSS-Feeds unterrichtet zu werden (dies wird auch als
\glqq lightweight publish/subscribe\grqq{} bezeichnet \cite{RSSSpecWi2004}). Somit kann unn�tiges Pollen eines RSS-Servers
vermieden werden. Doch k�nnen weitere, oben angesprochene Probleme damit nicht vermieden werden, und es ergeben sich zus�tzliche H�rden:
\begin{itemize}
\item Ein Publisher muss einen Web-Service einrichten, welcher Klienten �ber Aktualisierungen benachrichtigt.
\item Der Web-Service muss m�glicherweise eine Vielzahl von Klienten/Subscribern verwalten.
\item Umfangreichere Filterkriterien als nach bestehenden Kan�len auszuw�hlen, sind nicht m�glich.
\item Es wird nur ein unn�tiges Pollen verhindert. Klienten m�ssen nach wie vor den RSS-Server kontaktieren, um RSS-Feeds zu erhalten. Ein Austausch der Feeds
  unter den Klienten und somit eine Lastverteilung der Datenmengen ist damit nicht m�glich.
\end{itemize}

\section{Ziele}
\label{vs_ziele}
Unser prim�res Ziel ist es, ein Verteilungsschema zu konzipieren, bei dem jeder Subscriber die gew�nschten Informationen zeitgerecht erh�lt.
Die Definition von komplexen Filtern, die die Menge der Informationen einschr�nkt bzw. �ber die bestehenden Kan�le hinaus erweitert, soll erm�glicht werden.
Die sekund�ren Ziele ergeben sich im Verlauf der Arbeit.\\

Je fr�her ein Subscriber die neueste Information, die beim RSS-Server vorliegt, erh�lt, desto gr��er ist ihr Aktualit�tsgrad.
F�hren wir uns kurz vor Augen, welche Faktoren den Aktualit�tsgrad beeinflussen k�nnen. Denkbar sind drei Situationen, die den Aktualit�tsgrad
negativ beeinflussen:
\begin{enumerate}
  \item Anfrage des Subscribers an den RSS-Server kommt sp�t \label{enum:anf_zsp}
  \item Antwort des RSS-Servers erreicht den Subscriber sp�t \label{enum:ant_zsp}
  \item Subscriber erh�lt gar keine Antwort vom RSS-Server \label{enum:k_ant}
\end{enumerate}

Zu \ref{enum:anf_zsp}.: da wir es bei dem bestehenden Pull-Ansatz mit aktivem Polling der Subscriber zu tun haben, bewirkt eine h�here
Polling-Rate eine h�here Chance, dass Anfragen den RSS-Server mit geringer Verz�gerung erreichen (bezogen auf neue vorliegende Informationen).
Gegenma�nahme w�re also, die Polling-Rate eines Subscribers zu erh�hen, um die Anzahl der Anfragen, die den RSS-Server pro Zeiteinheit erreichen,
zu erh�hen.\\

Zu \ref{enum:ant_zsp}.: zu sp�t erhaltene Antworten k�nnen (abgesehen von langen �ber"-tra"-gungs"-zeiten) darauf zur�ckzuf�hren sein,
dass der RSS-Server mit der Beantwortung nicht nachkommt, seine Queue also zu voll ist. Abhilfe schafft hier eine Drosselung der Anzahl der
Anfragen an den RSS-Server.\\

Zu \ref{enum:k_ant}.: um trotzdem an die gew�nschten Informationen zu gelangen, muss daf�r gesorgt werden, dass der RSS-Server nicht die einzige
Quelle ist, von der jene Informationen bezogen werden k�nnen.\\

Die Ma�nahmen in den Punkten \ref{enum:anf_zsp}. und \ref{enum:ant_zsp}. widersprechen sich zun�chst. Eine Erh�hung der Anzahl der Anfragen kann also die Erf�llung
von \ref{enum:ant_zsp}. mit sich f�hren.
Zudem kann dies sogar die Erf�llung von \ref{enum:k_ant}. nach sich ziehen: bei einigen RSS-Servern ist vorgesehen, die
Anfragen von Subscribern, welche in zu geringen zeitlichen Abst�nden auf den RSS-Server treffen, zu blocken. Wir suchen also nach einer L�sung,
bei der sich die drei Punkte nicht gegenseitig negativ beeinflussen bzw. sich die Waage halten.

%%% Local Variables: 
%%% mode: latex
%%% TeX-master: "diplomarbeit"
%%% End: 

\section{Verwandte  Arbeiten}
\todo{f�llen}
\subsection{FeedTree}
\todo{f�llen}

\chapter{Adaptive Informationsverteilung}
\section{Das Verteilungsschema bei RSS und seine Problematik}
Ein Problem bei Dienstapplikationen im Internet ist eine starke Serverbelastung zu Sto�zeiten.
Die Masse der Anfragen an einen Server kann dazu f�hren, dass dieser der zeitgerechten Beantwortung nicht mehr nachkommen kann.
Im schlimmsten Fall
k�nnen einige Anfragen gar nicht beantwortet werden, da die Masse der Anfragen die zur Verf�gung gestellte Verarbeitungsapazit�t des Servers
�bersteigt. Ein Server unterh�lt im Regelfall eine Warteschlange (Queue), in der diejenigen Anfragen vorgehalten werden, welche aufgrund
anderer in Abfertigung befindlicher Anfragen momentan nicht bearbeitet werden k�nnen. Wird die Kapazit�tsgrenze dieser Warteschlange erreicht,
k�nnen weitere �bersch�ssige Anfragen nicht bearbeitet werden, sie werden verworfen.\\

Beim bestehenden Konzept zur Verteilung von RSS-Feeds kann es
vorkommen, dass Subscriber, deren Anfragen sich sehr weit hinten in der Warteschlange eines RSS-Servers befinden, erst sehr sp�t bzw. zu sp�t die gew�nschten
Informationen erhalten. Man denke z.B. an aktuelle B�rsennachrichten, welche im Sekundenbereich aktualisiert werden. Hier kann eine Nachricht
schon als veraltet gelten, erreicht sie den Interessenten einige Sekunden zu sp�t. Subscriber, deren Anfragen den RSS-Server bei bereits
ausgesch�pfter Kapazit�t der Warteschlange erreichen, kommen gar nicht an die gew�nschte Information.\\

Ein weiteres Problem im Zusammenhang mir RSS ist der erh�hte Verbrauch von Bandbreiten durch die �bertragung redundanter Daten. Die klientenseitige
Einstellung hoher Polling-Raten kann die wiederholte �bertragung unver�nderter RSS-Feeds provozieren. Diese zus�tzliche Belastung eines Web-Servers kann dazu
f�hren, dass dieser seiner normalen T�tigkeit nicht mehr zufriedenstellend nachkommen kann \cite{Hicks:2004:RSSBandwith}.
Zudem werden RSS-Feeds an jeden Klienten separat �bermittelt. Gleiche Interessen und nahegelegene Positionierung von Klienten im Internet k�nnen nicht ausgenutzt
werden, bzw. es obliegt Subnetzen durch die Einrichtung von Proxies und/oder Caches diese Vorteile auszunutzen. Um einige der oben genannten Probleme einzud�mmen,
wird ab der RSS-Version 0.92 der Parameter \texttt{cloud} unterst�tzt (siehe Abschnitt \ref{ch_rss}). Mit diesem Parameter kann ein Web-Service definiert werden,
bei dem sich Subscriber registrieren k�nnen, um �ber Aktualisierungen eines RSS-Feeds unterrichtet zu werden (dies wird auch als
\glqq lightweight publish/subscribe\grqq{} bezeichnet \cite{RSSSpecWi2004}). Somit kann unn�tiges Pollen eines RSS-Servers
vermieden werden. Doch k�nnen weitere, oben angesprochene Probleme damit nicht vermieden werden, und es ergeben sich zus�tzliche H�rden:
\begin{itemize}
\item Ein Publisher muss einen Web-Service einrichten, welcher Klienten �ber Aktualisierungen benachrichtigt.
\item Der Web-Service muss m�glicherweise eine Vielzahl von Klienten/Subscribern verwalten.
\item Umfangreichere Filterkriterien als nach bestehenden Kan�len auszuw�hlen, sind nicht m�glich.
\item Es wird nur ein unn�tiges Pollen verhindert. Klienten m�ssen nach wie vor den RSS-Server kontaktieren, um RSS-Feeds zu erhalten. Ein Austausch der Feeds
  unter den Klienten und somit eine Lastverteilung der Datenmengen ist damit nicht m�glich.
\end{itemize}

\section{Ziele}
\label{vs_ziele}
Unser prim�res Ziel ist es, ein Verteilungsschema zu konzipieren, bei dem jeder Subscriber die gew�nschten Informationen zeitgerecht erh�lt.
Die Definition von komplexen Filtern, die die Menge der Informationen einschr�nkt bzw. �ber die bestehenden Kan�le hinaus erweitert, soll erm�glicht werden.
Die sekund�ren Ziele ergeben sich im Verlauf der Arbeit.\\

Je fr�her ein Subscriber die neueste Information, die beim RSS-Server vorliegt, erh�lt, desto gr��er ist ihr Aktualit�tsgrad.
F�hren wir uns kurz vor Augen, welche Faktoren den Aktualit�tsgrad beeinflussen k�nnen. Denkbar sind drei Situationen, die den Aktualit�tsgrad
negativ beeinflussen:
\begin{enumerate}
  \item Anfrage des Subscribers an den RSS-Server kommt sp�t \label{enum:anf_zsp}
  \item Antwort des RSS-Servers erreicht den Subscriber sp�t \label{enum:ant_zsp}
  \item Subscriber erh�lt gar keine Antwort vom RSS-Server \label{enum:k_ant}
\end{enumerate}

Zu \ref{enum:anf_zsp}.: da wir es bei dem bestehenden Pull-Ansatz mit aktivem Polling der Subscriber zu tun haben, bewirkt eine h�here
Polling-Rate eine h�here Chance, dass Anfragen den RSS-Server mit geringer Verz�gerung erreichen (bezogen auf neue vorliegende Informationen).
Gegenma�nahme w�re also, die Polling-Rate eines Subscribers zu erh�hen, um die Anzahl der Anfragen, die den RSS-Server pro Zeiteinheit erreichen,
zu erh�hen.\\

Zu \ref{enum:ant_zsp}.: zu sp�t erhaltene Antworten k�nnen (abgesehen von langen �ber"-tra"-gungs"-zeiten) darauf zur�ckzuf�hren sein,
dass der RSS-Server mit der Beantwortung nicht nachkommt, seine Queue also zu voll ist. Abhilfe schafft hier eine Drosselung der Anzahl der
Anfragen an den RSS-Server.\\

Zu \ref{enum:k_ant}.: um trotzdem an die gew�nschten Informationen zu gelangen, muss daf�r gesorgt werden, dass der RSS-Server nicht die einzige
Quelle ist, von der jene Informationen bezogen werden k�nnen.\\

Die Ma�nahmen in den Punkten \ref{enum:anf_zsp}. und \ref{enum:ant_zsp}. widersprechen sich zun�chst. Eine Erh�hung der Anzahl der Anfragen kann also die Erf�llung
von \ref{enum:ant_zsp}. mit sich f�hren.
Zudem kann dies sogar die Erf�llung von \ref{enum:k_ant}. nach sich ziehen: bei einigen RSS-Servern ist vorgesehen, die
Anfragen von Subscribern, welche in zu geringen zeitlichen Abst�nden auf den RSS-Server treffen, zu blocken. Wir suchen also nach einer L�sung,
bei der sich die drei Punkte nicht gegenseitig negativ beeinflussen bzw. sich die Waage halten.

%%% Local Variables: 
%%% mode: latex
%%% TeX-master: "diplomarbeit"
%%% End: 

\section{Verteiltes Polling}
Betrachten wir die Gesamtheit der Subscriber und ihr Polling-Verhalten, so erkennen wir, dass aus Sicht eines RSS-Servers die Polling-Frequenz
dieser Gesamtheit (mittlere Ankunftsrate der Anfragen) gr��er ist als die Polling-Frequenz jedes einzelnen Subscribers.
W�nschenswert w�re es, wenn jeder Subscriber von der
Polling-Frequenz der Gesamtheit profitieren k�nnte, so dass die gesteigerte Frequenz zu einer erh�hten Aktualit�t der RSS-Feeds beim entsprechenden
Subscriber f�hrt. Es bietet sich an, die Subscriber �ber ein Overlay-Netzwerk zu verbinden, so dass die RSS-Feeds zwischen den Subscribern ausgetauscht werden
k�nnen. Das Polling wird also auf die beteiligten Subscriber verteilt.
Wir haben es dabei mit einer Kombination aus einem Pull- und einem Push-Ansatz zu tun. Allerdings �bernimmt die Push-Funktion nicht der RSS-Server,
sondern sie wird von den beteiligten Einheiten des Overlay-Netzes �bernommen, von dem der RSS-Server kein Bestandteil ist. Es stellt sich die Frage,
warum wir nicht gleich den Push-Ansatz bezogen auf den RSS-Server favorisieren. RSS ist eine schon seit l�ngerer Zeit bestehende Technik.
Dar�ber hinaus ist RSS weit verbreitet.
Eine Modifikation des Grundkonzeptes w�rde f�r Anbieter, die es unterst�tzen, bedeuten, bestehende Software austauschen zu m�ssen und
ganz neue Serviceleistungen bereitstellen zu m�ssen. Dies w�rde m�glicherweise auf Ablehnung sto�en und damit nicht die gew�nschte
Verbreitung des neuen Konzeptes mit sich bringen. Ziel ist es, auf dem bestehenden Konzept aufzubauen und es in ein erweitertes Konzept zu
integrieren, um f�r den Benutzer als auch f�r den Dienstanbieter m�glichst ein Minimum an Aufwand zu erreichen (Deolasee et al.
besch�ftigen sich in \cite{bhide02adaptive} mit einer Kombination aus Push-Pull und beschreiben die Probleme, die sich aus reinen Pull- bzw.
Push-Ans�tzen ergeben).
\subsubsection*{Einbettung in PubSub}
Es ist naheliegend, das Publish-Subscribe-Kommunikationsparadigma auf unser Problem anzuwenden: Die Informationen, konkret also die RSS-Feeds, sollen �ber ein
Notifikationssystem an die Interessenten (Subscriber) ausgeliefert werden. Da sich die Funktion bzw. Rolle des RSS-Servers nicht �ndern
soll, muss die Funktion des Publishers eine andere Einheit �bernehmen. Es bietet sich an, die Rolle des Publishers ebenfalls den Klienten
zuzuweisen. Ein Klient erh�lt in der Rolle des Subscribers nach einer Anfrage seinerseits den RSS-Feed von einem RSS-Server.
Der Klient kann nun diesen Feed in
der Rolle des Publishers in das Notifikationssystem einspeisen. Das Notifikationssystem soll aus einem System von vernetzten Brokern
bestehen. Ein Broker, welcher einen Feed erh�lt, liefert diesen an die �brigen mit ihm verbundenen Subscriber bzw. Broker aus. F�r einen
Subscriber gibt es also zwei M�glichkeiten, einen Feed zu erhalten: entweder auf direkte Anfrage von einem RSS-Server (Pull) oder �ber
das Notifikationssystem (Push). Um einen neuen Feed zu erhalten, kann ein Subscriber selbst aktiv werden und den Feed vom RSS-Server anfordern, oder er kann warten,
bis ihm ein neuer Feed durch das Netzwerk �ber einen Broker �bermittelt wird. Es ergeben sich dabei folgende Fragestellungen: wann soll ein
Klient aktiv werden, um den RSS-Server zu kontaktieren, und wann soll ein Klient inaktiv bleiben, um den Feed �ber das Brokernetzwerk zu erhalten? Denn
folgende Problemsituation ist denkbar: kontaktieren alle Klienten gleichzeitig den entsprechenden RSS-Server, so bringt dies keine Vorteile, da dies dem alten Ansatz
entspricht; erreicht der neue Feed
einen Klienten �ber einen Broker, so besitzt der Klient diesen neuen Feed bereits, da er zuvor eigenm�chtig den RSS-Server kontaktiert hat.
\todo{Filter}
\subsubsection*{Ansatz: Ein Publisher}
Die einfachste L�sung w�re, einen dedizierten Publisher zu bestimmen, welcher als alleiniger Klient das Polling �bernimmt. Alle weiteren Klienten
w�rden die Feeds �ber das Broker-Netzwerk erhalten. Vorteil w�re, da� es relativ wenig
Anfragen an den RSS-Server g�be. Die Nachteile �berwiegen jedoch: in Abh�ngigkeit von der Netzstruktur kann es zu langen �bertragungszeiten der Feeds f�r einzelne
oder mehrere Subscriber kommen. Wir m�ssten eine geringere Aktualit�t der Daten \todo{in Kauf} nehmen. Zudem h�tten \todo{}wir das Problem des
\glqq Single Point of Failure\grqq: f�llt der dedizierte Publisher aus, kann die Verbreitung der Feeds zun�chst nicht mehr erfolgen.
Erst, nachdem das Netzwerk den Ausfall registriert hat und entsprechende Gegenma�nahmen eingeleitet hat (z. B. Bestimmung eines neuen dedizierten
Publishers), kann es zu einer weiteren Verbreitung der Feeds kommen. Dar�ber hinaus k�nnten die Klienten nicht von verteiltem Polling profitieren.
\subsubsection*{Ansatz: Mehrere Publisher}
Daher favorisieren wir eine L�sung, bei der zu gegebener Zeit nur eine gewisse Auswahl der Klienten gleichzeitig Anfragen an den RSS-Server senden.
Es muss also zu einer Abstimmung bzw. Koordinierung der Klienten untereinander kommen, um diese Auswahl zu bestimmen. Auch hierbei ist zu beachten,
dass Ausf�lle von Publishern im Netz nicht zu Datenverlust f�hren sollen; d. h. jeder Subscriber sollte die Informationen bzw. Feeds, die er zu erhalten w�nscht,
auch dann erhalten, wenn die f�r das Polling vorgesehenen Klienten ausfallen.


\section{Koordinierung der Subscriber}

Um das Netz nicht noch zus�tzlich zu belasten, sollte die Netzbelastung, die durch eventuelle Abstimmungsnachrichten entsteht, minimal sein.
Die Konzeption eines Algorithmus sollte unter folgenden Gesichtspunkten erfolgen:
\begin{itemize}
  \item Polling durch mehrere bzw. wechselnde Klienten
  \item Anfragen an den RSS-Server sollten nicht gleichzeitig f�r alle Klienten geschehen 
  \item Ausfall von Klienten im Overlay-Netzwerk soll Informationsverteilung nicht blockieren
  \item Netzbelastung durch Abstimmungsnachrichten sollte gering gehalten werden
\end{itemize} 
Im Folgenden beschreiben wir einen Algorithmus bzw. eine Technik, die unsere bisher gestellten Anforderungen erf�llt.
\subsection{Der Grundlegende Algorithmus}
\label{cs:der_grundlegende_algorithmus}
Es sei $t_0$ immer der aktuelle Zeitpunkt. Ausgehend von einem beliebigen Zeitpunkt
$t_x$ mit $t_0\leq t_x$ und einer Intervallspanne $\Delta Z$ w�hlt sich jeder
Subscriber $i$ innerhalb des Zufallsintervalls $Z:=[t_x,t_x+\Delta Z]$ einen zuf�lligen Zeitpunkt $ttr_i$ (``time to tefresh'', $ttr$ im allgemeinen), zu dem
er den aktuellen Feed vom RSS-Server erfragt (siehe Abb. \ref{Abb:determine_ttr}). Im Folgenden nennen wir $t_x$ Einstiegspunkt und $\Delta Z$ Zufallsspanne.

\begin{picturehere}{3}{1.5}{$ttr$s}{Abb:determine_ttr}
 
%\psset{xunit=1cm,yunit=1cm,runit=1cm}
%\begin{picture}(1.5,-0.5)(7,1)
\begin{picture}(7,1)(1.5,-0.5)
  \put(0,0){\vector(1,0){7}}
  \put(0,-0.2){\line(0,1){0.4}}
  \put(0,-0.5){$t_0$}
  \put(3,-0.2){\line(0,1){0.4}}
  \put(3,-0.5){$t_x$}
  \put(6,-0.2){\line(0,1){0.4}}
  \put(6,-0.5){$t_x+\Delta Z$}
  \put(5,-0.1){\line(0,1){0.2}}
  \put(4.5,0.4){$ttr_i$}
  \put(7.8,0){$time$}
\end{picture}
% \includegraphics{determine_ttr}
\end{picturehere}


Ist $ttr_i$ erreicht, so erfragt Subscriber $i$ den aktuellen Feed vom RSS-Server und setzt nun $ttr_i$ auf einen
Zufallswert innerhalb des
Zeitintervalls $Z:=[t_x,t_x+\Delta Z]$, wobei $t_x$ ebenfalls neu gew�hlt wird.
Erh�lt Subscriber $i$ vor dem Erreichen des Zeitpunktes $ttr_i$ einen Feed $feed_{new}$ von einem Broker zum 
Zeitpunkt $t_f$ (sei $feed_{old}$ der bisher bei $i$ gespeicherte Feed), so geschieht folgendes:
\pagebreak[3]
\begin{description}[\compact]
  \item [Fall I:] $feed_{new}$ ist nicht aktueller als $feed_{old}$:
    \begin{description}[\breaklabel\compact]
      \item keine �nderungen
    \end{description}
  \item[Fall II:] $feed_{new}$ ist aktueller als $feed_{old}$:
    \begin{description}[\breaklabel\compact]
      \item w�hle $t_x$ neu mit $t_0\leq t_x$
      \item  $ttr_i$ wird auf einen Zufallswert gesetzt innerhalb des Zeitintervalls\\
        \mbox{$Z:=[t_x,t_x+\Delta Z]$}
    \end{description}
\end{description}

Bezeichne $\Delta ttr_i$ die Zeitspanne zwischen $t_0$ und $ttr_i$ ($\Delta ttr$ im allgemeinen), also gilt $ttr_i:=t_0+\Delta ttr_i$.
Die $ttr$s der verschiedenen Subscriber sollten bei der Wahl einer geeigneten Zufallsfunktion �ber $Z$ gleichm��ig
verteilt sein. Durch die Wahl eines zuf�lligen Wertes innerhalb von $Z$ ist gew�hrleistet, dass nur in extremen Ausnahmef�llen (theoretisch) 
alle Klienten gleichzeitig den RSS-Server kontaktieren.  Nat�rlich kann es vorkommen, dass $ttr$s verschiedener
Subscriber auf den gleichen Zeitpunkt fallen (je nach Gr��e der Zufallsspanne $\Delta Z$ und der Anzahl der Klienten).
Die Verteilung unterliegt jedoch einem kontinuierlichen Wechsel, da die $ttr$s immer
wieder neu berechnet werden. Ausgehend von $t_x$ bildet $\Delta Z$ eine obere Schranke f�r den Erhalt des n�chsten Feeds, da jeder Klient nach
sp�testens der Zeit $\Delta Z$ selbst�ndig den Server kontaktiert, falls in der Zwischenzeit kein aktueller Feed erhalten wurde. Dadurch k�nnen lange
�bertragungszeiten zwischen den Klienten ausgeglichen werden.
Ausf�lle von Klienten k�nnen zwar zu Verz�gerungen beim
Erhalt der Feeds f�hren, sie k�nnen aber die �bermittlung der Feeds zwischen den �brigen Klienten nicht st�ren,
solange physikalisches Netz und Brokernetz intakt sind.
\subsection{Konkrete Anpassung an RSS -- Bestimmung relevanter Parameter}
Im Folgenden betrachten wir, wie sich die relevanten Parameter in Zusammenhang mit RSS bestimmen lassen.
\subsubsection{Bestimmung des Einstiegspunktes}
L�sst sich der Zeitpunkt $nextBuild$ (Neue-Info-Punkt), zu dem der RSS-Server einen neuen Feed
bereitstellt, innerhalb eines gewissen Toleranzbereiches genau bestimmen, dann k�nnen wir den Einstiegspunkt $t_x:=nextBuild$ setzen. Kann
$nextBuild$ innerhalb des gew�nschten Toleranzbereiches nicht genau bestimmt werden, kann es n�tig sein $t_x:=t_0$ zu setzen. Unter welchen
Umst�nden welche Variante vorzuziehen ist, werden wir sp�ter noch er�rtern.
\subsubsection{Bestimmung des Neue-Info-Punktes}
Um $nextBuild$ zu bestimmen, definieren wir zwei weitere Parameter: $ttl$ und $lastBuildDate$. $ttl$ steht f�r Time-To-Live und bezeichnet
die Zeit, die ein Feed aktuell bleibt, bevor er Server-seitig aktualisiert wird. $lastBuildDate$ steht f�r den Zeitpunkt, zu dem ein
Feed vom Server aktualisiert wurde.
Der RSS 2.0 Standard\cite{RSSSpecWi2004} sieht unter anderem die optionalen Parameter $lastBuildDate$ und $pubDate$ vor. Setzen wir voraus,
dass mindestens der Parameter $lastBuildDate$ vom Server bereitgestellt wird.
(Beschreibung siehe Kapitel \ref{ch_rss} auf Seite \pageref{op_rss}). $nextBuild$ l�sst sich
aufgrund des letzten aktuellen Feeds wie folgt berechnen:
\pagebreak[3]
\[nextBuild:=t_0+\Delta t\] mit \[\Delta t:=\left\{\begin{array}{r@{\quad:\quad}l}
    0 & (t_0-lastBuildDate)>ttl \\ttl-(t_0-lastBuildDate) & sonst
  \end{array}\right. \]

Alternativ k�nnte statt $lastBuildDate$ auch $pubDate$ zur Berechnung genommen werden.
\subsubsection{Bestimmung von Time-To-Live}
Um $ttl$ zu bestimmen, gibt es zwei M�glichkeiten:
\begin{itemize}
  \item {\bf Bereitstellung des $ttl$ durch den Informationsanbieter:}
    RSS 2.0\cite{RSSSpecWi2004} sieht ebenfalls den optionalen Parameter $ttl$ vor.

  \item {\bf Bestimmung des $ttl$ durch den Klienten:}
    Wird der Parameter $ttl$ vom Informationsanbieter nicht unterst�tzt, so kann $ttl$ heuristisch durch den Klienten bestimmt werden.
\end{itemize}

Wie wir sehen, sind $ttl$ und $nextBuild$ eng miteinander verkn�pft. Wollen wir $ttl$ und damit $nextBuild$ ermitteln k�nnen, stellt sich zun�chst die Frage,
ob und in welchen F�llen dies �berhaupt sinnvoll ist. Informationen k�nnen vielf�ltiger Art sein, Informationsanbieter k�nnen ganz unterschiedliche Gewohnheiten an
den Tag legen. Es h�ngt von der Vorhersagbarkeit des Auftretens neuer Daten und der zeitlichen M�glichkeit ab, diese Daten bereit zu stellen, mit welcher G�te
der $ttl$ berechnet werden kann. Stellen wir uns eine
Person vor, die regelm��ig jeden Tag ihr Tagebuch in einem Blog samt RSS-Feeds ver�ffentlicht. Sie besitzt ein nicht besonders
leistungsf�higes Rechnersystem, welches bei einer gro�en und dauerhaften Anzahl von Webzugriffen schnell �berlastet wird. Die Person steht jeden Tag um 8.00 Uhr auf,
so dass sie um 9.00 Uhr die Eintr�ge des vorherigen Tages bereit gestellt hat. Sie kann somit in den RSS-Feed einen $ttl$-Wert von 24 Stunden eintragen. So wie es
aussieht, spielt Aktualit�t in diesem Fall keine gro�e Rolle, so dass f�r die Interessenten ein relativ gro�er $\Delta Z$ Wert festgelegt werden kann (z. B. 12
Stunden). Ein RSS-Reader eines Interessenten braucht somit fr�hsten um 9.00 beim Anbieter nachzufragen und hat einen Spielraum von 12 Stunden. Mit dem von uns
geplanten Pub/Sub-RSS-System reichen in diesem Falle schon sehr wenige Subscriber aus (vielleicht sogar nur einer), um den aktuellen Feed an die gesamte Fangemeinde
zu �bermitteln. Betrachten wir nun einen anderen Fall: eine Nachrichtenagentur stellt rund um die Uhr die neuesten Schlagzeilen in einem RSS-Feed zur Verf�gung. Es
ist nicht absehbar, wann ein neues Weltereignis eintritt, so dass die Nachrichtenagentur nicht plant, den RSS-Feed mit dem Wert $ttl$ zu versorgen. Eine
heuristische Bestimmung des $ttl$ durch den Klienten ist wahrscheinlich mit einer gro�en Varianz behaftet und dadurch sehr ungenau. Und dennoch ist der Spielraum
gro�, was eine empirische Datenerhebung verdeutlicht.

\importgnuplotps{RSS-Feed-Aktualisierung}{Abb:rss_aktualisierung}{rss_aktualisierung}

Abbildung \ref{Abb:rss_aktualisierung} zeigt, wie oft und regelm��ig verschiedene Anbieter von RSS-Feeds (Spiegel, Heise, NY-Times, Slashdot, Sourceforge)
diese aktualisieren. Der gemessene Zeitraum erstreckt sich �ber
24 Stunden, die Abtastrate betrug 60 Sekunden. Hierbei f�llt auf, dass Spiegel und Heise in der Zeit zwischen ca. 0.00 und 5.00 Uhr keine Aktualisierungen
vornehmen, wogegen zu den �brigen Zeiten die Aktualisierungsintervalle schwanken. Zur Nachtzeit w�rde es sich also anbieten, den $ttl$ zu setzten. Auch bei
den NY-Times f�llt ein Zeitraum auf, indem nicht aktualisiert wird. Die zeitliche Differenz zu den deutschen Betreibern l�sst sich vermutlich durch eine
Zeitverschiebung erkl�ren. Bei den NY-Times f�llt weiterhin auf, dass in der �brigen Zeit Aktualisierungen nur st�ndlich vorgenommen werden. Also auch hier ein
Fall f�r einen vom Anbieter vorgegebenen $ttl$. Ebenfalls l�sst sich bei Slashdot und Sourceforge eine gewisse Linearit�t der Aktualisierungsintervalle 
feststellen, wenn sie auch um einiges k�rzer sind.


\subsubsection{Heuristische Bestimmung von Time-To-Live}
Hierzu gibt es verschiedene Verfahren. Um $ttl$
berechnen zu k�nnen, muss zun�chst die Rate gesch�tzt werden, mit der Feeds Server-seitig aktualisiert werden. Daf�r misst ein Subscriber innerhalb eines
Zeitintervalles $T$ die Anzahl $X$ der aufgetretenen Aktualisierungen eines Feeds. Eine Aktualisierung wird dann festgestellt, wenn ein Subscriber eine neuen Feed
erh�lt. Dabei kann das Attribut $PubDate$ der einzelnen Ereignisse (Items) eines RSS-Feeds herangezogen werden, um eine feinere Bestimmung der Aktualisierungen
vorzunehmen. Jedes neue Event steht dabei f�r eine Aktualisierung. Bei Eintritt eines Subscribers in das Netzwerk sollte der $ttl$ zun�chst auf $0$ gesetzt werden,
er wird dann w�hrend der Zeit, die sich ein Subscriber aktiv im Overlay-Netzwerk befindet, angepasst. Nat�rlich kann der errechnet Wert bei Verlassen des
Systems zwischengespeichert werden, damit er beim n�chsten Eintritt in das System wieder zur Verf�gung steht.\\

Zun�chst beschreiben wir eine simple und intuitive Methode, welche jedoch starke Verzerrungen aufweisen kann. Im Anschluss daran werden wir ein verbessertes
Verfahren vorstellen, welches von Cho und Garcia-Molina entwickelt wurde.
\paragraph{IntuitiveMethode:}
$\hat\mu_r:=\frac{X}{T}$ liefert eine gesch�tzte Aktualisierungsrate der Feeds. Das Verh�ltnis zwischen der tats�chlichen Aktualisierungsrate $\mu$ und der
Abtastrate $f$ (Anzahl der erhaltenen RSS-Feeds bzw. Ereignisse pro Zeiteinheit) $r:=\frac{\mu}{f}$ kann �ber die G�te von $\hat\mu$ Auskunft geben: gilt $r>1$,
so hat es mehr Aktualisierungen als Zugriffe (Feeds) gegeben, und der berechnete Wert $\hat\mu$ weist eine gewisse Ungenauigkeit auf. Liegt die gesamte Historie der
Akzualisierungen vor, so ist $\frac{X}{T}$ ein guter Sch�tzwert \cite{ChGM:2003:ChangeFrequency}. Da innerhalb eines Feeds mehrere Ereignisse (Items)
zusammengefasst sind, ist die Wahrscheinlichkeit geringer, dass Aktualisierungen verloren gehen, als wenn ein Feed nur ein Ereignis beinhalten w�rde.
Falls jedoch $\varDelta Z$ und $cpp$ sehr gro� gew�hlt sind bei einer gleichzeitig geringen Anzahl von Subscribern im Netzwerk, k�nnen neue Ereignisse
verloren gehen.

\paragraph{Verbesserte Methode:}
Um eine bessere Ann�herung von $\hat\mu$ an $\mu$ zu erreichen, haben Cho und Garcia-Molina in \cite{ChGM:2003:ChangeFrequency} ein anderes Verfahren zu Bestimmung
von Aktualisierungsraten entwickelt (entgegen der Berechnung bei Cho und Garcia-Molina haben wir die Aktualisierungsrate statt $\lambda$ mit $\mu$ bezeichnet, da
$\lambda$ in unserem Kontext schon belegt ist). Dabei gehen sie von der Annahme bzw. Beobachtung aus, dass die Aktualisierungsrate von Web-Inhalten durch einen
Poisson-Prozess bestimmt wird. Diese Beobachtung l�sst sich auf die von uns betrachteten RSS-Feeds �bertragen, da es sich bei diesen technisch gesehen ebenfalls
um Web-Inhalte handelt. Eine genaue Herleitung und Beschreibung des Verfahrens geht �ber den Rahmen dieser Arbeit hinaus und findet sich
in \cite{ChGM:2003:ChangeFrequency}.\\
Innerhalb des Zeitintervalls $[t;t+1]$ wird $\mu$ geliefert durch den Erwartungswert
\[E[X(t+1)-X(t)]=\sum^\infty_{k=0}k\frac{\mu^k e^{-\mu}}{k!}=\mu.\]
Dann wird bei einer unvollst�ndigen Historie der Aktualisierungen ein besserer Sch�tzwert geliefert durch:
\[\hat\mu:=-log\left(\frac{\bar X-0.5}{n-0.5}\right)\]
wobei $n$ die Anzahl der Zugriffe (also Feeds bzw. Ereignisse innerhalb eines Feeds) und $\bar X:=n-X$ die Anzahl der Zugriffe ohne Aktualisierungen ist.\\

Ein noch besserer Sch�tzwert kann geliefert werden, falls der Zeitpunkt der letzten Aktualisierung bekannt ist. Dieser ist durch das Attribut $PubDate$ bei RSS-Feeds
gegeben. Cho und Garcia-Molina beschreiben daf�r in \cite{ChGM:2003:ChangeFrequency} folgenden Algorithmus.

\begin{verbatim}
Init() /* initialize variables */ 
  N = 0; /* total number of accesses */ 
  X = 0; /* number of detected changes */ 
  T = 0; /* sum of the times from changes */ 

Update(Ti, Ii) /* update variables */ 
  N = N + 1; 
  /* Has the element changed? */ 
  If (Ti < Ii) then 
  /* The element has changed. */ 
  X = X + 1; 
  T = T + Ti; 
  else 
  /* The element has not changed */ 
  T = T + Ii; 

Estimate() /* return the estimated lambda */ 
  X� = (X-1) - X/(N*log(1-X/N));
  return X�/T;

\end{verbatim} 

Dabei dient {\ttfamily Init()} zur einmaligen Initialisierung der Variablen auf null. Bei jedem Zugriff auf ein Element (Erhalt eines Feeds in unserem Fall) wird
{\ttfamily Update()} aufgerufen. {\ttfamily Ti} ist das Zeitintervall bis zur letzten Aktualisierung beim $i$ten Zugriff, {\ttfamily Ii} das Intervall zwischen
den Zugriffen. 
\input{bestimmung_der_intervallspanne_deltai}



\chapter{Anpassung an das �bertragungsprotokoll TCP}
\label{c:anpassung_an_tcp}
\chapter{Implementierung der Simulationsumgebung}
\chapter{Experimente und Auswertung}
\label{c:experimente}
\section{Gesichtspunkte der Experimente}
\chapter{Zusammenfassung und Ausblick}

\backmatter
%\newpage

\addcontentsline{toc}{section}{Glossar}
\glentry{ack}{Acknowledgement: Best�tigungsnachricht f�r erhaltenes Datenpaket}
\glentry{Z}{Zufallsintervall}
\glentry{$\varDelta Z$}{Zufallsspanne}
\glentry{$\varDelta ttr$}{Zeit zwischen $t_0$ und $ttr$}
\glentry{$\varDelta ttl$}{Zeit zwischen $t_0$ und $nextBuild$}
\glentry{$\lambda$}{mittlere Ankunftsrate der Anfragen}
\glentry{$\bar x$}{mittlere Bearbeitungszeit pro Anfrage}
\glentry{$\rho$}{utilization factor $:=\lambda\bar x$}
\glentry{$\Gamma_V$}{Funktion zur Bestimmung des Verz�gerungsgrades von RSS-Feeds}
\glentry{$\Gamma_A$}{Funktion zur Bestimmung des Aktualit�tsgrades von RSS-Feeds}
\glentry{IP}{Internet Protocol}
\glentry{TCP}{Transmission Control Protocol}
\glentry{UDP}{User Datagram Protocol}
\glentry{DHT}{Distributed Hashtable}
\glentry{ttl}{time to live: Zeitraum, indem eine Information als aktuell eingestuft wird und sich voraussichtlich nicht �ndern wird}
\glentry{ttr}{time to refresh: Zeitpunkt einer erneuten Anfrage}
\glentry{t$_x$}{Einstiegspunkt}
\glentry{t$_0$}{aktueller Zeitpunkt}
\glentry{nextBuild}{Zeitpunkt, an dem ein Server voraussichtlich neue Informationen bereitstellen wird}
\glentry{rto}{retransmission timeout intervall: Zeit bis zum erneuten Aussenden einer Anfrage}
\glentry{rtt}{N�herungswert f�r die roundtrip-time}
\glentry{artt}{skalierter rtt}
\glentry{srtt}{smoothed roundtrip time: gegl�tteter Wert bei der Berechnung des rtt zur Stauvermeidung bei TCP}
\glentry{serviceTimeFactor}{Simulationsparameter: Faktor f�r die Bearbeitungszeit pro Feed-Request; mit seiner Hilfe kann eine vor�bergehende
        vermehrte Serverbelastung bzw. ein weniger leistungsf�higer Hostrechner simuliert werden.}
\glentry{Aussendung}{Aussendung eines Datenpakets bzw. einer Anfrage an einen RSS-Server}
\glentry{Wiederholung}{Wiederholte Aussendung eines Datenpakets bzw. einer Anfrage an einen RSS-Server}
\glentry{roundtrip-time}{Zeit zwischen dem Versenden einer Anfrage und dem Erhalt der Antwort}
\glentry{Feed-Request}{eine Anfrage an den RSS-Server nach einem RSS-Feed}
\glentry{feed.artt}{$artt$ als Bestandteil eines erweiterten Feeds}
\glentry{RQT}{Request-Timer: Timer, nach dessen Ablauf ein Feed-Request ausgesandt wird}
\glentry{RT}{Retransmission-Timer: Timer, nach dessen Ablauf ein Feed-Request erneut ausgesandt wird (ohne Erhalt eines RSS-Feeds)}
\glentry{ppp}{bevorzugte Polling-Periode}
\glentry{cpp}{aktuelle Polling-Periode}
\glentry{mpp}{maximale Polling-Periode}
\glentry{icpp}{initial-cpp: der f�r die Berechnung von $rto$ und $rtt$ grundlegende Wert; wird durch einen $feed.rtt$ nicht modifiziert}

\glentry{RSS}{Relly Simple Syndication: Technik zur Bereitstellung von Kurznachrichten im WorldWideWeb}
\glentry{Blog}{auch Weblog: Webseite, die periodisch neue Eintr�ge enth�lt}
\glentry{RDF}{Resource description Framework: fromale Sprache zur Beschreibung von Webinhalten}
\glentry{URL}{Uniform Resource Locator: Zeichenkette, die eine Ressource in Computernetzwerken identifiziert}
\glentry{WWW}{World Wide Web}
\glentry{Id}{Identifier bzw. Bezeichner}
\printglossary


\bibliography{../bibdatabase}

\end{document}
