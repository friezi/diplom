\subsubsection{Heuristische Bestimmung von Time-To-Live}
Hierzu gibt es verschiedene Verfahren. Um $ttl$
berechnen zu k�nnen, muss zun�chst die Rate gesch�tzt werden, mit der Feeds Server-seitig aktualisiert werden. Daf�r misst ein Subscriber innerhalb eines
Zeitintervalles $T$ die Anzahl $X$ der aufgetretenen Aktualisierungen eines Feeds. Eine Aktualisierung wird dann festgestellt, wenn ein Subscriber eine neuen Feed
erh�lt. Dabei kann das Attribut $PubDate$ der einzelnen Ereignisse (Items) eines RSS-Feeds herangezogen werden, um eine feinere Bestimmung der Aktualisierungen
vorzunehmen. Jedes neue Event steht dabei f�r eine Aktualisierung. Bei Eintritt eines Subscribers in das Netzwerk sollte der $ttl$ zun�chst auf $0$ gesetzt werden,
er wird dann w�hrend der Zeit, die sich ein Subscriber aktiv im Overlay-Netzwerk befindet, angepasst. Nat�rlich kann der errechnet Wert bei Verlassen des
Systems zwischengespeichert werden, damit er beim n�chsten Eintritt in das System wieder zur Verf�gung steht.\\

Zun�chst beschreiben wir eine simple und intuitive Methode, welche jedoch starke Verzerrungen aufweisen kann. Im Anschluss daran werden wir ein verbessertes
Verfahren vorstellen, welches von Cho und Garcia-Molina entwickelt wurde.
\paragraph{IntuitiveMethode:}
$\hat\mu_r:=\frac{X}{T}$ liefert eine gesch�tzte Aktualisierungsrate der Feeds. Das Verh�ltnis zwischen der tats�chlichen Aktualisierungsrate $\mu$ und der
Abtastrate $f$ (Anzahl der erhaltenen RSS-Feeds bzw. Ereignisse pro Zeiteinheit) $r:=\frac{\mu}{f}$ kann �ber die G�te von $\hat\mu$ Auskunft geben: gilt $r>1$,
so hat es mehr Aktualisierungen als Zugriffe (Feeds) gegeben, und der berechnete Wert $\hat\mu$ weist eine gewisse Ungenauigkeit auf. Liegt die gesamte Historie der
Akzualisierungen vor, so ist $\frac{X}{T}$ ein guter Sch�tzwert \cite{ChGM:2003:ChangeFrequency}. Da innerhalb eines Feeds mehrere Items zusammengefasst sind,
ist die Wahrscheinlichkeit geringer, dass Aktualisierungen verloren gehen. Falls jedoch $\varDelta Z$ und $cpp$ sehr gro� gew�hlt sind bei einer gleichzeitig
geringen Anzahl von Subscribern im Netzwerk, k�nnen dennoch neue Ereignisse verloren gehen.

\paragraph{Verbesserte Methode:}
Um eine bessere Ann�herung von $\hat\mu$ an $\mu$ zu erreichen, haben Cho und Garcia-Molina in \cite{ChGM:2003:ChangeFrequency} ein anderes Verfahren zu Bestimmung
von Aktualisierungsraten entwickelt (entgegen der Berechnung bei Cho und Garcia-Molina haben wir die Aktualisierungsrate statt $\lambda$ mit $\mu$ bezeichnet, da
$\lambda$ in unserem Kontext schon belegt ist). Dabei gehen sie von der Annahme bzw. Beobachtung aus, dass die Aktualisierungsrate von Web-Inhalten durch einen
Poisson-Prozess bestimmt wird. Diese Beobachtung l�sst sich auf die von uns betrachteten RSS-Feeds �bertragen, da es sich bei diesen technisch gesehen ebenfalls
um Web-Inhalte handelt. Eine genaue Herleitung und Beschreibung des Verfahrens geht �ber den Rahmen dieser Arbeit hinaus und findet sich
in \cite{ChGM:2003:ChangeFrequency}.\\
Innerhalb des Zeitintervalls $[t;t+1]$ wird $\mu$ geliefert durch den Erwartungswert
\[E[X(t+1)-X(t)]=\sum^\infty_{k=0}k\frac{\mu^k e^{-\mu}}{k!}=\mu.\]
Dann wird bei einer unvollst�ndigen Historie der Aktualisierungen ein besserer Sch�tzwert geliefert durch:
\[\hat\mu:=-log\left(\frac{\bar X-0.5}{n-0.5}\right)\]
wobei $n$ die Anzahl der Zugriffe (also Feeds bzw. Ereignisse innerhalb eines Feeds) und $\bar X:=n-X$ die Anzahl der Zugriffe ohne Aktualisierungen ist.\\

Ein noch besserer Sch�tzwert kann geliefert werden, falls der Zeitpunkt der letzten Aktualisierung bekannt ist. Dieser ist durch das Attribut $PubDate$ bei RSS-Feeds
gegeben. Cho und Garcia-Molina beschreiben daf�r in \cite{ChGM:2003:ChangeFrequency} folgenden Algorithmus.

\begin{verbatim}
Init() /* initialize variables */ 
  N = 0; /* total number of accesses */ 
  X = 0; /* number of detected changes */ 
  T = 0; /* sum of the times from changes */ 

Update(Ti, Ii) /* update variables */ 
  N = N + 1; 
  /* Has the element changed? */ 
  If (Ti < Ii) then 
  /* The element has changed. */ 
  X = X + 1; 
  T = T + Ti; 
  else 
  /* The element has not changed */ 
  T = T + Ii; 

Estimate() /* return the estimated lambda */ 
  X� = (X-1) - X/(N*log(1-X/N));
  return X�/T;

\end{verbatim} 
