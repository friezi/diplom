\section{Rolle der Broker}
\todo{Literaturangaben}\\
Broker sind der zentrale Bestandteil des Notifikationssystems. Ein Broker empf�ngt Feeds von Brokern oder von Klienten/Publishern.
Anschlie�end sorgt er f�r eine Verteilung der
Feeds an eine bestimmte Auswahl von mit ihm verbundenen Brokern bzw. Subscribern. Ein Broker kann also eine zentrale Sammelstelle f�r Feeds
unterschiedlicher Anbieter sein. Die einzelnen in den Feeds zusammengefassten Ereignisse k�nnen durch den Broker neu zusammengestellt werden.
Als Kriterien f�r neue Zusammenstellungen bieten sich die Aktualit�t der einzelnen Ereignisse sowie bestimmte Filterregeln an. Einzelne 
Ereignisse, die den Broker bereits erreicht haben, brauchen nicht erneut weitergeleitet zu werden und finden daher nach unserem Konzept keinen
Eingang in die neu zusammengestellten Feeds. Dadurch k�nnen Netzressourcen gespart werden. Voraussetzung daf�r ist, dass der Broker einen Cache
unterh�lt, in dem einige Ereignisse zwischengespeichert werden.\\

Filter k�nnen durch Subscriber definiert und bei Brokern
hinterlegt werden. Aufgrund der Filterregeln k�nnen Ereignisse verschiedener
Anbieter aus den Feeds extrahiert und in einem neuen Feed gesammelt werden. Ein
Subscriber kann also eine individuelle Zusammenstellung der Ereignisse erhalten. Filterregeln und ihre Anwendung wurden schon ausf�hrlich
erforscht und sollen nicht Gegenstand dieser Arbeit sein. Deshalb fanden sie auch keinen Eingang in die weiter unten beschriebene
Simulationsumgebung. Sie erweitern die M�glichkeiten des Systems lediglich, haben aber keinen Einfluss auf das Grundkonzept (um das es hier geht).
Unser System \pubsubrss k�nnte mit bestehenden Pub/Sub-Systemen, welche Filtertechniken unterst�tzen (wie z. B. das System
REBECA \cite{MuFiBu:2001:ArchFrameECommApp}), kombiniert werden.\\
\todo{Hier folgen noch Bez�ge zur geplanten �bersicht �ber Filter im 1. Kapitel}

Um sich dem entsprechenden Broker bekannt zu machen und Filter zu hinterlegen, muss sich ein Subscriber bei diesem Broker mit einer speziellen
Registrierungsnachricht registrieren.
Ein Broker braucht nur an diejenigen Subscriber aktuelle Feeds zu �bermitteln, welche auch tats�chlich online sind. Zus�tzlich ist es also notwendig,
dass ein Subscriber in regelm��igen Zeitabst�nden den jeweiligen Broker �ber seinen Online-Status unterrichtet (nennen wir eine solche Nachrichte
``KEEPALIVE-Nachricht''). Erh�lt ein Broker eine KEEPALIVE-Nachricht eines Subscribers, f�r den er den Zustand ``ist offline'' gespeichert hat,
sollte er diesem eine Zusammenstellung der aktuellsten Ereignisse �bermitteln. Denn kommt es aufgrund von St�rungen im physischen Netzwerk oder
aufgrund von Netz�berlastung zu verloren gegangenen KEEPALIVE-Nachrichten, so gilt der Subscriber f�r seinen Broker als ``offline'', dieser wird ihm
daraufhin keine Feeds mehr �bermitteln.
Hat der Subscriber aufgrund von adaptiven Ma�nahmen seine aktuelle Polling-Periode stark angehoben, so wird er zun�chst keine weiteren Feeds beim
entsprechenden RSS-Server selbst�ndig erfragen, so dass ihm eventuell Informationen verloren gehen. Die Menge verloren gegangener Informationen
kann geringer gehalten werden, wenn dem Subscriber die letzten Ereignisse bei Wiedereintritt in das Overlay-Netzwerk als Feed �bermittelt werden.
Entsprechend sollte ein Subscriber seinen jeweiligen Broker dar�ber informieren, wenn er offline geht, um unn�tigen Datentransfer zu vermeiden.\\

Im Zusammenhang mit Brokern werden noch einige weitere Ma�nahmen und Nachrichtentypen vonn�ten sein, auf die wir jedoch erst in sp�teren Kapiteln zu sprechen
kommen werden, da sie Anpassungen an spezielle Bed�rfnisse darstellen (siehe Kapitel ``Churn'' \ref{cs:churn}) .