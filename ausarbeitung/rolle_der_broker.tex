\section{Rolle der Broker}
\todo{folgt}
Die Broker sind der zentrale Bestandteil des Notifikationssystems. Ein Broker empf�ngt Feeds von Brokern oder von Subscribern.
Anschlie�end sorgt er f�r eine Verteilung der
Feeds an eine bestimmte Auswahl von mit ihm verbundenen Brokern bzw. Subscribern. Ein Broker kann also eine zentrale Sammelstelle f�r Feeds
unterschiedlicher Anbieter sein. Die einzelnen in den Feeds zusammengefassten Ereignisse k�nnen durch den Broker neu zusammengestellt werden.
Als Kriterien f�r neue Zusammenstellungen bieten sich die Aktualit�t der einzelnen Ereignisse sowie bestimmte Filterregeln an. Einzelne 
Ereignisse, die den Broker bereits erreicht haben, brauchen nicht erneut weitergeleitet zu werden und finden daher nach unserem Konzept keinen
Eingang in die neu zusammengestellten Feeds. Dadurch k�nnen Netzresourcen gespart werden. Voraussetzung daf�r ist, dass der Broker einen Cache
unterh�lt, in dem eine bestimmte Menge an Ereignissen zwischengespeichert werden. Filter k�nnen durch Subscriber definiert und beim Broker
hinterlegt werden. Aufgrund der Filterregeln k�nnen Ereignisse verschiedener Anbieter z. B. themenbasiert in einem Feed gesammelt werden. Ein
Subscriber kann also eine individuelle Zusammenstellung der Ereignisse erhalten. Filterregeln und ihre Anwendung wurden schon ausf�hrlich
erforscht und sollen nicht Gegenstand dieser Arbeit sein. Deshalb fanden sie auch keinen Eingang in die weiter unten beschriebene
Simulationsumgebung. Unser System \pubsubrss k�nnte mit bestehenden Pub/Sub-Systemen, welche Filtertechniken unterst�tzen (wie z. B. das System
REBECA \cite{MuFiBu:2001:ArchFrameECommApp}), kombiniert werden.