\chapter{Anhang}
\label{anhang}
%\addcontentsline{toc}{section}{Herleitungen}

\addcontentsline{toc}{section}{Legende der Diagramme}
\section*{Legende der Diagramme}
\label{legende}
Wir werden ausschlie�lich Diagramme erl�utern, welche den Grad der Serverbelastung darstellen. Das sind jene Diagramme, in denen
die F�llh�he der Server-Queue und die Anzahl der abgewiesenen Feed-Requests pro Zeiteinheit dargestellt sind.\\

\importgnuplotps{Legende der Diagramme}{Abb:Legende_Diagramme}{legende}

Auf der y-Achse (siehe Abbildung \ref{Abb:Legende_Diagramme}) wird die Anzahl der Anfragen an einen RSS-Server dargestellt.
Unterhalb der Queue (\glqq $<$Anfragen in der Queue$>$\grqq{}) ist die F�llgr��e der Queue zu diesem Zeitpunkt zu sehen.
Oberhalb der Queue (\glqq $<$Verworfene Anfragen$>$\grqq{}) ist die Anzahl der abgewiesenen Nachrichten zu sehen. Die Darstellung beider Gr��en innerhalb
eines Diagramms ist dabei nicht ganz unproblematisch, da die F�llgr��e der Queue einen Zustand zum aktuellen Zeitpunkt repr�sentiert,
w�hrend sich die Anzahl der abgewiesenen Nachrichten �ber einen Zeitraum von einem Zeitpunkt in der
Vergangenheit bis zum aktuellen Zeitpunkt erstreckt. Da die Nachrichten in der Simulation aber sequenziell bearbeitet werden,
kummulieren sich die abgewiesenen Nachrichten
(Kurve �ber der Queue steigt), bis der Server eine n�chste Antwort aussendet (Kurve �ber der Queue bricht ab). Dadurch kann das
Verh�ltnis zwischen mittlerer Ankunftsrate der Anfragen und mittlerer Bearbeitungszeit gut nachvollzogen werden.\\
Im Bereich \glqq $<$Aktionsbereich$>$\grqq{} werden Aktionen angezeigt, die w�hrend der Simulation aufgetreten sind. Tabelle \ref{Tab:Aktionssymbole}
gibt eine �bersicht �ber die den Symbolen zugeordneten Aktionen mitsamt Beispielparametern.\\
\begin{table}
  \begin{center}
    \begin{tabular}{|rl|}
      \hline
      Aktionssymbol & Aktion \\
      \hline\hline
      &\\
      ST(1) & setze ServiceTimeFactor auf 1\\
      JB & Begin der Beitrittsphase: Subscriber beginnen,\\
      & dem Netzwerk beizutreten\\
      JE & Ende der Beitrittsphase\\
      CB(80,100) & Beginn der Churn-Phase: 80\% der Subscriber\\
      & werden innerhalb von 100 Sekunden ausgetauscht\\
      CE & Ende der Churn-Phase\\
      BS(50) & 50\% der Subscriber sind blockiert\\
      US & Subscriber sind nicht mehr blockiert\\
      SL(50) & 50\% der Subscriber verlassen das System\\
      SJ & Subscriber, die das System zuvor verlassen haben,\\
      & treten dem System wieder bei\\
      &\\
      \hline
    \end{tabular}
  \end{center}
  \caption{Aktionssymbole}
  \label{Tab:Aktionssymbole}
\end{table}

Die Kurven zeigen berechnete Mittelwerte und 95\%-Konfidenzintervalle um diese Mittelwerte. Die Mittelwerte bestimmen den Kurvenverlauf,
die Konfidenzintervalle die Strichst�rke der Kurve. So kann die Pr�zision der errechneten Mittelwerte leicht nachvollzogen werden.

%%% Local Variables: 
%%% mode: latex
%%% TeX-master: "diplomarbeit"
%%% End: 
