\chapter{Anhang}
\label{anhang}
\addcontentsline{toc}{section}{Herleitungen}
\section*{Herleitungen}
Hier finden sich die Herleitungen zu den Formeln zur Berechnung der mittleren roundtrip-times.
\subsection*{Mittlere roundtrip-time bei exponentieller Steigerung des $rto$}

Bei der Berechnung der mittleren roundtrip-time m�ssen wir zun�chst die m�glichen Zeitdifferenzen zwischen den Wiederholungen und Erhalt des Feeds aufsummieren.
$T_1$ bis $T_4$ in der Abbildung \ref{Abb:exp_steigerung_rto} sind als Beispiele die Zeitpunkte der einzelnen Aussendungen bzw. Wiederholungen,
$t_f$ ist der Zeitpunkt, zu dem der Feed den Subscriber erreicht. $i$ ist dabei die Anzahl der Wiederholungen (einschliesslich der ersten Aussendung
eines Feed-Requests).

\begin{picturehere}{3}{4.5}{\mbox{exp. Steigerung des $rto$}}{Abb:exp_steigerung_rto}
 
\begin{picture}(7,1)(1.5,-2.5)
  \put(0,0){\vector(1,0){9}}
  \put(0,-0.2){\line(0,1){0.4}}
  \put(0,-0.5){$T_1$}
  \put(1,-0.2){\line(0,1){0.4}}
  \put(1,-0.5){$T_2$}
  \put(3,-0.2){\line(0,1){0.4}}
  \put(3,-0.5){$T_3$}
  \put(7,-0.2){\line(0,1){0.4}}
  \put(7,-0.5){$T_4$}
  \put(8,-0.2){\line(0,1){0.4}}
  \put(8,-0.5){$t_f$}
  \put(7,0.3){$\overbrace{\hspace{1cm}}^{\varDelta t}$}
  \put(3,-0.65){$\underbrace{\hspace{5cm}}_{cpp\cdot 2^{i-1}+\varDelta t}$}
  \put(1,0.9){$\overbrace{\hspace{7cm}}^{cpp\cdot 2^{i-2}+cpp\cdot 2^{i-1}+\varDelta t}$}
  \put(0,-1.25){$\underbrace{\hspace{8cm}}_{cpp\cdot 2^{i-3}+cpp\cdot 2^{i-2}+cpp\cdot 2^{i-1}+\varDelta t}$}
  \put(9.8,0){$time$}
\end{picture}
\end{picturehere}

Wir erhalten folgende Summe:
\begin{equation}
  [\varDelta t]+[\varDelta t+cpp\cdot 2^{i-1}]+[\varDelta t+cpp\cdot 2^{i-1}+cpp\cdot 2^{i-2}]+\dots+[\varDelta t+cpp\cdot 2^{i-1}+\dots+cpp\cdot 2^1]
\end{equation}
\begin{equation}
  =i\cdot \varDelta t+(i-1)\cdot cpp\cdot 2^{i-1}+(i-2)\cdot cpp\cdot 2^{(i-2)}+\dots+1\cdot cpp\cdot 2^1
\end{equation}
\begin{equation}
  =i\cdot \varDelta t+cpp\sum^{i-1}_{k=1}2^kk.
\end{equation}

Um den Mittelwert zu erhalten, m�ssen wir durch $i$ teilen:

\begin{equation}
  \frac{i\cdot \varDelta t+cpp\sum^{i-1}_{k=1}2^kk}{i}
\end{equation}
\begin{equation}
  =\frac{cpp\sum^{i-1}_{k=1}2^kk}{i}+\varDelta t.
\end{equation}

\subsection*{Mittlere roundtrip-time bei polynomieller Steigerung des $rto$}
Analog zur Herleitung der mittleren roundtrip-time bei exponentieller Steigerung des $rto$ sind die Zeitdifferenzen wie in Abbildung \ref{Abb:poly_steigerung_rto}
dargestellt. Die Koeffizienten sind nun $i\dots 2$ (da bei der ersten Aussendung der $rto$ auf $rto:=2\cdot cpp$ gesetzt wird).
\begin{picturehere}{3}{4.5}{\mbox{polyn. Steigerung des $rto$}}{Abb:poly_steigerung_rto}
 
\begin{picture}(7,1)(1.5,-2.5)
  \put(0,0){\vector(1,0){9}}
  \put(0,-0.2){\line(0,1){0.4}}
  \put(0,-0.5){$T_1$}
  \put(1,-0.2){\line(0,1){0.4}}
  \put(1,-0.5){$T_2$}
  \put(3,-0.2){\line(0,1){0.4}}
  \put(3,-0.5){$T_3$}
  \put(6,-0.2){\line(0,1){0.4}}
  \put(6,-0.5){$T_4$}
  \put(8,-0.2){\line(0,1){0.4}}
  \put(8,-0.5){$t_f$}
  \put(6,0.3){$\overbrace{\hspace{2cm}}^{\varDelta t}$}
  \put(3,-0.65){$\underbrace{\hspace{5cm}}_{i\cdot cpp+\varDelta t}$}
  \put(1,0.9){$\overbrace{\hspace{7cm}}^{(i-1)\cdot cpp+i\cdot cpp+\varDelta t}$}
  \put(0,-1.25){$\underbrace{\hspace{8cm}}_{(i-2)\cdot cpp+(i-1)\cdot cpp+i\cdot cpp+\varDelta t}$}
  \put(9.8,0){$time$}
\end{picture}
\end{picturehere}

Die Berechnung der Summe ist wie folgt. Dabei greifen wir auf die Gleichungen
\begin{equation}
  \sum^n_{i=1}i=\frac{n(n+1)}{2}
\end{equation}
und
\begin{equation}
  \sum^n_{i=1}i^2=\frac{n(n+1)(2n+1)}{6}
\end{equation}
zur�ck:
\begin{equation}
  [\varDelta t]+[\varDelta t+i\cdot cpp]+[\varDelta t+i\cdot cpp+(i-1)\cdot cpp]+\dots+[\varDelta t+i\cdot cpp+\dots+2\cdot cpp]
\end{equation}
\begin{equation}
  =i\cdot \varDelta t+cpp\sum^{i-1}_{k=1}k(k+1)
\end{equation}
\begin{equation}
  =i\cdot \varDelta t+cpp\left(\sum^{i-1}_{k=1}k^2+\sum^{i-1}_{k=1}k\right)
\end{equation}
\begin{equation}
  =i\cdot \varDelta t+cpp\left(\frac{(i-1)i(2(i-1)+1)}{6}+\frac{(i-1)i}{2}\right)
\end{equation}
\begin{equation}
  =i\cdot \varDelta t+cpp\left(\frac{i(i-1)(2i-1)+3i(i-1)}{6}\right)
\end{equation}
\begin{equation}
  =i\cdot \varDelta t+cpp\left(\frac{i(i-1)(2i+2)}{6}\right).
\end{equation}

Als Mittelwert ergibt sich nun:
\begin{equation}
  \frac{i\cdot \varDelta t+cpp\left(\frac{i(i-1)(2i+2)}{6}\right)}{i}
\end{equation}
\begin{equation}
  =cpp\left(\frac{(i-1)(2i+2)}{6}\right)+\varDelta t.
\end{equation}

\addcontentsline{toc}{section}{Legende der Diagramme}
\section*{Legende der Diagramme}
\label{legende}
Hier findet sich eine Legende zur Erkl�rung der Diagramme.

\importgnuplotps{Legende der Diagramme}{Abb:Legende_Diagramme}{legende}

Auf der y-Achse wird die Anzahl der Anfragen an einen RSS-Server dargestellt. Unterhalb der Queue (``$<$Anfragen in der Queue$>$'') ist die F�llgr��e
der Queue zu diesem Zeitpunkt zu sehen.
Oberhalb der Queue (``$<$Verworfene Anfragen$>$'') ist die Anzahl der abgewiesenen Nachrichten zu sehen. Die Darstellung beider Gr��en innerhalb
eines Diagramms ist dabei nicht ganz unproblematisch, da die F�llgr��e der Queue einen Zustand zum aktuellen Zeitpunkt repr�sentiert,
w�hrend sich die Anzahl der abgewiesenen Nachrichten �ber einen Zeitraum von einem Zeitpunkt in der
Vergangenheit bis zum aktuellen Zeitpunkt erstreckt. Da die Nachrichten in der Simulation aber sequenziell bearbeitet werden,
kummulieren sich die abgewiesenen Nachrichten
(Kurve �ber der Queue steigt), bis der Server eine n�chste Antwort aussendet (Kurve �ber der Queue bricht ab). Dadurch kann das
Verh�ltnis zwischen mittlerer Ankunftsrate der Anfragen und mittlerer Bearbeitungszeit gut nachvollzogen werden.\\
Im Bereich ``$<$Aktionsbereich$>$'' werden Aktionen angezeigt, die w�hrend der Simulation aufgetreten sind. Tabelle \ref{Tab:Aktionssymbole}
gibt eine �bersicht �ber die den Symbolen zugeordneten Aktionen mitsamt Beispielparametern.\\
\begin{table}
  \begin{center}
    \begin{tabular}{|rl|}
      \hline
      Aktionssymbol & Aktion \\
      \hline\hline
      &\\
      ST(1) & setze ServiceTimeFactor auf 1\\
      JB & Begin der Beitrittsphase: Subscriber beginnen,\\
      & dem Netzwerk beizutreten\\
      JE & Ende der Beitrittsphase\\
      CB(80,100) & Beginn der Churn-Phase: 80\% der Subscriber\\
      & werden innerhalb von 100 Sekunden ausgetauscht\\
      CE & Ende der Churnphase\\
      BS(50) & 50\% der Subscriber sind blockiert\\
      US & Subscriber sind nicht mehr blockiert\\
      SL(50) & 50\% der Subscriber verlassen das System\\
      SJ & Subscriber, die das System zuvor verlassen haben,\\
      & treten dem System wieder bei\\
      &\\
      \hline
    \end{tabular}
  \end{center}
  \caption{Aktionssymbole}
  \label{Tab:Aktionssymbole}
\end{table}

Kurven zeigen berechnete Mittelwerte und 95\%-Konfidenzintervalle um diese Mittelwerte. Die Mittelwerte bestimmen den Kurvenverlauf,
die Konfidenzintervalle die Strichst�rke der Kurve. So kann leicht die Pr�zision der errechneten Mittelwerte nachvollzogen werden.

%%% Local Variables: 
%%% mode: latex
%%% TeX-master: "diplomarbeit"
%%% End: 
