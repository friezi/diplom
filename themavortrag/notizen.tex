\documentclass{article}
\usepackage{isolatin1}
\usepackage{latexsym}
\usepackage[german]{babel}
\usepackage[a4paper]{geometry}
\geometry{textwidth=18cm, textheight=24cm} 
\parindent0em
\pagenumbering{arabic}


\begin{document}
\section*{Outline}
�berblick:\\
- �berblick �ber zentrale Begriffe\\


Motivation:\\
- Erwartungen des Anwenders an das System\\
- Stand der Dinge\\
- Probleme, die er mit sich bringt\\

Umsetzung:\\
- welche Ans�tze kann man verfolgen, um bestehendes System mit Anspr�chen des Anwenders in Einklang zu bringen\\
\section*{Einleitung}
- �bersetzungen\\
- das am meisten verbreitete Pub/Sub-System im Netz\\
- RSS: nicht verteiltes Push sondern zentralisiertes Polling\\

{\bf Folie}
\section*{�berblick}
- Pub/Sub:\\

- RSS:\\
- im wesentlichen XML-basierte Dateien\\
- entahltene Informationen: z.B. Nachrichten-Schlagzeilen\\


\section*{Motivation}

- automatische Benachrichtigung �ber Feed-Updates\\
- Aktualit�t der Feeds\\
- Definition von Filtern auf h�herer Ebene zur:\\
-- individuellen Vorselektion der Feeds\\
-- anbieter�bergreifenden Auswahl von Feeds\\

\subsection*{Stand der Dinge: Folie}
\dots\\

Probleme:\\
Polling: gro�er Datenverkehr im Netz (3.5 GB New York Times)\\
-------- hohe server-seitige Last (Bsp. Server mit geringer Kapazit�t)\\
Filterdef. nur auf Anwenderseite

\subsection*{Ziel: Folie}

\section*{Umsetzung}
\subsection*{Ein Publisher}
\dots\\
Problem: Aktualit�t der Feeds, aufw�ndiger Auswahlprozess, Single Point of Failure

\subsection*{Aktualit�t vs. Server-Belastung}
\subsection*{mehrere Publisher}
- mindestens so viele, dass gewisse Aktualit�t erreicht wird\\
- h�chstens so viele, dass Server nicht �berlastet wird\\
- $\rightarrow$ optimaler Kompromiss soll erreicht werden\\

\subsection*{Frage: Folie}

\subsection*{Herausforderung}
- Koordination der Publisher\\
- Ermittlung Server-Auslastung\\
- Bestimmung des Update-Intervalls der Feeds\\
- (ev.) Bestimmung der Nachrichtenlaufzeiten\\

\subsection*{Erwartungen an Algorithmus}

- Hotspots vermeiden\\
- Broker automatisch Parameter einstellen\\
- dynamische Anpassung von Parametern\\
- selbst�ndige Beseitigung von Fehlern\\


\subsection*{Vorteile des Ansatzes}
- keine neue Server-Software  $\rightarrow$ L�sung integriert sich vollst�ndig\\
- ebenfalls Client-Software\\
- keine Neukonstruktion notwendig $\rightarrow$ REBECA, Filtertechniken
\end{document}
