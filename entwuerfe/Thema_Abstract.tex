\documentclass{article}
\usepackage[latin1]{inputenc}
\usepackage{latexsym}
\usepackage[german]{babel}
\usepackage[a4paper]{geometry}
\geometry{textwidth=18cm, textheight=24cm} 
\parindent0em
\pagenumbering{arabic}
\begin{document}

\section*{Integration von RSS mit verteiltem Publish/Subscribe}
\subsection*{Abstract}

Das am meisten verbreitete Publish/Subscribe-System im Internet basiert auf dem
Standard RSS (Rich Site Summary).  Im Gegensatz zu den am Lehrstuhl
untersuchten Systemen basiert es jedoch nicht auf verteiltem Push, sondern ist
zentralisiert und arbeitet mit Polling.  Es gibt also keine Message-Broker, die
Notifikationen empfangen und an die entsprechenden Subscriber weiterleiten.
Nachrichten werden lediglich anhand des Feeds ausgew�hlt -- Filterdefinitionen,
z.B. �ber den Inhalt einer Nachricht, werden nicht unterst�tzt.  Dagegen muss
ein Subscriber den entsprechenden Publisher eigenm�chtig kontaktieren, um den
aktuellen Stand der Nachrichten/Notifikationen (Feeds) bzw. die Notifikationen
selbst zu erhalten.  Dieses Vorgehen kann bereits bei relativ kleinen
Polling-Raten zur dauerhaften �berlastung der RSS-Server f�hren, weshalb
mittlerweile manche Server h�ufig pollende Nutzer blockieren.
\\\\
Ziel dieser Arbeit ist es, eine �berlastung des Servers zu verhindern und die Aktualit�t der Informationen beim Benutzer zu
erh�hen.  Damit dies erreicht werden kann, kooperieren die Nutzer, indem sie
die erhaltenen Informationen mittels eines Publish/Subscribe-Systems
miteinander austauschen.  Damit ist es den Nutzern au�erdem m�glich,
komplexere Filter zu definieren, die bei einfachem RSS nicht realisierbar sind.
\end{document}
