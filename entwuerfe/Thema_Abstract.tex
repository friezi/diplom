\documentclass{article}
\usepackage{isolatin1}
\usepackage{latexsym}
\usepackage[german]{babel}
\usepackage[a4paper]{geometry}
\geometry{textwidth=18cm, textheight=24cm} 
\parindent0em
\pagenumbering{arabic}
\begin{document}

\section*{Integration von RSS mit verteiltem Publish/Subscribe}
\subsection*{Abstract}

RSS ist ein Publish/Subscribe-System nicht im herk�mmlichen Sinne. Es gibt keine Message-Broker, die Notifikationen empfangen
und an die entsprechenden Subscriber weiterleiten, Filterdefinitionen werden nicht unterst�tzt. Dagegen muss ein Subscriber den entsprechenden
Publisher eigenm�chtig kontaktieren, um den aktuellen Stand der Nachrichten/Notifikationen (Feeds) bzw. die Notifikationen selbst zu erhalten. Dies kann
bei relativ kleinen Polling-Raten zur dauerhaften �berlastung der entsprechenden Server f�hren, weshalb manche Server stark-pollende Subscriber blocken.\\
Ziel dieser Arbeit ist es, ein selbskonfigurierendes
Overlay-Netzwerk aus Brokern/Subscribern zu konstruieren, bei dem die Broker f�r die Aktualisierung und Verteilung der Feeds verschiedener Publisher
an die Subscriber/Broker sorgen, um eine Lastverteilung in diesem Netzwerk zu erreichen.
Hierbei soll es den Subscribern m�glich sein, inhaltsbasierte Filter zu definieren, durch welche der M�glichkeitsraum zu
erhaltener RSS-Feeds erweitert bzw. eingeschr�nkt werden kann. Eine mit dem Publisher m�glichst synchronisierte Aktualisierung der Feeds auf
Subscriberebene soll erreicht werden, sprich hohe Pollingraten m�ssen ereicht werden k�nnen.

\end{document}
