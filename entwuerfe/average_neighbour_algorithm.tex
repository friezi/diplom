\documentclass{article}
\usepackage[latin1]{inputenc}
\usepackage{latexsym}
\usepackage[german]{babel}
\usepackage[a4paper]{geometry}
\geometry{textwidth=18cm, textheight=24cm} 
\parindent0em
\pagenumbering{arabic}
\begin{document}

\section*{Bestimmung der Nachbarn eines RSS-Subscribers}


Ein Subscriber subscribiert sich bei einem RSS-Publisher, indem er seinen ge\-w�nschten Channel und seine grobe Location mitteilt
(z.B. USA, Germany o. Westeuropa, ...). Sofern vorhanden erh�lt er vom Publisher bis zu vier Adressen weiterer Subscriber (potentielle Nachbarn, wenn
m�glich aus der gleichen Location ) und den aktuellen RSS-Feed (muss aktuelles Erstellungsdatum enthalten). Der Publisher merkt sich dabei
die Adresse des neuen Subscribers (grunds�tzlich beh�lt der Publisher die Adressen der letzten Subscriber, welche den aktuellen Feed von ihm erhalten
haben).
\\
M�chte ein Subscriber den aktuellen Feed erfragen, so sendet er eine Anfrage (TTL=2) an seine vier Nachbarn, dabei beinhaltet diese Anfrage den Channel,
das letzte aktuelle Erstellungsdatum, seine Adresse und die Adresse der Quelle der Anfrage (also ebenfalls seine).
Besitzt ein Nachbar einen aktuelleren Feed
(ist nat�rlich abh�ngig vom absoluten Datum, je nach Auffrischungsintervall des Publishers),
wird dieser an den Subscriber gesand (da Aktualit�t nicht vollkommen gew�hrleistet werden kann, k�nnen u.U. verschiedene Feeds an den Subscriber gesand
werden, wobei eine den aktuellsten Stand besitzt). Falls ein Nachbar keinen aktuelleren Feed besitzt, wird die Anfrage an alle weiteren vier Nachbarn
weitergeleitet, mit eigener Adresse und Quelladresse; der TTL wird dabei um eins erniedrigt. Jeder Empf�nger, der einen aktuelleren Feed besitzt, sendet
diesen direkt an den Anfragenden. Abgesehen davon senden die unmittelbaren Nachbarn die Adressen aller ihrer Nachbarn an den Anfragenden. Dieser
rekombiniert seine Liste der Nachbarn mit den neuen Werten. Erreicht den Subscriber kein aktuellerer Feed, wird dieser beim Publisher direkt erfragt,
welcher sich wiederum die Adresse des Subscribers merkt und ihm zus�tzlich vier Adressen potentieller Nachbarn zusendet. Auch in diesem Fall rekombiniert
der Subscriber seine Nachbarliste mit den neuen Adressen. Hat sich ein Nachbar unsubscribiert f�r eine Channel, so sendet er bei Anfrage diese Information
an den Anfragenden, welcher den bisherigen Nachbarn aus seine Liste entfernt.
\\\\
Mithilfe der �bersendeten Adressen k�nnen die Nachbarverbindungen dynamisch eingerichtet werden, das System ist selbstkonfigurierend. Da auch w�hrend
des Updatings von Nachrichtenfeeds Informationen �ber aktuelle Adressen und �ber nicht mehr bestehende mitgeteilt werden, ist das System selbsadaptierend
und selbststabilisierend. Rekombination von Nachbaradressen soll eine Optimierung der Verbindungen bez�glich Entfernung der
beteiligten Knoten gew�hrleisten. Somit ist das System selbstoptimierend.
\\\\
Maximale Anzahl kontaktierter Nachbarn MaxN berechnet sich durch:
\[MaxN=\sum_{i=1}^{TTL}n^i\]
 wobei n die maximale Anzahl von Nachbarn eines Knotens ist; in unserem Falle ist also $MaxN=20$, da n=4 und TTL=2.
Rekombination der Nachbaradressen:

Ziel ist eine Verbindung zu m�glichst nahe gelegenen Nachbarn. Werden neue Nachbaradressen �bersand, werden Entfernungen zu den neuen Adressen berechnet
(ping?) und alle Adressen neu kombiniert entsprechend der Entfernungen. Um zu vermeiden, dass abgeschlossene Gruppen entstehen, zwischen denen kein
Austausch stattfinden kann, kann wie folgt vorgegangen werden (Inspiration: genetische Algorithmen):
Von allen Kombinationen aus m�glichen vier Nachbarn werden die Mittelwerte der Entfernungen berechnet (ping-Zeiten). Nun wird aber nicht die Gruppe
mit dem minimalen Wert als L�sung betrachtet, sondern eine zuf�llige(?) Gruppe, welche bestimmt wird aus dem zweiten Viertel der in geordneter
aufsteigender Reihenfolge ermittelten Mittelwerte, deren Indizes die Gruppennummer bestimmen. Dies gew�hrleistet, dass bei sehr nahen Nachbarn auch
ein entfernter vorhanden sein muss. Gleichzeitig befindet sich die Mehrheit der Nachbarn in relativ nahr Entfernung, da ja die Gruppe aus einem
unteren Viertel bestimmt wurde.
Eine andere m�gliche Methode w�re, die zwei n�chsten Nachbarn festzuhalten und die anderen beiden neu zu bestimmen (Crossover). Im n�chsten Schritt
k�nnte ein Nachbar durch einen zuf�lligen anderen ausgetauscht werden (Mutation).
\\\\
Da in einem Netz, was die Small-world-property erf�llt, davon ausgegangen werden kann, dass es mindestens einen Knoten gibt, der die neueste
Information bereits hat, kommt es zu weniger Anfragen an den Publisher. Allerdings geht es ( da es sich nicht um ein Notifikationssystem handelt)
auf Kosten der Aktualit�t.

\newpage

\section*{Literaturliste}
\subsection*{Peer-To-Peer}
\begin{itemize}
\item Peer-to-peer information systems, Folien,  Karl Aberer, Manfred Hauswirth
\item P2P Overlay Broker Networks in an Event Based Middleware, P.Pietzuch, J.Bacon
\item PeerCQ: A Decentralized and Self-Configuring Peer-To-Peer Information-Monitoring System, Bugra Gedik, Ling Liu
\item A Definition of Peer-to-Peer Networking for the Classification of Peer-to-Peer Architectures and Applications, R�diger Schollmeier
\item Towards Adaptive, Resilient and Self-Organizing Peer-To-Peero Systems, Alberto Montresor, Hein Meling, �zalp Babaoglu
\subitem $\rightarrow$ genetic algorithms
\item A Genetic-Algorithm-Based Neighbour-Selection Strategie for Hybrid Peer-To-Peer Networks, Koo, Lee, Kannan
\end{itemize}
\subsection*{Genetic Algorithms}
\begin{itemize}
\item An Overview of Genetic Algorithms: Part 1, David Beasley, David R.Bull, Ralph R. Martin
\item Parallel Genetic Algorithms for Distributed Computing over Internet, Alessio Ceroni
\end{itemize}
\subsection*{PubSub}
\begin{itemize}
\item A Genetic Algorithm for Multicast-Mapping in Publish-Subscribe Systems, Mario Guimaraes, Luis Rodrigues
\item Self-Stabilizing Publish/Subscribe Systems: Algorithms and Evaluation, Gero M�hl, Michael Jaeger, ...
\end{itemize}
\subsection*{RSS}
\begin{itemize}
\item Client Behaviour and Feed Characteristics of RSS, a Publish-Subscribe System for Web Micronews, Honghzou Liou, V.Ramasubramaniam, E. G�n Sirer
\end{itemize}
\subsection*{andere}
\begin{itemize}
\item Self-Mamagement: The Solution to Complexity or Just another Problem?, K.Herrmann, G.M�hl, K.Geihs
\item Diplom H.P.
\item Diplom M.J�ger
\item BitTorrent-Protocol
\end{itemize}
\end{document}
